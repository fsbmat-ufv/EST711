\documentclass[12pt]{beamer}

\usepackage[brazil]{babel}
\usepackage[utf8]{inputenc}
\usepackage[T1]{fontenc}
\usepackage{animate}
\usepackage{amsbsy}
\usepackage{amsfonts}
\usepackage{amsmath}
\usepackage{amssymb}
\usepackage{amsthm}
\setbeamertemplate{theorems}[numbered] % to number
\usepackage[toc,page,title,titletoc]{appendix}
\usepackage{dsfont}
\usepackage{esvect}
\usepackage[labelfont=bf]{caption}
%\usepackage{subcaption}
\usepackage{float}
\usepackage[Glenn]{fncychap}%Sonny %Conny %Lenny %Glenn %Renje %Bjarne %Bjornstrup
\usepackage{graphicx}
\usepackage{subfig}
\usepackage{indentfirst}%Para indentar os paragrafos automáticamente
\usepackage{lipsum}
\usepackage{longtable}
\usepackage{mathtools}
\usepackage{listings}%Inserir codigo do R no latex
\usepackage{multirow}
\usepackage{multicol}
\usepackage{csquotes}
\usepackage[citestyle=authoryear,maxcitenames=2,terseinits=true,natbib=true, style=abnt, maxbibnames=99]{biblatex}
\addbibresource{../Referencias/Referencias.bib}
\usepackage[figuresright]{rotating}
\usepackage{spalign}
\usepackage{pgfplots}
\pgfplotsset{compat=1.17}
\usepackage{tikz}
\usepackage{fontawesome}
\usepackage{color, colortbl}
\usepackage{url}
\usepackage{cancel}
\usepackage{accents}
\usepackage{bm}
\usepackage{ragged2e}%para justificar o texto dentro de algum ambiente
\definecolor{Gray}{gray}{0.9}
\definecolor{LightCyan}{rgb}{0.88,1,1}


\usepackage[all]{xy}
\usepackage{hyperref,bookmark}
\hypersetup{
  colorlinks=true,
  linkcolor=blue,
  citecolor=red,
  filecolor=blue,
  urlcolor=blue,
}

\usetheme{Madrid}
\usecolortheme[RGB={193,0,0}]{structure}

%\setbeamertemplate{footline}[frame number]
%\setbeamertemplate{footline}[text line]{%
%  \parbox{\linewidth}{\vspace*{-8pt}\hfill\date{}\hfill\insertshortauthor\hfill\insertpagenumber}}
\beamertemplatenavigationsymbolsempty
\renewcommand{\vec}[1]{\mbox{\boldmath$#1$}}
\newtheorem{Teorema}{Teorema}
\newtheorem{Proposicao}{Proposição}
\newtheorem{definicao}{Definição}
\newtheorem{Corolario}{Corolário}
\newtheorem{Demonstracao}{Demonstração}
\newcommand{\bx}{\ensuremath{\bar{x}}}
\newcommand{\Ho}{\ensuremath{H_{0}}}
\newcommand{\Hi}{\ensuremath{H_{1}}}
\newcommand{\at}[2][]{#1|_{#2}}
\newcommand\xuparrow[1][2ex]{%
   \mathrel{\rotatebox{-90}{$\xleftarrow{\rule{#1}{0pt}}$}}
}
\apptocmd{\frame}{}{\justifying}{} % Allow optional arguments after frame.

\makeatletter
\setbeamertemplate{footline}
{
  \leavevmode%
  \hbox{%
  \begin{beamercolorbox}[wd=.3\paperwidth,ht=2.25ex,dp=1ex,center]{author in head/foot}%
    \usebeamerfont{author in head/foot}\mytext
  \end{beamercolorbox}%
  \begin{beamercolorbox}[wd=.3\paperwidth,ht=2.25ex,dp=1ex,center]{title in head/foot}%
    \usebeamerfont{title in head/foot}\mytextt
  \end{beamercolorbox}%
  \begin{beamercolorbox}[wd=.35\paperwidth,ht=2.25ex,dp=1ex,right]{site in head/foot}%
    \usebeamerfont{site in head/foot}\mytexttt\hspace*{2em}
    \insertframenumber{} / \inserttotalframenumber\hspace*{2ex} 
  \end{beamercolorbox}}%
  \vskip0pt%
}
\makeatother

\providecommand{\arcsin}{} \renewcommand{\arcsin}{\hspace{2pt}\textrm{arcsen}}
\providecommand{\sin}{} \renewcommand{\sin}{\hspace{2pt}\textrm{sen}}
\newcommand{\N}{\rm I\!N}
\newcommand{\I}{\rm I\!I}
\newcommand{\R}{\rm I\!R}
\newcommand{\Sim}{\overset{\text{iid}}{\sim}}
\newcommand{\Lim}{{\displaystyle \lim_{n\to\infty}}}
\newcommand{\LimInf}{{\displaystyle \liminf_{n\to\infty}}}
\newcommand{\rightLim}{\xrightarrow[n\rightarrow\infty]{}}
\newcommand{\Sumi}{{\displaystyle \sum_{i=1}^{n}}}
\newcommand{\Int}{{\displaystyle \int_{-\infty}^{+\infty}}}
\newcommand{\ConvD}{\overset{D}{\rightarrow}}
\newcommand{\ConvP}{\overset{P}{\rightarrow}}
\newcommand{\Prodi}{{\displaystyle \prod_{i=1}^{n}}}
\newcommand{\SetaUP}[2]{\underset{\mathclap{\substack{\xuparrow[30pt] \\ #1}}}{#2}}
%\newcommand{\SetaInclinada}[2]{\underset{\mathclap{\substack{\rotatebox{135}{\xuparrow[30pt] \\ #1}}}}{#2}}
\newcommand{\Home}{\begin{tikzpicture}
\node[scale=2] at (3,4) {\text{Para}~\faHome};
\end{tikzpicture}}
\newcommand{\vecX}{\boldsymbol{X}}
\newcommand{\Implica}[1]{\xRightarrow{#1}}
\newcommand{\SeSe}{\iff}
\newcommand{\EscoreA}{\dfrac{\partial}{\partial\theta}\log{f(x,\theta)}}
\newcommand{\EscoreB}{\dfrac{\partial^{2}}{\partial\theta^{2}}\log{f(x,\theta)}}
\newcommand{\cqd}{\text{cqd}~\blacksquare}
\newcommand{\seqX}{$X_{1},\ldots,X_{n}$}
\newcommand{\seqY}{$Y_{1},\ldots,Y_{n}$}
\newcommand{\tend}[1]{\hbox{\oalign{$\bm{#1}$\crcr\hidewidth$\scriptscriptstyle\bm{\sim}$\hidewidth}}}

%\newtheorem{Teorema}{Teorema}
%\newtheorem{Proposicao}{Proposição}
%\newtheorem{Definicao}{Definição}
%\newtheorem{Corolario}{Corolário}
%\newtheorem{Demonstracao}{Demonstração}

\titlegraphic{\hspace*{8cm}\href{https://fsbmat-ufv.github.io/}{\includegraphics[width=2cm]{figs/mylogo.png}}
}

%Continuar a numeracao em slides diferentes
\newcounter{saveenumi}
\newcommand{\seti}{\setcounter{saveenumi}{\value{enumi}}}
\newcommand{\conti}{\setcounter{enumi}{\value{saveenumi}}}

\resetcounteronoverlays{saveenumi}

% Layout da pagina
\hypersetup{pdfpagelayout=SinglePage}

%Para o \pause funcionar dentro do ambiente align
\makeatletter
\let\save@measuring@true\measuring@true
\def\measuring@true{%
  \save@measuring@true
  \def\beamer@sortzero##1{\beamer@ifnextcharospec{\beamer@sortzeroread{##1}}{}}%
  \def\beamer@sortzeroread##1<##2>{}%
  \def\beamer@finalnospec{}%
}
\makeatother


\title{Inferência Estatística II}
\author{Prof. Fernando de Souza Bastos\texorpdfstring{\\ fernando.bastos@ufv.br}{}}
\institute{Departamento de Estatística\texorpdfstring{\\ Programa de Pós-Graduação em Estatística Aplicada e Biometria}\texorpdfstring{\\ Universidade Federal de Viçosa}{}\texorpdfstring{\\ Campus UFV - Viçosa}{}}
\date{}
\newcommand\mytext{Aula 9}
\newcommand\mytextt{Fernando de Souza Bastos}
\newcommand\mytexttt{\url{https://est711.github.io/}}


\begin{document}
%\SweaveOpts{concordance=TRUE}

\frame{\titlepage}

\begin{frame}{}
\frametitle{\bf Sumário}
\tableofcontents
\end{frame}

\section{Caso Multiparamétrico}
\begin{frame}{}
\begin{definicao}
\justifying
Seja $X_{1},\ldots, X_{n}$ variáveis aleatórias iid com densidade ou fdp $f(x,\theta),~\theta \in \Omega\subset \R^{p}.$ Seja $\mathcal{S}$ o suporte de $X$ e $Y=(Y_{1},\ldots,Y_{m})^{\top}$ um vetor de estatísticas m-dimensional, isto é, $Y_{i}=u_{i}(X_{1},\ldots, X_{n}), i=1,\ldots,m.$ Denote a densidade conjunta ou fdp conjunta por $f_{Y}(y;\theta), y\in\R^{m}.$ O vetor de estatísticas $Y$ é conjuntamente suficiente para $\theta$ se 
\begin{align*}
    \dfrac{\Prodi f(x_{i},\theta)}{f_{Y}(y,\theta)}=H(x_{1},\ldots,x_{n}),\forall\theta\in \Omega, \forall x_{i}\in \mathcal{S}
\end{align*}
não depende de $\theta.$
\end{definicao}
\end{frame}

\begin{frame}{}
\begin{block}{}
\justifying
O critério da fatoração pode ser estendido para o caso multiparamétrico. Ou seja, $Y$ é conjuntamente suficiente para $\theta,$ se existem duas funções não negativas $k_{1}$ e $k_{2}$ tais que 
\begin{align*}
    \Prodi f(x_{i},\theta)=k_{1}(y,\theta)k_{2}(x_{1},\ldots,x_{n}), y=(y_{1},\ldots,y_{m}).
\end{align*}
O conceito de uma família completa de densidades ou fdp's é generalizado como segue nos próximos slides.
\end{block}
\end{frame}

\begin{frame}{}
\begin{block}{}
\justifying
Seja $\{f(v_{1},\ldots,v_{k},\theta);\theta\in\Omega\}$ uma família de densidades ou fdp's conjuntas de $k$ variáveis aleatórias $v_{1},\ldots,v_{k}.$ Seja $u(v_{1},\ldots,v_{k})$ uma função tal que $E(u(v_{1},\ldots,v_{k}))=0, \forall \theta\in\Omega,$ implica que $u(v_{1},\ldots,v_{k})=0$ exceto em um conjunto de probabilidade zero. Neste caso, dizemos que a familia $f$ é completa.
\end{block}
\end{frame}

\begin{frame}{}
\begin{block}{}
\justifying
Os teoremas de Rao-Blackwell e Lehman-Scheffé podem ser generalizados para este caso. Suponha que $\delta=g(\theta)$ é o vetor de parâmetros de interesse. Suponha que $Y$ é conjuntamente suficiente e completa para $\theta.$ Seja $T=T(Y)$ uma função de $Y$ tal que $E(T)=\delta.$ Então, $T$ é o ENVVUM de $\delta,$ pelo teorema de Lehman-Scheffé. $T$ também é suficiente e completa para $\delta.$ Daqui, em diante, suponha que o número de estatísticas suficientes $m$ seja igual a dimensão de $\theta$ (ou seja, $m=p$).
\end{block}
\end{frame}

\begin{frame}{}
\begin{definicao}
\justifying
Sejam $X_{1},\ldots,X_{m}$ uma amostra aleatória de uma distribuição com densidade ou fp $f(X,\theta),\theta \in \Omega^{p},$ dada por 
\begin{align*}
    f(x,\theta)=\exp{{\displaystyle \sum_{j=1}^{m}p_{j}(\theta)k_{j}(x)+S(x)+q(\theta_{1},\ldots,\theta_{m})}},x\in\mathcal{S}
\end{align*}
Se $X$ for contínua, assuma que $\mathcal{S}=(a,b),~a$ e $b$ podendo ser $-\infty$ e $\infty,$ respectivamente. Se $X$ for discreta, assuma $\mathcal{S}=\{a_{1},\ldots,a_{n},\ldots\}.$ Então dizemos que esta densidade ou função de probabilidade pertence a família exponencial. 
\end{definicao}
\end{frame}

\begin{frame}{}
\begin{block}{}
\justifying
Adicionalmente, dizemos que a família definida no slide anterior pertence a família exponencial regular se,
\begin{enumerate}
    \item O suporte $\mathcal{S}$ não depende do vetor de parâmetros $\theta.$
    \item $\Omega$ contém um retângulo m-dimensional aberto, não vazio.
    \item $P_{j}(\theta),j=1,\ldots,m$ são funções não triviais, independentes e contínuas.
    \item 
    \begin{enumerate}
        \item[a)] Se $X$ é uma variável aleatória contínua para $x\in(a,b)$ e nenhuma das funções $k_{j}$ é função linear das outras e $S(X)$ é uma função contínua para $x\in (a,b).$
        \item[b)] Se $X$ é v.a.d, $k_{j}(x),j=1,\ldots,m$ são funções não triviais de $x$ e nenhuma é combinação linear das outras.
    \end{enumerate}
\end{enumerate}
\end{block}
\end{frame}

\begin{frame}{}
\begin{block}{}
\justifying
Seja $X_{1},\ldots,X_{n}$ uma amostra aleatória de uma variável aleatória $X$ com distribuição pertencente a família exponencial regular. $f(\cdot,\theta)$ é a densidade ou f.p. de $X.$ Assim, 
\begin{align*}
    \Prodi f(x,\theta)&=\Prodi \exp{\Big\{{\displaystyle \sum_{j=1}^{m}p_{j}(\theta)k_{j}(x_{i})+S(x_{i})+q(\theta_{1},\ldots,\theta_{m})}\Big\}}\\
    &=\exp{\Big\{{\displaystyle \sum_{j=1}^{m}p_{j}(\theta)\sum_{i=1}^{n}k_{j}(x_{i}) + \sum_{i=1}^{n}S(x_{i}) + nq(\theta_{1},\ldots,\theta_{m})}\Big\}}\\
    &=\exp{\Big\{\sum_{i=1}^{n}S(x_{i})\Big\}}\exp{\Big\{{\displaystyle \sum_{j=1}^{m}p_{j}(\theta)y_{j} + nq(\theta_{1}, \ldots, \theta_{m})} \Big\}},    
\end{align*}
em que $y_{j}=\Sumi k_{j}(x_{i}),j=1,\ldots,m.$ Pelo critério da fatoração, $(y_{1},\ldots,y_{m})^{\top}=y$ é conjuntamente suficiente para $\theta.$
\end{block}
\end{frame}

\begin{frame}{}
\begin{block}{}
\justifying
Podemos verificar que a densidade ou fdp conjunta de $Y$ pode ser escrita como
\begin{align*}
    R(y)\exp{\Big\{{\displaystyle \sum_{j=1}^{m}p_{j}(\theta)y_{j}+nq(\theta_{1}, \ldots, \theta_{m})}\Big\}}
\end{align*}
e obter que $Y_{1},\ldots,Y_{m}$ é um vetor de estatísticas conjuntamente suficiente e completa para $\theta,$ quando $n>m.$
\end{block}
\end{frame}

\begin{frame}{Exemplo}
\vspace{-0.2cm}
\begin{block}{}
\justifying
$X_{1},\ldots,X_{n}\Sim N(\theta_{1},\theta_{2}),\theta_{1}\in \R$ e $\theta_{2}>0.$
\begin{align*}
    f(x_{i},\theta_{1},\theta_{2})&=\dfrac{1}{\sqrt{2\pi\theta_{2}}}\exp{-\dfrac{(x_{i}-\theta_{1})^{2})}{2\theta_{2}}}\\
    &=\exp{\Big\{-\frac{x_{i}^{2}}{2\theta_{2}}+\frac{x_{i}\theta_{1}}{\theta_{2}}-\frac{\theta_{1}^{2}}{2\theta_{2}}-\log{\sqrt{2\pi\theta_{2}}}\Big\}}
\end{align*}
\pause
%\vspace{-0.8cm}
\begin{align*}
k_{1}(x)=x^{2},~p_{1}(\theta)=-\frac{1}{2\theta_{2}},~k_{2}(x)=x~\text{e}~p_{2}(\theta)=\frac{\theta_{1}}{\theta_{2}}
\end{align*}
A distribuição normal $N(\theta_{1},\theta_{2})$ pertence a família exponencial regular, assim, $Y_{1}=\Sumi x_{i}^{2}$ e $Y_{2}=\Sumi x_{i}$ são estatísticas conjuntamente suficientes para $\theta=(\theta_{1},\theta_{2})$
\end{block}
\end{frame}

\begin{frame}{}
\begin{block}{}
\justifying
Assim, se definirmos $Z_{1}=\dfrac{Y_{2}}{n}=\Bar{X}$ e $Z_{2}=\dfrac{Y_{1}-\frac{Y_{2}^{2}}{n}}{n-1}=\dfrac{\sum (x_{i}-\Bar{x})}{n-1}$ com $E(Z_{1})=\theta_{1}$ e $E(Z_{2})=\theta_{2}^{2},~Z_{1}$ e $Z_{2}$ são, pelo teorema de Lehmann-Scheffé os ENVVUM para $\theta_{1}$ e $\theta_{2}^{2},$ respectivamente.
\end{block}
\end{frame}

\begin{frame}{}
\begin{block}{\Home}
\justifying
\begin{itemize}
    \item \textbf{Exercícios da seção 7.7:} 2,3,5,6,7,10,11,12 e 13.
\end{itemize}
\nocite{hogg, casella2021statistical, bolfarine}
\end{block}
\end{frame}

\begin{frame}[allowframebreaks]
\frametitle{\bf Referências}
\printbibliography
\end{frame}


\end{document}
