\documentclass[12pt]{beamer}

\usepackage[brazil]{babel}
\usepackage[utf8]{inputenc}
\usepackage[T1]{fontenc}
\usepackage{animate}
\usepackage{amsbsy}
\usepackage{amsfonts}
\usepackage{amsmath}
\usepackage{amssymb}
\usepackage{amsthm}
\setbeamertemplate{theorems}[numbered] % to number
\usepackage[toc,page,title,titletoc]{appendix}
\usepackage{dsfont}
\usepackage{esvect}
\usepackage[labelfont=bf]{caption}
%\usepackage{subcaption}
\usepackage{float}
\usepackage[Glenn]{fncychap}%Sonny %Conny %Lenny %Glenn %Renje %Bjarne %Bjornstrup
\usepackage{graphicx}
\usepackage{subfig}
\usepackage{indentfirst}%Para indentar os paragrafos automáticamente
\usepackage{lipsum}
\usepackage{longtable}
\usepackage{mathtools}
\usepackage{listings}%Inserir codigo do R no latex
\usepackage{multirow}
\usepackage{multicol}
\usepackage{csquotes}
\usepackage[citestyle=authoryear,maxcitenames=2,terseinits=true,natbib=true, style=abnt, maxbibnames=99]{biblatex}
\addbibresource{../Referencias/Referencias.bib}
\usepackage[figuresright]{rotating}
\usepackage{spalign}
\usepackage{pgfplots}
\pgfplotsset{compat=1.17}
\usepackage{tikz}
\usepackage{fontawesome}
\usepackage{color, colortbl}
\usepackage{url}
\usepackage{cancel}
\usepackage{accents}
\usepackage{bm}
\usepackage{ragged2e}%para justificar o texto dentro de algum ambiente
\definecolor{Gray}{gray}{0.9}
\definecolor{LightCyan}{rgb}{0.88,1,1}


\usepackage[all]{xy}
\usepackage{hyperref,bookmark}
\hypersetup{
  colorlinks=true,
  linkcolor=blue,
  citecolor=red,
  filecolor=blue,
  urlcolor=blue,
}

\usetheme{Madrid}
\usecolortheme[RGB={193,0,0}]{structure}

%\setbeamertemplate{footline}[frame number]
%\setbeamertemplate{footline}[text line]{%
%  \parbox{\linewidth}{\vspace*{-8pt}\hfill\date{}\hfill\insertshortauthor\hfill\insertpagenumber}}
\beamertemplatenavigationsymbolsempty
\renewcommand{\vec}[1]{\mbox{\boldmath$#1$}}
\newtheorem{Teorema}{Teorema}
\newtheorem{Proposicao}{Proposição}
\newtheorem{definicao}{Definição}
\newtheorem{Corolario}{Corolário}
\newtheorem{Demonstracao}{Demonstração}
\newcommand{\bx}{\ensuremath{\bar{x}}}
\newcommand{\Ho}{\ensuremath{H_{0}}}
\newcommand{\Hi}{\ensuremath{H_{1}}}
\newcommand{\at}[2][]{#1|_{#2}}
\newcommand\xuparrow[1][2ex]{%
   \mathrel{\rotatebox{-90}{$\xleftarrow{\rule{#1}{0pt}}$}}
}
\apptocmd{\frame}{}{\justifying}{} % Allow optional arguments after frame.

\makeatletter
\setbeamertemplate{footline}
{
  \leavevmode%
  \hbox{%
  \begin{beamercolorbox}[wd=.3\paperwidth,ht=2.25ex,dp=1ex,center]{author in head/foot}%
    \usebeamerfont{author in head/foot}\mytext
  \end{beamercolorbox}%
  \begin{beamercolorbox}[wd=.3\paperwidth,ht=2.25ex,dp=1ex,center]{title in head/foot}%
    \usebeamerfont{title in head/foot}\mytextt
  \end{beamercolorbox}%
  \begin{beamercolorbox}[wd=.35\paperwidth,ht=2.25ex,dp=1ex,right]{site in head/foot}%
    \usebeamerfont{site in head/foot}\mytexttt\hspace*{2em}
    \insertframenumber{} / \inserttotalframenumber\hspace*{2ex} 
  \end{beamercolorbox}}%
  \vskip0pt%
}
\makeatother

\providecommand{\arcsin}{} \renewcommand{\arcsin}{\hspace{2pt}\textrm{arcsen}}
\providecommand{\sin}{} \renewcommand{\sin}{\hspace{2pt}\textrm{sen}}
\newcommand{\N}{\rm I\!N}
\newcommand{\I}{\rm I\!I}
\newcommand{\R}{\rm I\!R}
\newcommand{\Sim}{\overset{\text{iid}}{\sim}}
\newcommand{\Lim}{{\displaystyle \lim_{n\to\infty}}}
\newcommand{\LimInf}{{\displaystyle \liminf_{n\to\infty}}}
\newcommand{\rightLim}{\xrightarrow[n\rightarrow\infty]{}}
\newcommand{\Sumi}{{\displaystyle \sum_{i=1}^{n}}}
\newcommand{\Int}{{\displaystyle \int_{-\infty}^{+\infty}}}
\newcommand{\ConvD}{\overset{D}{\rightarrow}}
\newcommand{\ConvP}{\overset{P}{\rightarrow}}
\newcommand{\Prodi}{{\displaystyle \prod_{i=1}^{n}}}
\newcommand{\SetaUP}[2]{\underset{\mathclap{\substack{\xuparrow[30pt] \\ #1}}}{#2}}
%\newcommand{\SetaInclinada}[2]{\underset{\mathclap{\substack{\rotatebox{135}{\xuparrow[30pt] \\ #1}}}}{#2}}
\newcommand{\Home}{\begin{tikzpicture}
\node[scale=2] at (3,4) {\text{Para}~\faHome};
\end{tikzpicture}}
\newcommand{\vecX}{\boldsymbol{X}}
\newcommand{\Implica}[1]{\xRightarrow{#1}}
\newcommand{\SeSe}{\iff}
\newcommand{\EscoreA}{\dfrac{\partial}{\partial\theta}\log{f(x,\theta)}}
\newcommand{\EscoreB}{\dfrac{\partial^{2}}{\partial\theta^{2}}\log{f(x,\theta)}}
\newcommand{\cqd}{\text{cqd}~\blacksquare}
\newcommand{\seqX}{$X_{1},\ldots,X_{n}$}
\newcommand{\seqY}{$Y_{1},\ldots,Y_{n}$}
\newcommand{\tend}[1]{\hbox{\oalign{$\bm{#1}$\crcr\hidewidth$\scriptscriptstyle\bm{\sim}$\hidewidth}}}

%\newtheorem{Teorema}{Teorema}
%\newtheorem{Proposicao}{Proposição}
%\newtheorem{Definicao}{Definição}
%\newtheorem{Corolario}{Corolário}
%\newtheorem{Demonstracao}{Demonstração}

\titlegraphic{\hspace*{8cm}\href{https://fsbmat-ufv.github.io/}{\includegraphics[width=2cm]{figs/mylogo.png}}
}

%Continuar a numeracao em slides diferentes
\newcounter{saveenumi}
\newcommand{\seti}{\setcounter{saveenumi}{\value{enumi}}}
\newcommand{\conti}{\setcounter{enumi}{\value{saveenumi}}}

\resetcounteronoverlays{saveenumi}

% Layout da pagina
\hypersetup{pdfpagelayout=SinglePage}

%Para o \pause funcionar dentro do ambiente align
\makeatletter
\let\save@measuring@true\measuring@true
\def\measuring@true{%
  \save@measuring@true
  \def\beamer@sortzero##1{\beamer@ifnextcharospec{\beamer@sortzeroread{##1}}{}}%
  \def\beamer@sortzeroread##1<##2>{}%
  \def\beamer@finalnospec{}%
}
\makeatother


\title{Inferência Estatística II}
\author{Prof. Fernando de Souza Bastos\texorpdfstring{\\ fernando.bastos@ufv.br}{}}
\institute{Departamento de Estatística\texorpdfstring{\\ Programa de Pós-Graduação em Estatística Aplicada e Biometria}\texorpdfstring{\\ Universidade Federal de Viçosa}{}\texorpdfstring{\\ Campus UFV - Viçosa}{}}
\date{}
\newcommand\mytext{Aula 6}
\newcommand\mytextt{Fernando de Souza Bastos}
\newcommand\mytexttt{\url{https://est711.github.io/}}


\begin{document}
%\SweaveOpts{concordance=TRUE}

\frame{\titlepage}

\begin{frame}{}
\frametitle{\bf Sumário}
\tableofcontents
\end{frame}

\section{Propriedades de Uma Estatística Suficiente}
\begin{frame}{Propriedades de Uma Estatística Suficiente}
\frametitle{}
\begin{block}{}
\justifying
Seja que $X_1, X_2, \ldots, X_n$ seja uma amostra aleatória de uma variável aleatória com função de densidade ou função de massa de probabilidade $f(x; \theta),~\theta \in \Omega$. \textbf{Observemos que uma estatística suficiente não é única}. Seja $Y_1 = u_1(X_1, X_2, \ldots, X_n)$ uma estatística suficiente para $\theta.$ Denote por $f_{Y_{1}}(y,\theta)$ a fdp de $Y_{1}.$ Seja $Y_2 = g(Y_1)$ uma estatística, em que $g$ é uma função bijetora, então pela definição de Estatística Suficiente temos,

\begin{align*}
 \dfrac{f(x_1; \theta)f(x_2; \theta) \ldots f(x_n; \theta)}{f_{Y_{1}}(y,\theta)} &= H(x_1, x_2, \ldots, x_n)\\
 \Rightarrow f(x_1; \theta)f(x_2; \theta) \ldots f(x_n; \theta) &= f_{Y_{1}}(y,\theta)H(x_1, x_2, \ldots, x_n)\\
\end{align*}
\end{block}
\end{frame}

\begin{frame}{}
\frametitle{}
\begin{block}{}
\justifying
Pelo teorema de Neyman (Critério da fatoração), 
\begin{align*}
f(x_1; \theta)f(x_2; \theta) \ldots f(x_n; \theta) &= k_{1}(y_{1},\theta)k_{2}(x_1, x_2, \ldots, x_n)\\
&=k_{1}(g^{-1}(y_{2}),\theta)k_{2}(x_1, x_2, \ldots, x_n)\\
&=k_{1}^{*}(y_{2},\theta)k_{2}(x_1, x_2, \ldots, x_n),
\end{align*}
logo $Y_2$ também é uma estatística suficiente para $\theta$.
\end{block}
\end{frame}

\begin{frame}{Propriedades Básicas}
\frametitle{}
\begin{block}{}
\justifying
Das propriedades básicas de Esperança Condicional, segue que, se $X_1$ e $X_2$ são variáveis aleatórias tais que a variância de $X_2$ existe, então

\begin{align*}
E[X_2] &= E[E(X_2|X_1)] \\
\text{Var}(X_2) &= \text{Var}[E(X_2|X_1)]+\underbrace{E[\text{Var}(X_2|X_1)]}_{\geq 0}\\
\Rightarrow \text{Var}(X_2) &\geq \text{Var}[E(X_2|X_1)].
\end{align*}
\end{block}
\end{frame}

\begin{frame}{Propriedades Básicas}
\frametitle{}
\begin{block}{}
\justifying
vamos considerar a estatística suficiente $Y_1$ como $X_1$ e $Y_2$, uma estatística não-viesada de $\theta$, como $X_2$. Assim, $X_2$ é um estimador não viesado de $\theta$ e $X_1$ uma estatística suficiente para $\theta$. Defina, $\xi(y_{1})=E(Y_2|Y_1=y_1).$ Logo, podemos escrever $\xi(Y_{1})=E(Y_2|Y_1)$ e,
\begin{align*}
 E[\xi(Y_1)]&= E(E(Y_2|Y_1))=E(Y_2) = \theta \\
\text{Var}[\xi(Y_1)]&=\text{Var}(E(Y_2|Y_1)) \leq \text{Var}(Y_2).
\end{align*}

Esses resultados podem ser enunciados como um teorema.
\end{block}
\end{frame}

\section{Teorema de Rao-Blackwell}
\begin{frame}{Teorema de Rao-Blackwell}
\frametitle{}
\begin{Teorema}
\justifying
Seja $X_1, X_2, \ldots, X_n$, uma amostra aleatória de uma distribuição (contínua ou discreta) com função de densidade ou função de massa de probabilidade $f(x; \theta),~\theta \in \Omega$. Seja $Y_1 = u_1(X_1, X_2, \ldots, X_n)$ uma estatística suficiente para $\theta$ e $Y_2 = u_2(X_1, X_2, \ldots, X_n)$ um estimador não-viesado de $\theta$. Considere, ainda, $E(Y_2|y_1) = \xi(y_1).$ Então $\xi(Y_1)$ é um estimador não-viesado de $\theta$ e sua variância é menor ou igual à de $Y_2$.
\end{Teorema}
\pause
\begin{block}{Demonstração}
\justifying
Ver slide anterior de propriedades básicas!
\end{block}
\end{frame}

\begin{frame}{}
\frametitle{}
\begin{block}{}
\justifying
Este teorema nos diz que, na busca por um EMVU (Estimador de Mínima Variância Não-viesado) de um parâmetro, podemos restringir essa busca a funções da estatística suficiente. Se começarmos com um estimador não-viesado $Y_2$, podemos sempre melhorá-lo calculando $E(Y_2|y_1) = \xi(y_1)$, de modo que $\xi(Y_1)$ seja um estimador não-viesado com uma variância menor que a de $Y_2$.

%Após o Teorema 7.3.1, muitos estudantes acreditam que é necessário encontrar primeiro algum estimador não-viesado $Y_2$ na busca por $\xi(Y_1)$, um estimador não-viesado de $\theta$ baseado na estatística suficiente $Y_1$. Isso não é o caso, e o Teorema 7.3.1 simplesmente nos convence de que podemos restringir nossa busca por um melhor estimador a funções de $Y_1$. Além disso, existe uma conexão entre estatísticas suficientes e estimadores de máxima verossimilhança, conforme mostrado no seguinte teorema:
\end{block}
\end{frame}

\begin{frame}{Teorema}
\frametitle{}
\begin{Teorema}
\justifying
Sejam  $X_1, X_2, \ldots, X_n$  uma amostra aleatória de uma distribuição com função de densidade ou função de massa de probabilidade  $f(x; \theta), \theta \in \Omega.$  Se uma estatística suficiente  $Y_1 = u_1(X_1, X_2, \ldots, X_n)$  para  $\theta$  existe e se um estimador de máxima verossimilhança  $\hat{\theta}$  para  $\theta$  também existe de forma única, então  $\hat{\theta}$  é uma função da estatística suficiente $Y_1 = u_1(X_1, X_2, \ldots, X_n).$
\end{Teorema}
\end{frame}

\begin{frame}{Demonstração}
\frametitle{}
\begin{block}{}
\justifying
Seja $f_{Y_1}(y_1; \theta)$ a função de densidade ou função de massa de probabilidade de $Y_1$. Então, pela definição de suficiência (teorema de Neyman), a função de verossimilhança
\begin{align*}
L(\theta) &= f(x_1; \theta)f(x_2; \theta) \ldots f(x_n; \theta)\\ 
&= f_{Y_1}[u_1(x_1, x_2, \ldots, x_n); \theta]H(x_1, x_2, \ldots, x_n)\\
&=k_{1}(y_{1},\theta)k_{2}(x_1, x_2, \ldots, x_n)
\end{align*}
Assim, $L$ e $k_{1}(y_{1},\theta)$, como funções de $\theta$, são maximizadas simultaneamente. Por hipótese, o EMV $\hat{\theta}$ existe e é único, logo, há apenas um valor de $\theta$ que maximiza $L$ e, portanto, $k_{1}[u_1(x_1, x_2, \ldots, x_n); \theta]$, esse valor de $\theta$ deve ser uma função de $u_1(x_1, x_2, \ldots, x_n)$. Portanto, o EMV $\hat{\theta}$ é uma função da estatística suficiente $Y_1 = u_1(X_1, X_2, \ldots, X_n)$.
\end{block}
\end{frame}

\begin{frame}{}
\frametitle{}
\begin{block}{}
\justifying
Sabemos, de aulas anteriores, que, em geral, os EMVs são estimadores assintoticamente não-viesados de $\theta$. Portanto, uma maneira de proceder é encontrar uma estatística suficiente e, em seguida, encontrar o EMV. Com base nisso, muitas vezes podemos obter um estimador não-viesado que é uma função da estatística suficiente. Esse processo é ilustrado no exemplo a seguir.
\end{block}
\end{frame}

\begin{frame}{Exemplo}
\frametitle{}
\begin{block}{}
\justifying
Sejam $X_1, \ldots, X_n\Sim exp(\theta).$

Suponha que desejamos um EMVU (Estimador de Variância Mínima Não-viesado) para $\theta$. 
\begin{equation}
\ell(\theta) = n \log \theta - \theta \sum_{i=1}^n x_i
\end{equation}
$\hat{\theta}=\dfrac{1}{\Bar{X}}$ é o EMV de $\theta.$ Também podemos verificar que $\Bar{X}$ é uma estatística suficiente para $\theta.$ 
\end{block}
\end{frame}

\begin{frame}{}
\frametitle{}
\begin{block}{}
\justifying
Note que $\hat{\theta} = \frac{n}{Y_1}$ é uma função da estatística suficiente $Y_1$. Além disso, como $\hat{\theta}$ é o EMV de $\theta$, ele é assintoticamente não-viesado. Portanto, como um primeiro passo, vamos determinar sua esperança. Neste problema, $X_i$ são variáveis aleatórias independentes e identicamente distribuídas com distribuição Gama $\Gamma(1, 1/\theta)$; portanto, $Y_1 = \sum_{i=1}^n X_i$ segue uma distribuição $\Gamma(n, 1/\theta)$. Assim,

\begin{align*}
E(\hat{\theta}) &= E\left(\frac{n}{Y_1}\right)= n E\left(\frac{1}{\Sumi X_{i}}\right) 
\end{align*}
\end{block}
\end{frame}

\begin{frame}{}
\frametitle{}
\begin{block}{}
\justifying
\begin{align*}
n E\left(\frac{1}{\Sumi X_{i}}\right) &= n \int_0^{\infty} \frac{(\frac{1}{\theta})^n}{\Gamma(n)} t^{-1}t^{n-1} e^{-\frac{t}{\theta}} dt\\
&= n \int_0^{\infty} \frac{(\frac{1}{\theta})^n}{\Gamma(n)} t^{(n-1)-1} e^{-\frac{t}{\theta}} dt\\
&= \dfrac{n}{\theta^{n}\Gamma(n)} \int_0^{\infty} \underbrace{t^{(n-1)-1} e^{-\frac{t}{\theta}}}_{\text{Núcleo da densidade de uma}~\Gamma(n-1,\theta)} dt\\
&= \dfrac{n}{\theta^{n}\Gamma(n)} \int_0^{\infty}\frac{(\frac{1}{\theta})^{(n-1)}}{\Gamma(n-1)} t^{(n-1)-1} e^{-\frac{t}{\theta}}\frac{\Gamma(n-1)}{(\frac{1}{\theta})^{(n-1)}} dt\\
&=\dfrac{n\Gamma(n-1)}{\dfrac{\theta^{n}}{\theta^{(n-1)}}\Gamma(n)}=\theta\dfrac{n}{n-1}
\end{align*}
\end{block}
\end{frame}

\begin{frame}{}
\frametitle{}
\begin{block}{}
\justifying
Portanto, o estimador $\Tilde{\theta}=\hat{\theta}\frac{(n-1)}{n}$ é um estimador não viesado de variância mínima para $\theta$.
\end{block}
\end{frame}

\begin{frame}{}
\frametitle{}
\begin{block}{}
\justifying
\textbf{Observação:} Como o estimador não-viesado $\xi(Y_1)$, em que $\xi(Y_1) = E(\hat{\theta}|y_1)$, possui uma variância menor do que o estimador não-viesado $Y_2$ de $\theta$, às vezes, raciocinamos da seguinte maneira. Seja a função $\Upsilon(y_3) = E[\xi(Y_1)|Y_3 = y_3]$, onde $Y_3$ é uma estatística que não é suficiente para $\theta$. Pelo teorema de Rao-Blackwell, temos $E[\Upsilon(Y_3)] = \theta$ e $\Upsilon(Y_3)$ possui uma variância menor do que $\xi(Y_1)$. Consequentemente, $\Upsilon(Y_3)$ deve ser melhor do que $\xi(Y_1)$ como um estimador não-viesado de $\theta$. No entanto, isso não é verdade, porque $Y_3$ não é suficiente; assim, $\theta$ está presente na distribuição condicional de $Y_1$, dado $Y_3 = y_3$, e na média condicional $\Upsilon(y_3)$. Embora de fato $E[\Upsilon(Y_3)] = \theta$, $\Upsilon(Y_3)$ nem sequer é uma estatística, pois envolve o parâmetro desconhecido $\theta$ e, portanto, não pode ser usado como uma estimativa.
\end{block}
\end{frame}

\section{Completude e Unicidade}
\begin{frame}{}
\frametitle{Completude}
\begin{definicao}
\justifying
Seja a variável aleatória $Z$ do tipo contínuo ou discreto, com função de densidade ou massa de probabilidade que pertence à família $\{h(z; \theta), \theta \in \Omega\}$. Se a condição $E[u(Z)] = 0$, para todo $\theta \in \Omega$, implica que $u(z)$ seja igual a zero, exceto em um conjunto de pontos com probabilidade zero, então para cada $h(z; \theta)$, $\theta \in \Omega$, a família $\{h(z; \theta) : \theta \in \Omega\}$ é chamada de família completa de funções de densidade ou massa de probabilidade.
\end{definicao}
\end{frame}

\begin{frame}{Exemplo 1}
\begin{block}{}
\justifying
$X_{1},\ldots,X_{n}\Sim$Poisson($\theta$). Sabemos que $\Sumi X_{i}$ é uma estatística suficiente para $\theta$ e $Z=\Sumi X_{i}\sim$Poisson$(n\theta),~\Omega=(0,+\infty).$ Seja $u(\cdot)$ tal que $E(u(Z))=0,~\forall\theta>0.$
\begin{align*}
0=E(u(Z))&={\displaystyle \sum_{z=0}^{+\infty}u(z)\dfrac{(n\theta)^{z}}{z!}e^{-n\theta}}\\
\Rightarrow 0&={\displaystyle \sum_{z=0}^{+\infty}u(z)\dfrac{n^{z}}{z!}\theta^{z}}, \forall \theta>0\\
\Rightarrow \dfrac{u(z)n^{z}}{z!}&=0,~\forall z\in\{0,1,\ldots\} \Rightarrow u(z)=0,\forall z\in\{0,1,\ldots\}
\end{align*}

\end{block}
\end{frame}

\begin{frame}{Exemplo 2}
\begin{block}{}
\justifying
Considere a família $\{h(z,\theta); \theta>0\}$ dada por $$h(z;\theta)=\dfrac{1}{\theta}e^{-\big(\dfrac{z}{\theta}\big)},~\theta>0,~\text{ou seja,}~Z\sim\exp{(\theta)}.$$
\end{block}
\pause
\begin{block}{}
\justifying
\begin{align*}
    E(u(Z))=0,\forall \theta>0&\Rightarrow \int_{0}^{+\infty}u(z)\dfrac{1}{\theta}e^{-\Big(\dfrac{z}{\theta}\Big)}dz=0,\forall \theta>0\\
    &\Rightarrow \int_{0}^{+\infty}u(z)e^{-\Big(\dfrac{z}{\theta}\Big)}dz=0,\forall \theta>0
\end{align*}
Notem que $\ell(\theta)=\int_{0}^{+\infty}u(z)e^{-\Big(\frac{z}{\theta}\Big)}dz=0$ é a transformada de Laplace de $u(\cdot).$ Assim, $\ell(\theta)=0\Rightarrow u(z)=0,\forall z>0.$
\end{block}
\end{frame}

\begin{frame}{}
\begin{block}{}
\justifying
Seja $Y1 = u_{1}(X_1,X_2, . . .,X_n)$ uma estatística suficiente para $\theta.$ Tome $\xi(Y_{1})$ tal que $E(\xi(Y_{1}))=\theta.$ Seja $\psi(Y_1)$ outra função de $Y_1$ de forma que, também tenhamos $E[\psi(Y_1)] = \theta$ para todos os valores de $\theta$, $\theta \in \Omega$. Portanto,
$E[\xi(Y_1) - \psi(Y_1)] = 0$, $\theta \in \Omega$. Se a família de $Y_{1}$ é completa, $\xi(Y_1) - \psi(Y_1)=0,$ ou seja, $\xi(Y_1) = \psi(Y_1).$ Assim, $\xi(Y_{1})$ é a única (com prob=1) função de $Y_{1}$ que é um estimador não viesado para $\theta.$
\end{block}
\pause
\begin{block}{}
\justifying
A conclusão anterior juntamente com o Teorema de Rao-Blackwel garantem que $\xi(Y_{1})$ é o ENVVUM de $\theta,$ conforme próximo teorema. 
\end{block}
\end{frame}

\section{Teorema de Lehmann-Scheffé}
\begin{frame}{}
\begin{Teorema}
\justifying
Seja $Y_{1}$ uma estatística suficiente para $\theta$ e assuma que a família de $Y_{1}$ é completa. Se existe uma função $\xi(\cdot)$ tal que $E(\xi(Y_{1}))=\theta,~\forall \theta\in \Omega,$ então, $\xi(Y_{1})$ é o ENVVUM de $\theta.$
\end{Teorema}
\end{frame}

\begin{frame}{}
\begin{block}{\Home}
\justifying
\begin{itemize}
    \item \textbf{Exercícios da seção 7.3:} 1,3,4,5,6
    \item \textbf{Exercícios da seção 7.4:} 1,2,3,4,5,7,8,9
\end{itemize}
\nocite{hogg, casella2021statistical, bolfarine}
\end{block}
\end{frame}


\begin{frame}[allowframebreaks]
\frametitle{\bf Referências}
\printbibliography
\end{frame}


\end{document}

Foi observado que se existe algum estimador não enviesado $Y_{2}$ (que não é uma função apenas de $Y_1$) de $\theta$, então existe pelo menos uma função de $Y_1$ que é um estimador não enviesado de $\theta$, e nossa busca pelo melhor estimador de $\theta$ pode ser restrita a funções de $Y_1$.

Suponha que tenha sido verificado que uma certa função $\xi(Y_1)$, que não é uma função de $\theta$, satisfaça $E[\xi(Y_1)] = \theta$ para todos os valores de $\theta$, $\theta \in \Omega$. Seja $\psi(Y_1)$ outra função da estatística suficiente $Y_1$ isoladamente, de forma que também tenhamos $E[\psi(Y_1)] = \theta$ para todos os valores de $\theta$, $\theta \in \Omega$. Portanto,
$E[\xi(Y_1) - \psi(Y_1)] = 0$, $\theta \in \Omega$.

\subsection{Consistência}
\begin{frame}{}
\begin{block}{}
\justifying
Consistência está ligada ao conceito de convergência em probabilidade.

Sejam $X_1, \ldots, X_n$ uma amostra aleatória da distribuição da variável aleatória $X$ que depende do parâmetro $\theta$. Dizemos que o estimador $\hat{\theta} = \hat{\theta}(X_1, \ldots, X_n)$ é consistente para o parâmetro $\theta,$ se,
\begin{align*}
    \lim_{n\rightarrow \infty} P(|\hat{\theta}-\theta|>\epsilon)=0
\end{align*}
\end{block}
\end{frame}

\begin{frame}{Exemplo}
\begin{block}{}
\justifying
Sejam $X_1, \ldots, X_n$ uma amostra aleatória de tamanho $n$ da distribuição da variável aleatória $X$ com média $\theta$ e variância $\sigma^2$. Temos, usando a desigualdade de Chebyshev, que, $$P(|\Bar{X}-\theta|>\epsilon)\leq \dfrac{\sigma^{2}}{n\epsilon^{2}},$$ 
de modo que,
$$\lim_{n\rightarrow \infty} P(|\Bar{X}-\theta|>\epsilon)=0,$$
e portanto $\Bar{X}$ é consistente para $\theta$.
\end{block}
\nocite{hogg}\nocite{casella2021statistical} \nocite{bolfarine}
\end{frame}