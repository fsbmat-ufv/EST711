\documentclass[12pt]{beamer}

\usepackage[brazil]{babel}
\usepackage[utf8]{inputenc}
\usepackage[T1]{fontenc}
\usepackage{animate}
\usepackage{amsbsy}
\usepackage{amsfonts}
\usepackage{amsmath}
\usepackage{amssymb}
\usepackage{amsthm}
\setbeamertemplate{theorems}[numbered] % to number
\usepackage[toc,page,title,titletoc]{appendix}
\usepackage{dsfont}
\usepackage{esvect}
\usepackage[labelfont=bf]{caption}
%\usepackage{subcaption}
\usepackage{float}
\usepackage[Glenn]{fncychap}%Sonny %Conny %Lenny %Glenn %Renje %Bjarne %Bjornstrup
\usepackage{graphicx}
\usepackage{subfig}
\usepackage{indentfirst}%Para indentar os paragrafos automáticamente
\usepackage{lipsum}
\usepackage{longtable}
\usepackage{mathtools}
\usepackage{listings}%Inserir codigo do R no latex
\usepackage{multirow}
\usepackage{multicol}
\usepackage{csquotes}
\usepackage[citestyle=authoryear,maxcitenames=2,terseinits=true,natbib=true, style=abnt, maxbibnames=99]{biblatex}
\addbibresource{../Referencias/Referencias.bib}
\usepackage[figuresright]{rotating}
\usepackage{spalign}
\usepackage{pgfplots}
\pgfplotsset{compat=1.17}
\usepackage{tikz}
\usepackage{fontawesome}
\usepackage{color, colortbl}
\usepackage{url}
\usepackage{cancel}
\usepackage{accents}
\usepackage{bm}
\usepackage{ragged2e}%para justificar o texto dentro de algum ambiente
\definecolor{Gray}{gray}{0.9}
\definecolor{LightCyan}{rgb}{0.88,1,1}


\usepackage[all]{xy}
\usepackage{hyperref,bookmark}
\hypersetup{
  colorlinks=true,
  linkcolor=blue,
  citecolor=red,
  filecolor=blue,
  urlcolor=blue,
}

\usetheme{Madrid}
\usecolortheme[RGB={193,0,0}]{structure}

%\setbeamertemplate{footline}[frame number]
%\setbeamertemplate{footline}[text line]{%
%  \parbox{\linewidth}{\vspace*{-8pt}\hfill\date{}\hfill\insertshortauthor\hfill\insertpagenumber}}
\beamertemplatenavigationsymbolsempty
\renewcommand{\vec}[1]{\mbox{\boldmath$#1$}}
\newtheorem{Teorema}{Teorema}
\newtheorem{Proposicao}{Proposição}
\newtheorem{definicao}{Definição}
\newtheorem{Corolario}{Corolário}
\newtheorem{Demonstracao}{Demonstração}
\newcommand{\bx}{\ensuremath{\bar{x}}}
\newcommand{\Ho}{\ensuremath{H_{0}}}
\newcommand{\Hi}{\ensuremath{H_{1}}}
\newcommand{\at}[2][]{#1|_{#2}}
\newcommand\xuparrow[1][2ex]{%
   \mathrel{\rotatebox{-90}{$\xleftarrow{\rule{#1}{0pt}}$}}
}
\apptocmd{\frame}{}{\justifying}{} % Allow optional arguments after frame.

\makeatletter
\setbeamertemplate{footline}
{
  \leavevmode%
  \hbox{%
  \begin{beamercolorbox}[wd=.3\paperwidth,ht=2.25ex,dp=1ex,center]{author in head/foot}%
    \usebeamerfont{author in head/foot}\mytext
  \end{beamercolorbox}%
  \begin{beamercolorbox}[wd=.3\paperwidth,ht=2.25ex,dp=1ex,center]{title in head/foot}%
    \usebeamerfont{title in head/foot}\mytextt
  \end{beamercolorbox}%
  \begin{beamercolorbox}[wd=.35\paperwidth,ht=2.25ex,dp=1ex,right]{site in head/foot}%
    \usebeamerfont{site in head/foot}\mytexttt\hspace*{2em}
    \insertframenumber{} / \inserttotalframenumber\hspace*{2ex} 
  \end{beamercolorbox}}%
  \vskip0pt%
}
\makeatother

\providecommand{\arcsin}{} \renewcommand{\arcsin}{\hspace{2pt}\textrm{arcsen}}
\providecommand{\sin}{} \renewcommand{\sin}{\hspace{2pt}\textrm{sen}}
\newcommand{\N}{\rm I\!N}
\newcommand{\I}{\rm I\!I}
\newcommand{\R}{\rm I\!R}
\newcommand{\Sim}{\overset{\text{iid}}{\sim}}
\newcommand{\Lim}{{\displaystyle \lim_{n\to\infty}}}
\newcommand{\LimInf}{{\displaystyle \liminf_{n\to\infty}}}
\newcommand{\rightLim}{\xrightarrow[n\rightarrow\infty]{}}
\newcommand{\Sumi}{{\displaystyle \sum_{i=1}^{n}}}
\newcommand{\Int}{{\displaystyle \int_{-\infty}^{+\infty}}}
\newcommand{\ConvD}{\overset{D}{\rightarrow}}
\newcommand{\ConvP}{\overset{P}{\rightarrow}}
\newcommand{\Prodi}{{\displaystyle \prod_{i=1}^{n}}}
\newcommand{\SetaUP}[2]{\underset{\mathclap{\substack{\xuparrow[30pt] \\ #1}}}{#2}}
%\newcommand{\SetaInclinada}[2]{\underset{\mathclap{\substack{\rotatebox{135}{\xuparrow[30pt] \\ #1}}}}{#2}}
\newcommand{\Home}{\begin{tikzpicture}
\node[scale=2] at (3,4) {\text{Para}~\faHome};
\end{tikzpicture}}
\newcommand{\vecX}{\boldsymbol{X}}
\newcommand{\Implica}[1]{\xRightarrow{#1}}
\newcommand{\SeSe}{\iff}
\newcommand{\EscoreA}{\dfrac{\partial}{\partial\theta}\log{f(x,\theta)}}
\newcommand{\EscoreB}{\dfrac{\partial^{2}}{\partial\theta^{2}}\log{f(x,\theta)}}
\newcommand{\cqd}{\text{cqd}~\blacksquare}
\newcommand{\seqX}{$X_{1},\ldots,X_{n}$}
\newcommand{\seqY}{$Y_{1},\ldots,Y_{n}$}
\newcommand{\tend}[1]{\hbox{\oalign{$\bm{#1}$\crcr\hidewidth$\scriptscriptstyle\bm{\sim}$\hidewidth}}}

%\newtheorem{Teorema}{Teorema}
%\newtheorem{Proposicao}{Proposição}
%\newtheorem{Definicao}{Definição}
%\newtheorem{Corolario}{Corolário}
%\newtheorem{Demonstracao}{Demonstração}

\titlegraphic{\hspace*{8cm}\href{https://fsbmat-ufv.github.io/}{\includegraphics[width=2cm]{figs/mylogo.png}}
}

%Continuar a numeracao em slides diferentes
\newcounter{saveenumi}
\newcommand{\seti}{\setcounter{saveenumi}{\value{enumi}}}
\newcommand{\conti}{\setcounter{enumi}{\value{saveenumi}}}

\resetcounteronoverlays{saveenumi}

% Layout da pagina
\hypersetup{pdfpagelayout=SinglePage}

%Para o \pause funcionar dentro do ambiente align
\makeatletter
\let\save@measuring@true\measuring@true
\def\measuring@true{%
  \save@measuring@true
  \def\beamer@sortzero##1{\beamer@ifnextcharospec{\beamer@sortzeroread{##1}}{}}%
  \def\beamer@sortzeroread##1<##2>{}%
  \def\beamer@finalnospec{}%
}
\makeatother


\title{Inferência Estatística II}
\author{Prof. Fernando de Souza Bastos\texorpdfstring{\\ fernando.bastos@ufv.br}{}}
\institute{Departamento de Estatística\texorpdfstring{\\ Programa de Pós-Graduação em Estatística Aplicada e Biometria}\texorpdfstring{\\ Universidade Federal de Viçosa}{}\texorpdfstring{\\ Campus UFV - Viçosa}{}}
\date{}
\newcommand\mytext{Aula 3}
\newcommand\mytextt{Fernando de Souza Bastos}
\newcommand\mytexttt{\url{https://est711.github.io/}}


\begin{document}
%\SweaveOpts{concordance=TRUE}

\frame{\titlepage}

\begin{frame}{}
\frametitle{\bf Sumário}
\tableofcontents
\end{frame}

\section{Função Geradora de Momentos}
\begin{frame}{Função Geradora de Momentos}
\begin{definicao}
\justifying
A função geradora de momentos de uma variável aleatória $X$ é definida por $M_{X}(t)=E(e^{tX}),~t\in \R$
\end{definicao}
\end{frame}

\begin{frame}{}
\begin{Teorema}
\justifying
Seja $\{X_{n}\}_{n\geq 1}$ uma sequência de variáveis aleatórias com fgm $M_{X_n}(t)$ que existe para $|t|<h$ para todo $n$. Seja $X$ uma variável aleatória com fgm $M_{X}(t)$, que existe para $|t| \leq h_1 \leq h$. Se $\lim_{n \to \infty} M_{X_n}(t) = M_{X}(t)$ para $|t| \leq h_1$, então $X_{n} \overset{D}{\rightarrow} X.$
\end{Teorema}
\pause
\begin{block}{Observação importante na resolução de exercícios:}
\justifying
Se ${\displaystyle \lim_{n \to \infty} \left(1 + \frac{b}{n} + \frac{\psi(n)}{n}\right)^{cn}}$, em que $b$ e $c$ não dependem de $n$ e, em que, ${\displaystyle\lim_{n \to \infty} \psi(n) = 0}$. Então,
${\displaystyle \lim_{n \to \infty} \left(1 + \frac{b}{n} + \frac{\psi(n)}{cn}\right) = \lim_{n \to \infty} \left(1 + \frac{b}{n}\right)^{cn} = e^{bc}}.$
\end{block}
\end{frame}

\begin{frame}{Exemplo 1}
\begin{block}{}
\justifying

${\displaystyle \lim_{{n \to \infty}} \left(1 - \frac{{t^2}}{{n}} + \frac{{t^2}}{{n^{3/2}}}\right)^{-n/2}} = {\displaystyle \lim_{{n \to \infty}} \left(1 - \frac{{t^2}}{{n}} + \frac{{t^2}/\sqrt{n}}{{n}}\right)^{-n/2}}$.

Aqui, $b = -t^2$, $c = -\frac{1}{2}$ e $\psi(n) = \frac{{t^2}}{{\sqrt{n}}}$. Consequentemente, para cada valor fixo de $t$, o limite é $e^{t^2/2}$.
\end{block}
\end{frame}

\begin{frame}{Exemplo 2}
	\vspace{-0.5cm}
\begin{block}{}
\justifying
Considere $X_{n}\sim Binomial(n,p_{n})$ e suponha ${\displaystyle \Lim np_{n}=\lambda>0}$ (por exemplo, $p_{n}=\dfrac{1}{n+1},~\Lim np_{n}=1$). Então, $X_{n} \overset{D}{\rightarrow} X,$ em que $X\sim$Poisson($\lambda$).
\end{block}
\pause
	\vspace{-0.3cm}
\begin{block}{Demonstração}
\justifying
Temos que,
\begin{align*}
    M_{X_{n}}(t)&=E(e^{tX_{n}})={\displaystyle \sum_{k=0}^{n}e^{tk}\binom{n}{k}p_{n}^{k}(1-p_{n})^{n-k}}\\
    &=\Big(1-p_{n}+p_{n}e^{t}\Big)^{n}=\Big(1+\dfrac{np_{n}}{n}(e^{t}-1)\Big)^{n}\\
\text{(para n grande)}    &=\Big(1+\dfrac{\lambda}{n}(e^{t}-1)\Big)^{n}\rightLim \exp{\{\lambda(e^{t}-1)\}}
\end{align*}
Logo, $X_{n} \overset{D}{\rightarrow} X\sim$Poisson($\lambda$).
\end{block}
\end{frame}

\begin{frame}
	\begin{block}{}
		\justifying
Quando a quantidade \( np_n \) se estabiliza em um valor \( \lambda > 0 \), estamos essencialmente controlando a média da binomial. À medida que \( n \to \infty \) e \( p_n \) diminui de forma controlada, mantemos \( np_n \) constante, aproximando o comportamento da binomial ao de uma distribuição Poisson com parâmetro \( \lambda \). A essência é que estamos explorando o comportamento assintótico da binomial, com \( p_n \) diminuindo à medida que \( n \) cresce, mas de modo que \( np_n \) permaneça fixo e igual a \( \lambda \). Isso faz com que a média e variância da binomial ``convirjam'' para os parâmetros de uma Poisson.
%\textbf{Transformação da expressão}
%Quando reescrevemos o momento gerador:
%\[
%\left(1 - p_n + p_n e^t\right)^n = \left(1 + \frac{np_n}{n}(e^t - 1)\right)^n
%\]
%fazemos isso porque \( np_n \) tende a \( \lambda \). Ou seja, estamos preparando a fórmula para o limite \( n \to \infty \).

%No limite:
%\[
%np_n \approx \lambda \quad \text{para grandes } n
%\]
%Assim, substituímos \( \frac{np_n}{n} \) por \( \frac{\lambda}{n} \) e fazemos \( n \to \infty \). Isso nos leva ao limite exponencial:
%\[
%\left(1 + \frac{\lambda}{n}(e^t - 1)\right)^n \to \exp\left(\lambda (e^t - 1)\right)
%\]
%Essa é a razão pela qual manipulamos a expressão de modo a colocar \( np_n \) em evidência.

	\end{block}
\end{frame}

\section{Teorema Central do Limite}
\begin{frame}{Teorema Central do Limite}
\begin{block}{}
\justifying
Seja $\{X_{n}\}_{n\geq 1}$ uma sequência de variáveis aleatórias iid com média $\mu$ e variância $\sigma^{2}<\infty.$ Então,
\begin{align*}
   \dfrac{\Sumi X_{i}-n\mu}{\sqrt{n}\sigma}=\dfrac{\sqrt{n}(\Bar{X}_{n}-\mu)}{\sigma} \ConvD N(0,1)
\end{align*}
\end{block}
\end{frame}

\begin{frame}{Demonstração}
\begin{block}{}
\justifying
Assuma, sem perda de generalidade, $\mu=0$ e $\sigma^{2}=1.$
\begin{align*}
    M_{\frac{\sum X_{i}-n\mu}{\sqrt{n}\sigma}}(t)&=M_{\frac{\sum X_{i}}{\sqrt{n}}}(t)
    =E\Big(e^{\frac{t\sum X_{i}}{\sqrt{n}}}\Big)=E\Big(\Prodi e^{\frac{tX_{i}}{\sqrt{n}}}\Big)\\
    &=\Prodi E\Big(e^{\frac{tX_{i}}{\sqrt{n}}}\Big)=\Prodi M_{X_{i}}\big(\frac{t}{\sqrt{n}}\big)=\Big(M_{X_{i}}(\dfrac{t}{\sqrt{n}})\Big)^{n}\\
\end{align*}
Ou seja, $\ln{M_{\frac{\sum X_{i}}{\sqrt{n}}}(t)}=n\ln{M_{X_{i}}(\frac{t}{\sqrt{n}})}=\dfrac{\ln{M_{X_{i}}(\frac{t}{\sqrt{n}})}}{\frac{1}{n}},$ aplicando L'Hôpital, temos:
\end{block}
\end{frame}

\begin{frame}{}
\begin{block}{}
\justifying
\begin{align*}
    \Lim \frac{\ln{M_{X_{i}}(\frac{t}{\sqrt{n}})}}{\frac{1}{n}}&=\Lim \dfrac{\frac{M_{X_{i}}^{'}(\frac{t}{\sqrt{n}})}{M_{X_{i}}(\frac{t}{\sqrt{n}})}t(-\frac{1}{2})n^{-\frac{3}{2}}}{-\frac{1}{n^{2}}}\\
    &=\dfrac{t}{2}\Lim \sqrt{n}M_{X_{i}}^{'}(\frac{t}{\sqrt{n}})\\
    &=\dfrac{t}{2}\Lim \dfrac{M_{X_{i}}^{'}(\frac{t}{\sqrt{n}})}{\frac{1}{\sqrt{n}}}\\
    &=\dfrac{t}{2}\Lim \dfrac{M_{X_{i}}^{''}(\frac{t}{\sqrt{n}})t(-1/2)n^{-3/2}}{-\frac{1}{2}n^{-3/2}}\\
    &=\dfrac{t^{2}}{2}\Lim M_{X_{i}}^{''}(\frac{t}{\sqrt{n}})=\dfrac{t^{2}}{2}
\end{align*}
Logo, $\dfrac{\sum X_{i}}{\sqrt{n}}\ConvD X\sim N(0,1)$
\end{block}
\end{frame}

\section{Método Delta}
\begin{frame}{Método Delta}
\begin{block}{}
\justifying
Suponha que conhecemos a distribuição de uma variável aleatória, mas que queremos determinar a distribuição de uma função dela. Isso também é verdade na teoria assintótica, o teorema de Slutsky's e o teorema visto em aula, imediatamente anterior a ele, são ilustrações disso. Outro resultado desse tipo é chamado de método delta. Vejamos o próximo slide!
\end{block}
\end{frame}

%\begin{frame}{}
%\begin{Teorema}
%\justifying   
%Suponha que g(x) seja diferenciável em x. Então podemos escrever
%$$g(y) = g(x) + g'(x)(y - x) + o(|y - x|).$$ Em que, a notação $o$ significa que $a = o(b)$ se e somente se $\frac{a}{b} \rightarrow 0$, conforme $b \rightarrow 0$.
%\end{Teorema}
%\end{frame}

\begin{frame}{Método Delta}
\begin{Teorema}
\justifying
Seja $\{X_{n}\}_{n\geq 1}$ uma sequência de variáveis aleatórias satisfazendo a seguinte convergência em distribuição:
$$\sqrt{n}(X_{n}-\theta)\ConvD N(0,\sigma^{2}).$$ Se $g$ é uma função diferenciável em $\theta$ e $g'(\theta)\neq 0,$ então $$\sqrt{n}(g(X_{n})-g(\theta))\ConvD N(0, [g'(\theta)]^{2}\sigma^{2}).$$
\end{Teorema}
\end{frame}

\begin{frame}{Demonstração}
\begin{block}{}
\justifying
Vamos mostrar primeiro que $X_{n}\ConvP \theta.$ Seja $\varepsilon>0$ e $m>0$ inteiro ($m\in \N^{*}$) fixado. 
\begin{align*}
    P(|X_{n}-\theta|<\varepsilon)&=P(|\sqrt{n}(X_{n}-\theta)|<\varepsilon\sqrt{n})\\
    &\SetaUP{\text{para}~n\geq \Big(\dfrac{m}{\varepsilon}\Big)^{2}}{\geq} P(|\sqrt{n}(X_{n}-\theta)|<m)
\end{align*}
\end{block}
\end{frame}

\begin{frame}{Continuação da Demonstração}
\vspace{-0.2cm}
\begin{block}{}
\justifying
Segue que, 
\begin{align*}
    \LimInf P(|X_{n}-\theta|<\varepsilon)&\geq \LimInf P(|\sqrt{n}(X_{n}-\theta)|<m)\\
    &=\Lim P(|\sqrt{n}(X_{n}-\theta)|<m)\\
    &\SetaUP{\text{Usando o fato de que}\\ \sqrt{n}(X_{n}-\theta)\ConvD N(0,\sigma^{2})}{=}P(|Z|<\sigma m),~Z\sim N(0,1)
\end{align*}
\end{block}
\pause
\vspace{-0.2cm}
\begin{block}{}
\justifying
Assim,
\begin{align*}
    \LimInf P(|X_{n}-\theta|<\varepsilon)&\geq P(|Z|<\sigma m)=\Phi(\sigma m)-\Phi(-\sigma m)\\
    &=2\Phi(\sigma m)-1
\end{align*}
\end{block}
\end{frame}

\begin{frame}{}
\begin{block}{}
\justifying
Em resumo,
\begin{align*}
    \LimInf P(|X_{n}-\theta|<\varepsilon)&\geq 2\Phi(\sigma m)-1,~\forall m\in \mathbb{Z}^{*}\\  
    \LimInf P(|X_{n}-\theta|<\varepsilon)&\geq \lim_{m\rightarrow +\infty} \Big[2\Phi(\sigma m)-1\Big]\\
    &=1
\end{align*}
\end{block}
\pause
\begin{block}{}
\justifying
Como $\limsup\geq \liminf,$ temos que
\begin{align}\label{num1}
    \limsup P(|X_{n}-\theta|<\varepsilon)=1\Rightarrow \Lim P(|X_{n}-\theta|<\varepsilon)
\end{align}
\end{block}
\pause
\begin{block}{}
\justifying
Expandindo $g$ em série de Taylor até a primeira ordem em torno de $\theta,$ temos que $$g(x)=g(\theta)+g'(\theta)(x-\theta)(x-\theta)+\mathcal{C}(x)(x-\theta),$$ em que ${\displaystyle \lim_{x\rightarrow \theta} \mathcal{C}(x)=0}.$
\end{block}
\end{frame}

\begin{frame}{}
\begin{block}{}
\justifying
Segue que, 
\begin{align*}
g(x)-g(\theta)&=g'(\theta)(x-\theta)(x-\theta)+\mathcal{C}(x)(x-\theta)\\
\sqrt{n}\big(g(X_{n})-g(\theta)\big)&=\sqrt{n}(X_{n}-\theta)\big(g'(\theta)-\mathcal{C}(X_{n}\big).
\end{align*}
Para mostrar o resultado desejado, devemos mostrar que $\mathcal{C}(X_{n})\ConvP 0.$ Com isso, utilizando o teorema de Slutsky, o resultado é obtido. Como ${\displaystyle \lim_{x\rightarrow \theta} \mathcal{C}(x)=0},~\text{para}~\varepsilon>0, \exists~\delta>0$ tal que $|x-\theta|<\delta\Rightarrow~|\mathcal{C}(x)|<\varepsilon.$
\end{block}
\pause
\begin{block}{}
\justifying
Daí,
\begin{align*}
    \{|X_{n}-\theta|<\delta\}&\subset \{|\mathcal{C}(X_{n})|<\varepsilon\}\Rightarrow P(|X_{n}-\theta|<\delta)\leq P(|\mathcal{C}(X_{n})|<\varepsilon)
\end{align*}
Esse resultado, juntamente com $(\ref{num1})$ implica que,
\begin{align*}
    \Lim P(|\mathcal{C}(X_{n})|<\varepsilon)=1,~\text{ou seja},~\mathcal{C}(X_{n})\ConvP 0.
\end{align*}
\end{block}
\end{frame}

\begin{frame}{Exercício}
\begin{block}{\Home}
\justifying
Seja $X_{n}\sim Binomial(n,p).$ Mostre que 
\begin{align*}
    \sqrt{n}\big(\arcsin{\sqrt{\dfrac{X_{n}}{n}}}-\arcsin{\sqrt{p}}\big)\ConvD N\big(0,\dfrac{1}{4}\big)
\end{align*}
\end{block}
\nocite{casella2021statistical} \nocite{hogg}
\end{frame}

\begin{frame}{}
\begin{block}{\Home}
\justifying
Exercícios 5.3.1 à 5.3.8, 5.3.11, 5.3.12
\end{block}
\end{frame}


\begin{frame}[allowframebreaks]
\frametitle{\bf Referências}
\printbibliography
\end{frame}


\end{document}
