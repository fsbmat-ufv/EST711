\documentclass[12pt]{beamer}

\usepackage[brazil]{babel}
\usepackage[utf8]{inputenc}
\usepackage[T1]{fontenc}
\usepackage{animate}
\usepackage{amsbsy}
\usepackage{amsfonts}
\usepackage{amsmath}
\usepackage{amssymb}
\usepackage{amsthm}
\setbeamertemplate{theorems}[numbered] % to number
\usepackage[toc,page,title,titletoc]{appendix}
\usepackage{dsfont}
\usepackage{esvect}
\usepackage[labelfont=bf]{caption}
%\usepackage{subcaption}
\usepackage{float}
\usepackage[Glenn]{fncychap}%Sonny %Conny %Lenny %Glenn %Renje %Bjarne %Bjornstrup
\usepackage{graphicx}
\usepackage{subfig}
\usepackage{indentfirst}%Para indentar os paragrafos automáticamente
\usepackage{lipsum}
\usepackage{longtable}
\usepackage{mathtools}
\usepackage{listings}%Inserir codigo do R no latex
\usepackage{multirow}
\usepackage{multicol}
\usepackage{csquotes}
\usepackage[citestyle=authoryear,maxcitenames=2,terseinits=true,natbib=true, style=abnt, maxbibnames=99]{biblatex}
\addbibresource{../Referencias/Referencias.bib}
\usepackage[figuresright]{rotating}
\usepackage{spalign}
\usepackage{pgfplots}
\pgfplotsset{compat=1.17}
\usepackage{tikz}
\usepackage{fontawesome}
\usepackage{color, colortbl}
\usepackage{url}
\usepackage{cancel}
\usepackage{accents}
\usepackage{bm}
\usepackage{ragged2e}%para justificar o texto dentro de algum ambiente
\definecolor{Gray}{gray}{0.9}
\definecolor{LightCyan}{rgb}{0.88,1,1}


\usepackage[all]{xy}
\usepackage{hyperref,bookmark}
\hypersetup{
  colorlinks=true,
  linkcolor=blue,
  citecolor=red,
  filecolor=blue,
  urlcolor=blue,
}

\usetheme{Madrid}
\usecolortheme[RGB={193,0,0}]{structure}

%\setbeamertemplate{footline}[frame number]
%\setbeamertemplate{footline}[text line]{%
%  \parbox{\linewidth}{\vspace*{-8pt}\hfill\date{}\hfill\insertshortauthor\hfill\insertpagenumber}}
\beamertemplatenavigationsymbolsempty
\renewcommand{\vec}[1]{\mbox{\boldmath$#1$}}
\newtheorem{Teorema}{Teorema}
\newtheorem{Proposicao}{Proposição}
\newtheorem{definicao}{Definição}
\newtheorem{Corolario}{Corolário}
\newtheorem{Demonstracao}{Demonstração}
\newcommand{\bx}{\ensuremath{\bar{x}}}
\newcommand{\Ho}{\ensuremath{H_{0}}}
\newcommand{\Hi}{\ensuremath{H_{1}}}
\newcommand{\at}[2][]{#1|_{#2}}
\newcommand\xuparrow[1][2ex]{%
   \mathrel{\rotatebox{-90}{$\xleftarrow{\rule{#1}{0pt}}$}}
}
\apptocmd{\frame}{}{\justifying}{} % Allow optional arguments after frame.

\makeatletter
\setbeamertemplate{footline}
{
  \leavevmode%
  \hbox{%
  \begin{beamercolorbox}[wd=.3\paperwidth,ht=2.25ex,dp=1ex,center]{author in head/foot}%
    \usebeamerfont{author in head/foot}\mytext
  \end{beamercolorbox}%
  \begin{beamercolorbox}[wd=.3\paperwidth,ht=2.25ex,dp=1ex,center]{title in head/foot}%
    \usebeamerfont{title in head/foot}\mytextt
  \end{beamercolorbox}%
  \begin{beamercolorbox}[wd=.35\paperwidth,ht=2.25ex,dp=1ex,right]{site in head/foot}%
    \usebeamerfont{site in head/foot}\mytexttt\hspace*{2em}
    \insertframenumber{} / \inserttotalframenumber\hspace*{2ex} 
  \end{beamercolorbox}}%
  \vskip0pt%
}
\makeatother

\providecommand{\arcsin}{} \renewcommand{\arcsin}{\hspace{2pt}\textrm{arcsen}}
\providecommand{\sin}{} \renewcommand{\sin}{\hspace{2pt}\textrm{sen}}
\newcommand{\N}{\rm I\!N}
\newcommand{\I}{\rm I\!I}
\newcommand{\R}{\rm I\!R}
\newcommand{\Sim}{\overset{\text{iid}}{\sim}}
\newcommand{\Lim}{{\displaystyle \lim_{n\to\infty}}}
\newcommand{\LimInf}{{\displaystyle \liminf_{n\to\infty}}}
\newcommand{\rightLim}{\xrightarrow[n\rightarrow\infty]{}}
\newcommand{\Sumi}{{\displaystyle \sum_{i=1}^{n}}}
\newcommand{\Int}{{\displaystyle \int_{-\infty}^{+\infty}}}
\newcommand{\ConvD}{\overset{D}{\rightarrow}}
\newcommand{\ConvP}{\overset{P}{\rightarrow}}
\newcommand{\Prodi}{{\displaystyle \prod_{i=1}^{n}}}
\newcommand{\SetaUP}[2]{\underset{\mathclap{\substack{\xuparrow[30pt] \\ #1}}}{#2}}
%\newcommand{\SetaInclinada}[2]{\underset{\mathclap{\substack{\rotatebox{135}{\xuparrow[30pt] \\ #1}}}}{#2}}
\newcommand{\Home}{\begin{tikzpicture}
\node[scale=2] at (3,4) {\text{Para}~\faHome};
\end{tikzpicture}}
\newcommand{\vecX}{\boldsymbol{X}}
\newcommand{\Implica}[1]{\xRightarrow{#1}}
\newcommand{\SeSe}{\iff}
\newcommand{\EscoreA}{\dfrac{\partial}{\partial\theta}\log{f(x,\theta)}}
\newcommand{\EscoreB}{\dfrac{\partial^{2}}{\partial\theta^{2}}\log{f(x,\theta)}}
\newcommand{\cqd}{\text{cqd}~\blacksquare}
\newcommand{\seqX}{$X_{1},\ldots,X_{n}$}
\newcommand{\seqY}{$Y_{1},\ldots,Y_{n}$}
\newcommand{\tend}[1]{\hbox{\oalign{$\bm{#1}$\crcr\hidewidth$\scriptscriptstyle\bm{\sim}$\hidewidth}}}

%\newtheorem{Teorema}{Teorema}
%\newtheorem{Proposicao}{Proposição}
%\newtheorem{Definicao}{Definição}
%\newtheorem{Corolario}{Corolário}
%\newtheorem{Demonstracao}{Demonstração}

\titlegraphic{\hspace*{8cm}\href{https://fsbmat-ufv.github.io/}{\includegraphics[width=2cm]{figs/mylogo.png}}
}

%Continuar a numeracao em slides diferentes
\newcounter{saveenumi}
\newcommand{\seti}{\setcounter{saveenumi}{\value{enumi}}}
\newcommand{\conti}{\setcounter{enumi}{\value{saveenumi}}}

\resetcounteronoverlays{saveenumi}

% Layout da pagina
\hypersetup{pdfpagelayout=SinglePage}

%Para o \pause funcionar dentro do ambiente align
\makeatletter
\let\save@measuring@true\measuring@true
\def\measuring@true{%
  \save@measuring@true
  \def\beamer@sortzero##1{\beamer@ifnextcharospec{\beamer@sortzeroread{##1}}{}}%
  \def\beamer@sortzeroread##1<##2>{}%
  \def\beamer@finalnospec{}%
}
\makeatother

\title{Inferência Estatística II}
\author{Prof. Fernando de Souza Bastos\texorpdfstring{\\ fernando.bastos@ufv.br}{}}
\institute{Departamento de Estatística\texorpdfstring{\\ Programa de Pós-Graduação em Estatística Aplicada e Biometria}\texorpdfstring{\\ Universidade Federal de Viçosa}{}\texorpdfstring{\\ Campus UFV - Viçosa}{}}
\date{}
\newcommand\mytext{Aula 12}
\newcommand\mytextt{Fernando de Souza Bastos}
\newcommand\mytexttt{\url{https://est711.github.io/}}


\begin{document}
%\SweaveOpts{concordance=TRUE}

\frame{\titlepage}

\begin{frame}{}
\frametitle{\bf Sumário}
\tableofcontents
\end{frame}

\section{Testes Mais Poderosos}
\begin{frame}{}
\begin{block}{}
\justifying
Sejam \seqX~ variáveis aleatórias iid de uma distribuição dependendo de um vetor de parâmetros $\theta \in \Omega.$ Assuma que $\theta\in W_{0}$ ou $\theta\in W_{1},$ com $W_{0}\cap W_{1}=\emptyset$ e $W_{0}\cup W_{1}=\Omega.$ Com isso, definimos as hipóteses:
\begin{align*}
    H_{0}:\theta\in W_{0}~\text{contra}~H_{1}:\theta\in W_{1}.
\end{align*}
\end{block}
\end{frame}

\begin{frame}{}
\begin{block}{}
\justifying
O teste de $H_{0}$ contra $H_{1}$ é baseado na amostra \seqX, considere um subconjunto (dependendo da amostra) $\mathcal{C}$ de $\mathcal{S},$ em que $\mathcal{S}$ é o suporte da amostra aleatória. Essa região $\mathcal{C}$ é conhecida como região crítica e sua correspondente regra de decisão é:
\begin{itemize}
    \item Rejeite $H_{0}$ (Aceite $H_{1}$) se (\seqX)$\in\mathcal{C}.$
    \item Aceita $H_{0}$ (Rejeite $H_{1}$) se (\seqX)$\notin\mathcal{C}~(\in \mathcal{C}^{c}).$
\end{itemize}
\end{block}
\pause
\begin{block}{}
\justifying
O erro tipo I ocorre se $H_{0}$ é rejeitada quando ela é verdadeira, enquanto o erro tipo II ocorre se $H_{0}$ é aceita quando $H_{1}$ é verdadeira. Nível de significância é o erro do tipo I, isto é, 
\begin{align*}
    \alpha=\max_{\theta\in W_{0}}P_{\theta}[(X_{1},\ldots,X_{n})\in\mathcal{C}].
\end{align*}
\end{block}
\end{frame}

\begin{frame}{}
\begin{block}{}
\justifying
Restrito a testes de tamanho $\alpha,$ queremos selecionar testes que minimizam o erro do tipo II, que é equivalente a maximizar a função poder. A função poder é definida por,
\begin{align*}
    \gamma_{\mathcal{C}}(\theta)=P_{\theta}[(X_{1},\ldots,X_{n})\in\mathcal{C}],~\theta\in W_{1}
\end{align*}
\end{block}
%\begin{block}{}
%\justifying
%Acrescentar os gráficos da função poder aqui!
%\end{block}
\end{frame}

\begin{frame}{}
\begin{definicao}
\justifying
Seja $\mathcal{C}$ um subconjunto do suporte de (\seqX). Dizemos que $\mathcal{C}$ é a melhor região critica de tamanho $\alpha$ se,
\begin{enumerate}
    \item[a)~] $\alpha=\max_{\theta\in W_{0}}P((X_{1},\ldots,X_{n})\in \mathcal{C})$
    \item[b)~] Para qualquer outra região $A$ com $\alpha=P_{\theta\in W_{0}}((X_{1},\ldots,X_{n})\in A),$
\begin{align*}
    P_{\theta}((X_{1},\ldots,X_{n})\in \mathcal{C})\geq P_{\theta}((X_{1},\ldots,X_{n})\in A)~\text{quando}~\theta\in W_{1}
\end{align*}
\end{enumerate}
\end{definicao}
\end{frame}

\section{Exemplos}
\begin{frame}{}
\begin{block}{}
\justifying
Considere $X\Sim$Binomial$(5,\theta).$ Seja $f(x;\theta)$ a função de probabilidade de $X$ e considere $H_{0}:\theta=\dfrac{1}{2}$ e $H_{1}:\theta=\dfrac{3}{4}.$ Além disso, considere:
\end{block}
\pause
\begin{block}{}
\justifying
\begin{table}[h]
    \centering
    \begin{tabular}{c|c|c|c|c|c|c}
         x& 0&1&2&3&4&5 \\
         \hline
                  & &&&&& \\
         $f(x;\frac{1}{2})$&$\frac{1}{32}$&$\frac{5}{32}$&$\frac{10}{32}$&$\frac{10}{32}$&$\frac{5}{32}$&$\frac{1}{32}$\\
                  & &&&&& \\
         \hline
         & &&&&& \\
         $f(x;\frac{3}{4})$&$\frac{1}{1024}$&$\frac{15}{1024}$&$\frac{90}{1024}$&$\frac{270}{1024}$&$\frac{405}{1024}$&$\frac{243}{1024}$\\
         & &&&&& \\
         \hline
         & &&&&& \\
         $\dfrac{f(x;\frac{1}{2})}{f(x;\frac{3}{4})}$&$32$&$\frac{32}{3}$&$\frac{32}{9}$&$\frac{32}{27}$&$\frac{32}{81}$&$\frac{32}{243}$\\
                 & &&&&& \\
         \hline
    \end{tabular}
\end{table}
\end{block}
\end{frame}

\begin{frame}{}
\begin{block}{}
\justifying
Considere o nível de significância do teste como \(\alpha = \frac{1}{32}\). Buscamos uma melhor região crítica de tamanho \(\alpha = \frac{1}{32}\). Se \(A_1 = \{x : x = 0\}\) ou \(A_2 = \{x : x = 5\}\), então \(P_{\{\theta=\frac{1}{2}\}}(X \in A_1) = P_{\{\theta=\frac{1}{2}\}}(X \in A_2) = \frac{1}{32}\) e não há outro subconjunto \(A_3\) do espaço \(\{x : x = 0, 1, 2, 3, 4, 5\}\) tal que \(P_{\{\theta=\frac{1}{2}\}}(X \in A_3) = \frac{1}{32}\). Portanto, ou \(A_1\) ou \(A_2\) é a melhor região crítica \(C\) de tamanho \(\alpha = \frac{1}{32}\) para testar \(H_0\) contra \(H_1\). 
\end{block}
\end{frame}

\begin{frame}{}
\begin{block}{}
\justifying
Observamos que \(P_{\{\theta=\frac{1}{2}\}}(X \in A_1) = \frac{1}{32}\) e \(P_{\{\theta=\frac{3}{4}\}}(X \in A_1) = \frac{1}{1024}\). Assim, se o conjunto \(A_1\) for usado como região crítica de tamanho \(\alpha = \frac{1}{32}\), temos a situação inaceitável de que a probabilidade de rejeitar \(H_0\) quando \(H_1\) é verdadeira (\(H_0\) é falsa) é muito menor do que a probabilidade de rejeitar \(H_0\) quando \(H_0\) é verdadeira.
\end{block}
\end{frame}

\begin{frame}{}
\begin{block}{}
\justifying
Por outro lado, se o conjunto \(A_2\) for usado como região crítica, então \(P_{\{\theta=\frac{1}{2}\}}(X \in A_2) = \frac{1}{32}\) e \(P_{\{\theta=\frac{3}{4}\}}(X \in A_2) = \frac{243}{1024}\). Ou seja, a probabilidade de rejeitar \(H_0\) quando \(H_1\) é verdadeira é muito maior do que a probabilidade de rejeitar \(H_0\) quando \(H_0\) é verdadeira. Certamente, esta é uma situação mais desejável, e na verdade, \(A_2\) é a melhor região crítica de tamanho \(\alpha = \frac{1}{32}\). Esta última afirmação decorre do fato de que quando \(H_0\) é verdadeira, existem apenas dois subconjuntos, \(A_1\) e \(A_2\), do espaço amostral, cada um com medida de probabilidade igual a \(\frac{1}{32}\), e do fato de que \(\frac{243}{1024} = P_{\{\theta=\frac{3}{4}\}}(X \in A_2) > P_{\{\theta=\frac{3}{4}\}}(X \in A_1) = \frac{1}{1024}\).
\end{block}
\end{frame}

\begin{frame}{}
\begin{block}{}
\justifying
Deve ser observado neste problema que a melhor região crítica \(C = A_2\) de tamanho \(\alpha = \frac{1}{32}\) é encontrada incluindo em \(C\) o ponto (ou pontos) em que \(\frac{f(x;\frac{1}{2})}{f(x;\frac{3}{4})}\) é mínimo. Assim, a razão \(\frac{f(x;\frac{1}{2})}{f(x;\frac{3}{4})}\), que é dada na última linha da tabulação acima, nos fornece uma ferramenta precisa para encontrar uma melhor região crítica \(C\) para determinados valores dados de \(\alpha\). 
\end{block}
\end{frame}

\begin{frame}{}
	\begin{block}{}
		\justifying
		Para ilustrar o último slide, suponha \(\alpha = \frac{6}{32}\). Quando \(H_0\) é verdadeira, cada um dos subconjuntos \(\{x : x = 0, 1\}\), \(\{x : x = 0, 4\}\), \(\{x : x = 1, 5\}\), \(\{x : x = 4, 5\}\) tem medida de probabilidade \(\frac{6}{32}\). Por cálculo direto, é encontrado que a melhor região crítica desse tamanho é \(\{x : x = 4, 5\}\). Isso reflete o fato de que a razão \(\frac{f(x;\frac{1}{2})}{f(x;\frac{3}{4})}\) tem seus dois valores mínimos em \(x = 4\) e \(x = 5\). O poder deste teste, que tem \(\alpha = \frac{6}{32}\), é \(P_{\{\theta=\frac{3}{4}\}}(X = 4, 5) = \frac{405}{1024} + \frac{243}{1024} = \frac{648}{1024}\).
	\end{block}
\end{frame}

\section{Teorema de Neyman-Pearson}
\begin{frame}{Teorema de Neyman-Pearson}
\vspace{-0.5cm}
\begin{Teorema}
\justifying
Sejam \seqX~ uma amostra aleatória de uma distribuição dependendo de um vetor de parâmetros $\theta.$ Denote,
$L(\theta,X)=\Prodi f(x_{i},\theta)$ a função de verossimilhança e $\undertilde{x}=(x_{1},\ldots,x_{n})^{\top}.$ Sejam $\theta^{'}$ e $\theta^{''}$ dois valores distintos de $\theta,~\Omega=\{\theta^{'},\theta^{''}\},$ e $k$ um número positivo, Seja $\mathcal{C}$ tal que,
\begin{enumerate}
    \item[a)~]$\dfrac{L(\theta^{'},\undertilde{x})}{L(\theta^{''},\undertilde{x})}\leq k,$ para cada $\undertilde{x}\in\mathcal{C},~\undertilde{X}=(X_{1},\ldots,X_{n})^{\top}$
    \item[b)~]$\dfrac{L(\theta^{'},\undertilde{x})}{L(\theta^{''},\undertilde{x})}\geq k,$ para cada $\undertilde{x}\in\mathcal{C}^{c}$
    \item[c)~]$\alpha=P_{\theta^{'}}(\undertilde{X}\in\mathcal{C}).$
\end{enumerate}
Então, $\mathcal{C}$ é a melhor região crítica de tamanho $\alpha$ para testar as hipóteses $H_{0}:\theta=\theta^{'}$ contra $H_{1}:\theta=\theta^{''}.$
\end{Teorema}
\end{frame}

\begin{frame}{Demonstração}
	\begin{block}{}
		\justifying
		É importante entender que deseja-se mostrar que, satisfeita as condições do teorema, $\mathcal{C}$ é a melhor região crítica de tamanho $\alpha.$ Isso pode ser escrito matematicamente como $P_{\theta^{''}}(\undertilde{X}\in\mathcal{C})>P_{\theta^{''}}(\undertilde{X}\in\mathcal{A})$ para qualquer outra região crítica $\mathcal{A}$ de tamanho $\alpha.$
	\end{block}
	\pause
	\begin{block}{}
		\justifying
		Ou seja, queremos mostrar que $$P_{\theta^{''}}(\undertilde{X}\in\mathcal{C})-P_{\theta^{''}}(\undertilde{X}\in\mathcal{A})>0.$$
	\end{block}
\end{frame}

\begin{frame}{Demonstração}
\begin{block}{}
\justifying
Consideramos o caso em que \seqX~ são variáveis continuas (o caso discreto é análogo).

\textbf{Notação Simplificada:} $\int_{\R}\cdots\int_{\R}L(\theta,\undertilde{X})d\undertilde{X}\equiv \int L(\theta)$
\end{block}
\pause
\begin{block}{}
\justifying
Queremos mostrar que,
\begin{align*}
    \int_{\mathcal{C}}L(\theta^{''})-\int_{A}L(\theta^{''})>0,~\text{em que}~A~\text{tem tamanho}~\alpha.
\end{align*}
Note que $\mathcal{C}=(\mathcal{C}\cap A)\cup (\mathcal{C}\cap A^{c})$ e $A=(\mathcal{C}\cap A)\cup (\mathcal{C}^{c}\cap A).$
\end{block}
\end{frame}

\begin{frame}{}
\begin{block}{}
\justifying
Logo,
{\footnotesize
\begin{align*}
\int_{\mathcal{C}}L(\theta^{''})-\int_{A}L(\theta^{''})&= \int_{\mathcal{C}\cap A}L(\theta^{''}) + \int_{\mathcal{C}\cap A^{c}}L(\theta^{''}) - \int_{\mathcal{C}\cap A}L(\theta^{''})-\int_{\mathcal{C}^{c}\cap A}L(\theta^{''})\\
&= \int_{\mathcal{C}\cap A^{c}}L(\theta^{''}) - \int_{\mathcal{C}^{c}\cap A}L(\theta^{''})
\end{align*}
}
\end{block}
\end{frame}

\begin{frame}{}
\begin{block}{}
\justifying
Segue que,
{\footnotesize
\begin{align*}
\int_{\mathcal{C}\cap A^{c}}L(\theta^{''}) -\int_{\mathcal{C}^{c}\cap A}L(\theta^{''})&\SetaUP{\text{Usando a parte a e b do teorema}}{\geq} \int_{\mathcal{C}\cap A^{c}}\dfrac{1}{k}L(\theta^{'})-\int_{\mathcal{C}^{c}\cap A}\dfrac{1}{k}L(\theta^{'})\\
&=\dfrac{1}{k}\left(\int_{\mathcal{C}\cap A^{c}}L(\theta^{'})-\int_{\mathcal{C}^{c}\cap A}L(\theta^{'})\right)\\
&=\dfrac{1}{k}\left(\int_{\mathcal{C}\cap A^{c}}L(\theta^{'})+\int_{\mathcal{C}\cap A}L(\theta^{'}) - \right.\\
&\left.-\int_{\mathcal{C}\cap A}L(\theta^{'})-\int_{\mathcal{C}^{c}\cap A}L(\theta^{'})\right)\\
&=\dfrac{1}{k}\left(\int_{\mathcal{C}}L(\theta^{'})-\int_{A}L(\theta^{'})\right)\Rightarrow \text{Sob}~H_{0}\\
&=\dfrac{1}{k}\left(\alpha-\alpha\right)=0
\end{align*}
}
\end{block}
\end{frame}

%\begin{frame}{}
%\begin{block}{Corolário}
%\justifying
%Sob as condições do teorema anterior, $\alpha\leq \gamma_{\mathcal{C}}(\theta),\forall \theta.$
%\end{block}
%\pause
%\begin{block}{Demonstração:}
%\justifying
%Tome um ensaio Bernoulli com probabilidade $\alpha$ de sucesso. Se acontecer sucesso, rejeite $H_{0}.$ Neste teste o nível de significância e a função poder são iguais a $\alpha.$ Pelo teorema anterior, $\alpha\leq \gamma_{\mathcal{C}}(\theta),\forall \theta.$
%\end{block}
%\end{frame}

\begin{frame}{Exemplo}
\begin{block}{}
\justifying
\seqX$\Sim N(\theta,1)$
\begin{align*}
    H_{0}:\theta&=\theta^{'}=0\\
    H_{1}:\theta&=\theta^{''}=1,\quad \quad L(\theta,\undertilde{X})=\left(\dfrac{1}{\sqrt{2\pi}}\right)^{n}\exp{\left\{-\frac{1}{2}\sum(X_{i}-\theta)^{2}\right\}}
\end{align*}
\end{block}
\end{frame}

\begin{frame}{}
\begin{block}{}
\justifying
$\dfrac{L(\theta^{'},\undertilde{x})}{L(\theta^{''},\undertilde{x})}\leq k,$
\begin{align*}
\dfrac{L(\theta^{'},\undertilde{x})}{L(\theta^{''},\undertilde{x})}&=\exp{\left\{-\Sumi x_{i}+\frac{n}{2}\right\}}\leq k\\
&\Rightarrow -\sum x_{i}+\frac{n}{2}\leq k^{'}=\log{k}\\
&\Rightarrow \underbrace{\sum x_{i}\geq k^{''}}_{\text{Região Crítica}}
\end{align*}
\end{block}
\end{frame}

\begin{frame}{}
\begin{block}{}
\justifying
Portanto, a melhor região crítica $C$ é o conjunto de todas as amostras $(x_1, x_2, \ldots, x_n)$ para as quais a soma dos $x_i$ é maior ou igual a $k^{''}$. O valor de $k^{''}$ pode ser determinado de modo que o tamanho da região crítica seja igual ao nível de significância desejado $\alpha$. Essa região crítica é dada por:

\[
C = \{(x_1, x_2, \ldots, x_n) : \sum_{i=1}^n x_i \geq k^{''}\}.
\]

O teste pode ser baseado na estatística $\Bar{X}$, pois $\sum_{i=1}^n x_i \geq k^{''}$ é equivalente a $\Bar{X}\geq c_{1}.$ Se $H_0$ for verdadeira, ou seja, $\theta = \theta_0 = 0$, então $\Bar{X}$ segue uma distribuição $N(0, 1/n)$. Dado um nível de significância $\alpha$, podemos calcular $c_{1}$ em R como $c_{1} = qnorm(1-\alpha, 0, 1/\sqrt{n})$.



\end{block}
\end{frame}

\begin{frame}{}
\begin{block}{}
\justifying
Portanto, se os valores experimentais de $X_1, X_2, \ldots, X_n$ forem, respectivamente, $x_1, x_2, \ldots, x_n$, podemos calcular $\Bar{x} = \frac{1}{n} \sum_{i=1}^n x_i$. Se $\Bar{x} \geq c_{1}$, a hipótese simples $H_0 : \theta = \theta_0 = 0$ será rejeitada no nível de significância $\alpha$; caso contrário, a hipótese $H_0$ será aceita. A probabilidade de rejeitar $H_0$ quando $H_0$ é verdadeira é igual a $\alpha$, o nível de significância. A probabilidade de rejeitar $H_0$ quando $H_0$ é falsa, ou seja, o valor de poder do teste quando $\theta = \theta_1 = 1$, pode ser calculada como indicado na equação fornecida.
\end{block}
\begin{block}{}
\justifying
\begin{align*}
    P_{H1}(\Bar{X} \geq c_1) = \int_{c_1}^{\infty} \frac{1}{\sqrt{2\pi}} \frac{1}{\sqrt{1/n}} \exp\left(-\frac{(\Bar{x} - 1)^2}{2(1/n)}\right) \, d\Bar{x}.
\end{align*}
\end{block}
\pause
\begin{block}{}
\justifying
Por exemplo, se $n = 25$ e $\alpha$ for $0.05$, então $c_1 = \text{qnorm}(0.95, 0, 1/5) = 0.329$, usando R. Portanto, o poder do teste para detectar $\theta = 1$, é calculado por $1 - \text{pnorm}(0.329, 1, 1/5) = 0.9996$.
\end{block}
\end{frame}


\begin{frame}{}
\begin{block}{Exercício 8.1.2}
\justifying
Vamos considerar o problema de testar a hipótese simples $H_0: \theta = \theta_0 = 2$ contra a hipótese alternativa simples $H_1: \theta = \theta_1 = 4$ para a variável aleatória $X$, que possui a função de densidade de probabilidade (pdf) dada por $f(x; \theta) = \frac{1}{\theta} e^{-\frac{x}{\theta}}$, onde $0 < x < \infty$ e zero caso contrário. Seja $X_1$ e $X_2$ representam uma amostra aleatória de tamanho 2 desta distribuição. Mostre que o melhor teste de $H_0$ contra $H_1$ é usando a estatística $X_1 + X_2$!
\end{block}
\end{frame}

\begin{frame}{}
\begin{block}{}
\justifying
A razão de verossimilhança é dada por:

\begin{align*}
    \frac{L(\theta_0; X_1, X_2)}{L(\theta_1; X_1, X_2)} &= \frac{\frac{1}{\theta_0} e^{-\frac{X_1}{\theta_0}} \frac{1}{\theta_0} e^{-\frac{X_2}{\theta_0}}}{\frac{1}{\theta_1} e^{-\frac{X_1}{\theta_1}} \frac{1}{\theta_1} e^{-\frac{X_2}{\theta_1}}}\\ 
&= 4e^{-\left(\dfrac{X_1+X_2}{4}\right)},
\end{align*}
em que utilizamos as hipóteses $H_0: \theta = 2$ e $H_1: \theta = 4.$ Agora, queremos encontrar a melhor região crítica $C$ de tamanho $\alpha$ para testar $H_0$ contra $H_1$. Neste caso, $\alpha$ representa o nível de significância do teste.
\end{block}
\end{frame}


\begin{frame}{}
\begin{block}{}
\justifying
Usando a razão de verossimilhança, temos que:

\[
\frac{L(\theta_0; X_1, X_2)}{L(\theta_1; X_1, X_2)} \leq k \quad \text{se, e somente se,} \quad 4e^{-\left(\dfrac{X_1+X_2}{4}\right)} \leq k,
\]

ou seja, $X_1 + X_2 \geq k_{3}$. Portanto, a região crítica $C$ pode ser definida como:

\[
C = \{(X_1, X_2) : X_1 + X_2 \geq k_{3}\},
\]

em que $k$ é escolhido de forma a controlar o nível de significância $\alpha$. O teste de hipóteses consiste em verificar se a amostra cai dentro ou fora da região crítica $C$. Portanto, o melhor teste usa a estatística $X_1 + X_2$ e a melhor região crítica $C$ é definida por $C=\{(x_{1},x_{2})/x_{1}+x_{2}\geq k\}$.
\end{block}
\end{frame}

\begin{frame}{\Home}
\begin{block}{}
\justifying

\begin{itemize}
    \item \textbf{Exercícios da seção 8.1:} 2, 3, 5, 6, 8, 9, 10.
    %\item \textbf{Exercícios da seção 8.2:} 1,3,5,6,11,13.
    %\item \textbf{Exercícios da seção 8.3:} 4,6,8,11,12,13.
\end{itemize}
\end{block}
\nocite{hogg}
\end{frame}

\begin{frame}[allowframebreaks]
\frametitle{\bf Referências}
\printbibliography
\end{frame}


\end{document}

