\documentclass[12pt]{beamer}

\usepackage[brazil]{babel}
\usepackage[utf8]{inputenc}
\usepackage[T1]{fontenc}
\usepackage{animate}
\usepackage{amsbsy}
\usepackage{amsfonts}
\usepackage{amsmath}
\usepackage{amssymb}
\usepackage{amsthm}
\setbeamertemplate{theorems}[numbered] % to number
\usepackage[toc,page,title,titletoc]{appendix}
\usepackage{dsfont}
\usepackage{esvect}
\usepackage[labelfont=bf]{caption}
%\usepackage{subcaption}
\usepackage{float}
\usepackage[Glenn]{fncychap}%Sonny %Conny %Lenny %Glenn %Renje %Bjarne %Bjornstrup
\usepackage{graphicx}
\usepackage{subfig}
\usepackage{indentfirst}%Para indentar os paragrafos automáticamente
\usepackage{lipsum}
\usepackage{longtable}
\usepackage{mathtools}
\usepackage{listings}%Inserir codigo do R no latex
\usepackage{multirow}
\usepackage{multicol}
\usepackage{csquotes}
\usepackage[citestyle=authoryear,maxcitenames=2,terseinits=true,natbib=true, style=abnt, maxbibnames=99]{biblatex}
\addbibresource{../Referencias/Referencias.bib}
\usepackage[figuresright]{rotating}
\usepackage{spalign}
\usepackage{pgfplots}
\pgfplotsset{compat=1.17}
\usepackage{tikz}
\usepackage{fontawesome}
\usepackage{color, colortbl}
\usepackage{url}
\usepackage{cancel}
\usepackage{accents}
\usepackage{bm}
\usepackage{ragged2e}%para justificar o texto dentro de algum ambiente
\definecolor{Gray}{gray}{0.9}
\definecolor{LightCyan}{rgb}{0.88,1,1}


\usepackage[all]{xy}
\usepackage{hyperref,bookmark}
\hypersetup{
  colorlinks=true,
  linkcolor=blue,
  citecolor=red,
  filecolor=blue,
  urlcolor=blue,
}

\usetheme{Madrid}
\usecolortheme[RGB={193,0,0}]{structure}

%\setbeamertemplate{footline}[frame number]
%\setbeamertemplate{footline}[text line]{%
%  \parbox{\linewidth}{\vspace*{-8pt}\hfill\date{}\hfill\insertshortauthor\hfill\insertpagenumber}}
\beamertemplatenavigationsymbolsempty
\renewcommand{\vec}[1]{\mbox{\boldmath$#1$}}
\newtheorem{Teorema}{Teorema}
\newtheorem{Proposicao}{Proposição}
\newtheorem{definicao}{Definição}
\newtheorem{Corolario}{Corolário}
\newtheorem{Demonstracao}{Demonstração}
\newcommand{\bx}{\ensuremath{\bar{x}}}
\newcommand{\Ho}{\ensuremath{H_{0}}}
\newcommand{\Hi}{\ensuremath{H_{1}}}
\newcommand{\at}[2][]{#1|_{#2}}
\newcommand\xuparrow[1][2ex]{%
   \mathrel{\rotatebox{-90}{$\xleftarrow{\rule{#1}{0pt}}$}}
}
\apptocmd{\frame}{}{\justifying}{} % Allow optional arguments after frame.

\makeatletter
\setbeamertemplate{footline}
{
  \leavevmode%
  \hbox{%
  \begin{beamercolorbox}[wd=.3\paperwidth,ht=2.25ex,dp=1ex,center]{author in head/foot}%
    \usebeamerfont{author in head/foot}\mytext
  \end{beamercolorbox}%
  \begin{beamercolorbox}[wd=.3\paperwidth,ht=2.25ex,dp=1ex,center]{title in head/foot}%
    \usebeamerfont{title in head/foot}\mytextt
  \end{beamercolorbox}%
  \begin{beamercolorbox}[wd=.35\paperwidth,ht=2.25ex,dp=1ex,right]{site in head/foot}%
    \usebeamerfont{site in head/foot}\mytexttt\hspace*{2em}
    \insertframenumber{} / \inserttotalframenumber\hspace*{2ex} 
  \end{beamercolorbox}}%
  \vskip0pt%
}
\makeatother

\providecommand{\arcsin}{} \renewcommand{\arcsin}{\hspace{2pt}\textrm{arcsen}}
\providecommand{\sin}{} \renewcommand{\sin}{\hspace{2pt}\textrm{sen}}
\newcommand{\N}{\rm I\!N}
\newcommand{\I}{\rm I\!I}
\newcommand{\R}{\rm I\!R}
\newcommand{\Sim}{\overset{\text{iid}}{\sim}}
\newcommand{\Lim}{{\displaystyle \lim_{n\to\infty}}}
\newcommand{\LimInf}{{\displaystyle \liminf_{n\to\infty}}}
\newcommand{\rightLim}{\xrightarrow[n\rightarrow\infty]{}}
\newcommand{\Sumi}{{\displaystyle \sum_{i=1}^{n}}}
\newcommand{\Int}{{\displaystyle \int_{-\infty}^{+\infty}}}
\newcommand{\ConvD}{\overset{D}{\rightarrow}}
\newcommand{\ConvP}{\overset{P}{\rightarrow}}
\newcommand{\Prodi}{{\displaystyle \prod_{i=1}^{n}}}
\newcommand{\SetaUP}[2]{\underset{\mathclap{\substack{\xuparrow[30pt] \\ #1}}}{#2}}
%\newcommand{\SetaInclinada}[2]{\underset{\mathclap{\substack{\rotatebox{135}{\xuparrow[30pt] \\ #1}}}}{#2}}
\newcommand{\Home}{\begin{tikzpicture}
\node[scale=2] at (3,4) {\text{Para}~\faHome};
\end{tikzpicture}}
\newcommand{\vecX}{\boldsymbol{X}}
\newcommand{\Implica}[1]{\xRightarrow{#1}}
\newcommand{\SeSe}{\iff}
\newcommand{\EscoreA}{\dfrac{\partial}{\partial\theta}\log{f(x,\theta)}}
\newcommand{\EscoreB}{\dfrac{\partial^{2}}{\partial\theta^{2}}\log{f(x,\theta)}}
\newcommand{\cqd}{\text{cqd}~\blacksquare}
\newcommand{\seqX}{$X_{1},\ldots,X_{n}$}
\newcommand{\seqY}{$Y_{1},\ldots,Y_{n}$}
\newcommand{\tend}[1]{\hbox{\oalign{$\bm{#1}$\crcr\hidewidth$\scriptscriptstyle\bm{\sim}$\hidewidth}}}

%\newtheorem{Teorema}{Teorema}
%\newtheorem{Proposicao}{Proposição}
%\newtheorem{Definicao}{Definição}
%\newtheorem{Corolario}{Corolário}
%\newtheorem{Demonstracao}{Demonstração}

\titlegraphic{\hspace*{8cm}\href{https://fsbmat-ufv.github.io/}{\includegraphics[width=2cm]{figs/mylogo.png}}
}

%Continuar a numeracao em slides diferentes
\newcounter{saveenumi}
\newcommand{\seti}{\setcounter{saveenumi}{\value{enumi}}}
\newcommand{\conti}{\setcounter{enumi}{\value{saveenumi}}}

\resetcounteronoverlays{saveenumi}

% Layout da pagina
\hypersetup{pdfpagelayout=SinglePage}

%Para o \pause funcionar dentro do ambiente align
\makeatletter
\let\save@measuring@true\measuring@true
\def\measuring@true{%
  \save@measuring@true
  \def\beamer@sortzero##1{\beamer@ifnextcharospec{\beamer@sortzeroread{##1}}{}}%
  \def\beamer@sortzeroread##1<##2>{}%
  \def\beamer@finalnospec{}%
}
\makeatother
\setLayoutColor{1} 

\title{Inferência Estatística II}
\author{Prof. Fernando de Souza Bastos\texorpdfstring{\\ fernando.bastos@ufv.br}{}}
\institute{Departamento de Estatística\texorpdfstring{\\ Programa de Pós-Graduação em Estatística Aplicada e Biometria}\texorpdfstring{\\ Universidade Federal de Viçosa}{}\texorpdfstring{\\ Campus UFV - Viçosa}{}}
\date{}
\newcommand\mytext{Aula 13}
\newcommand\mytextt{Fernando de Souza Bastos}
\newcommand\mytexttt{\url{https://est711.github.io/}}


\begin{document}
%\SweaveOpts{concordance=TRUE}

\frame{\titlepage}

\begin{frame}{}
\frametitle{\bf Sumário}
\tableofcontents
\end{frame}

\section{Testes Uniformemente Mais Poderosos}
\begin{frame}{}
\begin{block}{}
\justifying
Considere a função densidade de probabilidade (pdf)
\begin{align*}
f(x; \theta) = \begin{cases}
\frac{1}{\theta} e^{-x/\theta}, & 0 < x < \infty \\
0, & \text{Caso Contrário}
\end{cases}
\end{align*}
Deseja-se testar a hipótese simples $H_0: \theta = 2$ contra a hipótese alternativa composta $H_1: \theta > 2$. Assim, $\Omega = \{\theta: \theta \geq 2\}$. Uma amostra aleatória, $X_1, X_2$, de tamanho $n = 2$ é utilizada, assumindo $\alpha=0,05$ a região crítica é $C = \{(x_1, x_2): x_1 + x_2 \geq k\}$ em que $k$ é tal que $P_{\theta=2}(x_1 + x_2 \geq k)=0,05.$ Usando {k=\texttt qgamma(0.95,2,1/2)} obtemos $k\approx 9.5$. Observe que assumi que $T = X_1 + X_2$ segue uma distribuição gama com parâmetros $k = 2$ e taxa $\lambda = 1/\theta$.  
\end{block}
\end{frame}
%Foi mostrado nos exercícios citados que o nível de significância do teste é aproximadamente $0.05$ e o poder do teste quando $\theta = 4$ é aproximadamente $0.31$. 

\begin{frame}{}
	\begin{block}{}
		\justifying
		Usando que $T \sim \text{Gamma}(2, 1/\theta),$ a função poder é dada por:
		\begin{align*}
			\gamma(\theta) &= P_\theta(T > 9.5) \\
			&= 1 - F_T(9.5; 2, 1/\theta),
		\end{align*}
		onde $F_T(\cdot)$ é a função de distribuição acumulada da variável gama. Como
		\[
		F_T(t; 2, 1/\theta) = 1 - e^{-t/\theta}\left(1 + \frac{t}{\theta}\right),
		\]
		segue que
		\[
		\boxed{\gamma(\theta) = e^{-9.5/\theta}\left(1 + \frac{9.5}{\theta}\right)}, \quad \theta \geq 2.
		\]
	\end{block}
\end{frame}

\begin{frame}{}
	\begin{block}{De outra forma:}
		\justifying
		A função poder $\gamma(\theta)$ do teste para todos os $\theta \geq 2$ é:
		\begin{align*}
			\gamma(\theta) &= P_\theta(X_{1}+X_{2} > 9.5)\\ 
			&=1 - \int_0^{9.5} \int_0^{9.5-x_2} \frac{1}{\theta^2} e^{-\frac{x_1+x_2}{\theta}} dx_1dx_2 \\
			&= \left(\dfrac{\theta+9.5}{\theta}\right)e^{-9.5/\theta}, \quad \theta\geq 2.
		\end{align*}
		Por exemplo, $\gamma(2) = 0.05$, $\gamma(4) = 0.31$ e $\gamma(9.5) = \frac{2}{e} \approx 0.74$. O conjunto $C = \{(x_1, x_2): x_1 + x_2 \geq 9,5\}$ é uma melhor região crítica de tamanho $0.05$ para testar a hipótese simples $H_0: \theta = 2$ contra cada hipótese simples na hipótese composta $H_1: \theta > 2$.
	\end{block}
\end{frame}

\begin{frame}{}
	\begin{block}{Mesmo Exemplo (variação com $\theta_1 = 6$)}
		\justifying
		Vamos considerar o problema de testar a hipótese simples $H_0: \theta = \theta_0 = 2$ contra a hipótese alternativa simples $H_1: \theta = \theta_1 = 6.$
	\end{block}
\pause
	\begin{block}{Razão de verossimilhança}
	\justifying
	A razão de verossimilhança é dada por:
	\begin{align*}
		\frac{L(\theta_0; X_1, X_2)}{L(\theta_1; X_1, X_2)}
		&=  
		\left(\frac{\theta_1}{\theta_0}\right)^2 
		e^{-(X_1+X_2)\left(\frac{1}{\theta_0}-\frac{1}{\theta_1}\right)}.
	\end{align*}
	
	Substituindo $\theta_0 = 2$ e $\theta_1 = 6$, obtemos:
	\[
	\frac{L(2)}{L(6)} = 9\, e^{-(X_1+X_2)\left(\frac{1}{2}-\frac{1}{6}\right)}
	= 9\, e^{-(X_1+X_2)/3}.
	\]
\end{block}
\end{frame}



\begin{frame}{}
	\begin{block}{Região crítica ótima}
		\justifying
		O teste de Neyman–Pearson rejeita $H_0$ quando a razão de verossimilhança é pequena, ou seja:
		\[
		\frac{L(2)}{L(6)} \leq k
		\quad \Longleftrightarrow \quad 
		9e^{-(X_1+X_2)/3} \leq k.
		\]
	Assim, a melhor região crítica de tamanho $\alpha$ é:
		\[
		\boxed{C = \{(x_1, x_2): x_1 + x_2 \geq c\}},
		\]
		onde $c$ é escolhido tal que $P_{\theta=2}(X_1 + X_2 \geq c) = \alpha.$ Pelo mesmo motivo das hipóteses anteriores, se $\alpha=0.05$ obtemos $c \approx 9{,}49$.
	\end{block}
\end{frame}

%\begin{frame}{}
%	\begin{block}{Função poder para $\theta_1 = 6$}
%		\justifying
%		A estatística de teste é $T = X_1 + X_2$, e sabemos que
%		\[
%		T \mid \theta \sim \text{Gamma}(2, \text{escala} = \theta).
%		\]
%		A função poder do teste é definida por:
%		\[
%		\gamma(\theta) = P_\theta(T > 9.5) = 1 - F_T(9.5; 2, \text{escala} = \theta),
%		\]
%		em que $F_T$ é a função de distribuição acumulada da variável gama. Para a parametrização por taxa ($\lambda = 1/\theta$):
%		\[
%		F_T(t; 2, 1/\theta) = 1 - e^{-t/\theta}\left(1 + \frac{t}{\theta}\right).
%		\]
%		Logo,
%		\[
%		\boxed{\gamma(\theta) = e^{-9.5/\theta}\left(1 + \frac{9.5}{\theta}\right)}.
%		\]
%	\end{block}
%\end{frame}

\begin{frame}{}
	\begin{block}{Exemplo numérico: $\theta_1 = 6$}
		\justifying
		Substituindo $\theta = 6$ na expressão da função poder, já calculada para todo $\theta\geq 2$, obtemos:
		\begin{align*}
		\gamma(6) = e^{-9.5/6}\left(1 + \frac{9.5}{6}\right)
		= e^{-1.5833}(2.5833)\\
		\approx 0.2059 \times 2.5833
		\approx 0.532.
		\end{align*}
		
		Portanto, a probabilidade de rejeitar corretamente $H_0$ quando $\theta = 6$ é aproximadamente:
		\[
		\boxed{\gamma(6) \approx 0.53.}
		\]
		
		Em R, o mesmo resultado é obtido por:
		{\texttt pgamma(9.5, shape = 2, rate = 1/6, lower.tail = FALSE) = 0.532}
		Assim, o teste tem potência de aproximadamente $53\%$ quando $\theta = 6$.
	\end{block}
\end{frame}

\begin{frame}{}
	\begin{block}{Justificativa: o poder em $\theta = 2$ é o supremo sob $H_0$}
		\justifying
		Por definição, o \textbf{nível de significância} $\alpha$ de um teste é o maior valor da probabilidade de rejeitar $H_0$ quando $H_0$ é verdadeira.%, ou seja:
		%\[
		%\alpha = \sup_{\theta \in \Omega_0} P_\theta(T \in C),
		%\]
		%em que $C$ é a região crítica e $\Omega_0$ é o conjunto de parâmetros sob $H_0$.
		
		No problema em questão, temos:
		\[
		H_0: \theta = 2, \quad H_1: \theta > 2,
		\]
		portanto, $\Omega_0 = \{2\}$ e a função poder é:
		\[
		\gamma(\theta) = e^{-9.5/\theta}\left(1 + \frac{9.5}{\theta}\right).
		\]
		Como $\gamma(\theta)$ é \textbf{crescente} em $\theta$ (maior $\theta$ implica maior chance de rejeitar $H_0$), temos:
		\[
		\sup_{\theta \in \Omega_0} \gamma(\theta) = \gamma(2).
		\]
	\end{block}
\end{frame}

\begin{frame}{}
	\begin{block}{Interpretação prática}
		\justifying
		O valor $\gamma(2)$ representa a probabilidade de rejeitar $H_0$ quando ela é verdadeira, isto é, o \textbf{erro tipo I}. Por construção da região crítica,
		\[
		\gamma(2) = P_{\theta=2}(T > 9.5) = 0.05 = \alpha.
		\]
		Portanto, o poder em $\theta = 2$ é o \textbf{supremo da função poder sob $H_0$}, e o teste é dito ter \textbf{tamanho 0.05}. Essa propriedade garante que o teste é válido, isto é, não ultrapassa o nível de significância especificado, sendo também o mais poderoso entre todos os testes com esse mesmo tamanho.
	\end{block}
\end{frame}


\begin{frame}{}
\begin{block}{}
\justifying
%O exemplo anterior ilustra um teste de uma hipótese simples $H_0$ que é o melhor teste de $H_0$ contra todas as hipóteses simples na hipótese alternativa composta $H_1$. Agora definimos uma região crítica, quando ela existe, que é a melhor região crítica para testar uma hipótese simples $H_0$ contra uma hipótese alternativa composta $H_1$. Parece desejável que essa região crítica seja a melhor região crítica para testar $H_0$ contra cada hipótese simples em $H_1$. Ou seja, a função poder do teste que corresponde a essa região crítica deve ser pelo menos tão grande quanto a função poder de qualquer outro teste com o mesmo nível de significância para cada hipótese simples em $H_1$.
\begin{itemize}
	\item \textbf{Resumo do exemplo anterior:}
	\begin{itemize}
		\item exemplo anterior ilustra um teste de uma hipótese simples $H_0$ que é o melhor teste de $H_0$ contra todas as hipóteses simples na hipótese alternativa composta $H_1$.
	\end{itemize}
	
	\item \textbf{Queremos agora definir uma Região Crítica para uma hipótese composta que:}
	\begin{itemize}
		\item Seja a ``melhor'' para testar $H_0$ contra $H_1$.\pause
		\item Melhor região crítica significa:
		\begin{itemize}
			\item A função poder do teste deve ser \textbf{maior ou igual} à de qualquer outro teste para cada hipótese simples em $H_1$.\pause
			\item Deve ter o \textbf{mesmo nível de significância}.
		\end{itemize}
	\end{itemize}
\end{itemize}

\end{block}
\end{frame}

\begin{frame}{}
\begin{definicao}
\justifying
A região crítica $C$ é uma região crítica uniformemente mais poderosa (UMP) de tamanho $\alpha$ para testar a hipótese simples $H_0$ contra uma hipótese alternativa composta $H_1$ se o conjunto $C$ é a melhor região crítica de tamanho $\alpha$ para testar $H_0$ contra cada hipótese simples em $H_1$. Um teste definido por essa região crítica $C$ é chamado de teste uniformemente mais poderoso (UMP), com nível de significância $\alpha$, para testar a hipótese simples $H_0$ contra a hipótese alternativa composta $H_1$.
\end{definicao}
\end{frame}

\begin{frame}{}
\begin{block}{}
\justifying
Como será visto posteriormente, testes uniformemente mais poderosos nem sempre existem. No entanto, quando existem, o teorema de Neyman-Pearson fornece uma técnica para encontrá-los. Alguns exemplos ilustrativos são dados aqui.
\end{block}
\end{frame}

\section{Exemplo 1}
\begin{frame}{Exemplo}
\begin{block}{}
\justifying
Seja $X_1, X_2, \ldots, X_n$ uma amostra aleatória de uma distribuição normal $N(0, \theta)$, onde a variância $\theta$ é um número positivo desconhecido. Será demonstrado que existe um teste uniformemente mais poderoso com nível de significância $\alpha$ para testar a hipótese simples $H_0: \theta = \theta_0$, onde $\theta_0$ é um número positivo fixo, em oposição à hipótese alternativa composta $H_1: \theta > \theta_0$. Assim, $\Omega = \{\theta : \theta \geq \theta_0\}$.

A função de densidade conjunta de $X_1, X_2, \ldots, X_n$ é dada por
\begin{equation}
L(\theta; x_1, x_2, \ldots, x_n) = \frac{1}{(2\pi\theta)^{n/2}} \exp\left(-\frac{1}{2\theta}\sum_{i=1}^{n}x_i^2\right).
\end{equation}
\end{block}
\end{frame}

\begin{frame}{}
\begin{block}{}
\justifying
Seja $\theta_1$ representando um número maior que $\theta_0$, e $k$ denote um número positivo. Seja $C$ o conjunto de pontos onde
\begin{equation}
\frac{L(\theta_0; x_1, x_2, \ldots, x_n)}{L(\theta_1; x_1, x_2, \ldots, x_n)} \leq k,
\end{equation}
isto é, o conjunto de pontos onde
\begin{equation}
\frac{\exp\left(-\frac{1}{2\theta_0}\sum_{i=1}^{n}x_i^2\right)}{\exp\left(-\frac{1}{2\theta_1}\sum_{i=1}^{n}x_i^2\right)} \leq k,
\end{equation}
ou equivalentemente,
\begin{align*}
\sum_{i=1}^{n}x_i^2 \geq \dfrac{2\theta_1\theta_0}{\theta_1-\theta_0}\left(\dfrac{n}{2}\log{\left(\dfrac{\theta_1}{\theta_0}\right)}-\log{k}\right)=c
\end{align*}
\end{block}
\end{frame}

\begin{frame}{}
\begin{block}{}
\justifying
O conjunto $C = \{(x_1, x_2, \ldots, x_n) : \sum_{i=1}^{n}x_i^2 \geq c\}$ é então uma melhor região crítica para testar a hipótese simples $H_0: \theta = \theta_0$ em oposição à hipótese simples $\theta = \theta_1$. Resta determinar $c$ de forma que esta região crítica tenha o tamanho desejado $\alpha$. Se $H_0$ for verdadeira, a variável aleatória
\begin{equation}
\frac{\sum_{i=1}^{n}X_i^2}{\theta_0}
\end{equation}
tem uma distribuição qui-quadrado com $n$ graus de liberdade. Uma vez que $\alpha = P_{\theta_0} \left(\frac{\sum_{i=1}^{n}X_i^2}{\theta_0} \geq \frac{c}{\theta_0}\right)$, $c/\theta_0$ pode ser calculado, por exemplo, usando o código R 

\begin{align*}
    {\texttt qchisq(1 - \alpha, n)}.
\end{align*}

Então, $C = \{(x_1, x_2, \ldots, x_n) : \sum_{i=1}^{n}x_i^2 \geq c\}$ é uma melhor região crítica de tamanho $\alpha$ para testar $H_0: \theta = \theta_0$ em oposição à hipótese $\theta = \theta_1$.


\end{block}
\end{frame}

\begin{frame}{}
\begin{block}{}
\justifying
Além disso, para cada número $\theta_1$ maior que $\theta_0$, o argumento anterior se mantém. Ou seja, $C = \{(x_1, \ldots, x_n) : \sum_{i=1}^{n}x_i^2 \geq c\}$ é uma região crítica uniformemente mais poderosa de tamanho $\alpha$ para testar $H_0: \theta = \theta_0$ em oposição a $H_1: \theta > \theta_0$. Se $x_1, x_2, \ldots, x_n$ denotam os valores experimentais de $X_1, X_2, \ldots, X_n$, então $H_0: \theta = \theta_0$ é rejeitada ao nível de significância $\alpha$, e $H_1: \theta > \theta_0$ é aceita se $\sum_{i=1}^{n}x_i^2 \geq c$; caso contrário, $H_0: \theta = \theta_0$ é aceita.

Se, na discussão anterior, tomarmos $n = 15$, $\alpha = 0,05$ e $\theta_0 = 3$, então as duas hipóteses são $H_0: \theta = 3$ e $H_1: \theta > 3$. Usando R, $c/3$ é calculado por \texttt{qchisq(0.95, 15)} $= 24,996$. Portanto, $c = 74,988$.

\end{block}
\end{frame}

\section{Exemplo 2}
\begin{frame}{Exemplo 2}
\begin{block}{}
\justifying
Seja $X_1, X_2, \ldots, X_n$ uma amostra aleatória de uma distribuição normal $N(\theta, 1)$, onde $\theta$ é desconhecido. Será demonstrado que não existe um teste uniformemente mais poderoso para a hipótese simples $H_0 : \theta = \theta_0$, onde $\theta_0$ é um número fixo, em oposição à hipótese alternativa composta $H_1 : \theta \neq \theta_0$. Assim, $\Omega = \{\theta : -\infty < \theta < \infty\}$.
\end{block}
\pause
\begin{block}{}
	\justifying
	Seja $\theta_1$ um número diferente de $\theta_0$. Seja $k$ um número positivo e considere
	\[
	\dfrac{\frac{1}{(2\pi)^{n/2}} \exp\left(-\frac{1}{2}{\displaystyle \sum_{i=1}^{n}(x_i - \theta_{0})^2}\right) }{\frac{1}{(2\pi)^{n/2}} \exp\left(-\frac{1}{2}{\displaystyle \sum_{i=1}^{n}(x_i - \theta_1)^2}\right)} \leq k.
	\]
\end{block}
\end{frame}

\begin{frame}{}
\begin{block}{}
\justifying
A desigualdade anterior pode ser escrita como
\[
(\theta_{0} - \theta_{1})\sum_{i=1}^{n}x_i \leq \dfrac{2\log k -n(\theta_{1}^{2}-\theta_{0}^{2})}{2}.
\]
\end{block}
\pause
\begin{block}{}
	\justifying
	\begin{itemize}
		\item Se $\theta_{0} > \theta_1$ essa última desigualdade é equivalente a
		\[
		\sum_{i=1}^{n}x_i \leq \dfrac{2\log k -n(\theta_{1}^{2}-\theta_{0}^{2})}{2(\theta_{0} - \theta_{1})},
		\]\pause
		\item Por outro lado, se $\theta_{0} < \theta_1,$ é equivalente a
		\[
		\sum_{i=1}^{n}x_i \geq \dfrac{2\log k -n(\theta_{1}^{2}-\theta_{0}^{2})}{2(\theta_{0} - \theta_{1})}.
		\] 
	\end{itemize}
\end{block}
\end{frame}

\begin{frame}{}
\begin{block}{}
\justifying

%A primeira dessas duas expressões define uma melhor região crítica para testar $H_0 : \theta = \theta_0$ contra a hipótese $\theta = \theta_1$, desde que $\theta_1 > \theta$, enquanto a segunda expressão define uma melhor região crítica para testar $H_0 : \theta = \theta_0$ contra a hipótese $\theta = \theta_1$, desde que $\theta_1 < \theta$. Ou seja, uma melhor região crítica para testar a hipótese simples contra uma hipótese simples alternativa, digamos $\theta = \theta_0 + 1$, não serve como uma melhor região crítica para testar $H_0 : \theta = \theta_0$ contra a hipótese simples alternativa $\theta = \theta_0 - 1$. Por definição, então, não existe um teste uniformemente mais poderoso no caso em consideração.
%
%Deve-se notar que se a hipótese alternativa composta fosse unidirecional, seja $H_1 : \theta > \theta_0$ ou $H_1 : \theta < \theta_0$, um teste uniformemente mais poderoso existiria em cada instância.
\begin{itemize}
	\item \textbf{Definição de Melhor Região Crítica:}
	\begin{itemize}
		\item \textbf{Primeira expressão:} Melhor região crítica para testar $H_0 : \theta = \theta_0$ contra $H_1 : \theta > \theta_0$.\pause
		\item \textbf{Segunda expressão:} Melhor região crítica para testar $H_0 : \theta = \theta_0$ contra $H_1 : \theta < \theta_0$.
	\end{itemize}
	\pause
	\item \textbf{Limitações da Melhor Região Crítica:}
	\begin{itemize}
		\item Uma melhor região crítica para $\theta = \theta_0 + 1$ não serve para testar $H_0 : \theta = \theta_0$ contra $\theta = \theta_0 - 1$.\pause
		\item Portanto, não existe um teste uniformemente mais poderoso neste caso.
	\end{itemize}
	\pause
	\item \textbf{Teste Uniformemente Mais Poderoso:}
	\begin{itemize}
		\item Caso a hipótese alternativa fosse composta unidirecional:
		\begin{itemize}
			\item $H_1 : \theta > \theta_0$ ou $H_1 : \theta < \theta_0$. Um teste uniformemente mais poderoso existiria para cada instância.
		\end{itemize}
	\end{itemize}
\end{itemize}
\end{block}
\end{frame}

\begin{frame}{\Home}
\begin{block}{}
\justifying

\begin{itemize}
    \item \textbf{Exercícios da seção 8.2:} 1,3,5,6,11,13.
\end{itemize}
\end{block}
\nocite{hogg}
\end{frame}

\begin{frame}[allowframebreaks]
\frametitle{\bf Referências}
\printbibliography
\end{frame}


\end{document}

