\documentclass[12pt]{beamer}

\usepackage[brazil]{babel}
\usepackage[utf8]{inputenc}
\usepackage[T1]{fontenc}
\usepackage{animate}
\usepackage{amsbsy}
\usepackage{amsfonts}
\usepackage{amsmath}
\usepackage{amssymb}
\usepackage{amsthm}
\setbeamertemplate{theorems}[numbered] % to number
\usepackage[toc,page,title,titletoc]{appendix}
\usepackage{dsfont}
\usepackage{esvect}
\usepackage[labelfont=bf]{caption}
%\usepackage{subcaption}
\usepackage{float}
\usepackage[Glenn]{fncychap}%Sonny %Conny %Lenny %Glenn %Renje %Bjarne %Bjornstrup
\usepackage{graphicx}
\usepackage{subfig}
\usepackage{indentfirst}%Para indentar os paragrafos automáticamente
\usepackage{lipsum}
\usepackage{longtable}
\usepackage{mathtools}
\usepackage{listings}%Inserir codigo do R no latex
\usepackage{multirow}
\usepackage{multicol}
\usepackage{csquotes}
\usepackage[citestyle=authoryear,maxcitenames=2,terseinits=true,natbib=true, style=abnt, maxbibnames=99]{biblatex}
\addbibresource{../Referencias/Referencias.bib}
\usepackage[figuresright]{rotating}
\usepackage{spalign}
\usepackage{pgfplots}
\pgfplotsset{compat=1.17}
\usepackage{tikz}
\usepackage{fontawesome}
\usepackage{color, colortbl}
\usepackage{url}
\usepackage{cancel}
\usepackage{accents}
\usepackage{bm}
\usepackage{ragged2e}%para justificar o texto dentro de algum ambiente
\definecolor{Gray}{gray}{0.9}
\definecolor{LightCyan}{rgb}{0.88,1,1}


\usepackage[all]{xy}
\usepackage{hyperref,bookmark}
\hypersetup{
  colorlinks=true,
  linkcolor=blue,
  citecolor=red,
  filecolor=blue,
  urlcolor=blue,
}

\usetheme{Madrid}
\usecolortheme[RGB={193,0,0}]{structure}

%\setbeamertemplate{footline}[frame number]
%\setbeamertemplate{footline}[text line]{%
%  \parbox{\linewidth}{\vspace*{-8pt}\hfill\date{}\hfill\insertshortauthor\hfill\insertpagenumber}}
\beamertemplatenavigationsymbolsempty
\renewcommand{\vec}[1]{\mbox{\boldmath$#1$}}
\newtheorem{Teorema}{Teorema}
\newtheorem{Proposicao}{Proposição}
\newtheorem{definicao}{Definição}
\newtheorem{Corolario}{Corolário}
\newtheorem{Demonstracao}{Demonstração}
\newcommand{\bx}{\ensuremath{\bar{x}}}
\newcommand{\Ho}{\ensuremath{H_{0}}}
\newcommand{\Hi}{\ensuremath{H_{1}}}
\newcommand{\at}[2][]{#1|_{#2}}
\newcommand\xuparrow[1][2ex]{%
   \mathrel{\rotatebox{-90}{$\xleftarrow{\rule{#1}{0pt}}$}}
}
\apptocmd{\frame}{}{\justifying}{} % Allow optional arguments after frame.

\makeatletter
\setbeamertemplate{footline}
{
  \leavevmode%
  \hbox{%
  \begin{beamercolorbox}[wd=.3\paperwidth,ht=2.25ex,dp=1ex,center]{author in head/foot}%
    \usebeamerfont{author in head/foot}\mytext
  \end{beamercolorbox}%
  \begin{beamercolorbox}[wd=.3\paperwidth,ht=2.25ex,dp=1ex,center]{title in head/foot}%
    \usebeamerfont{title in head/foot}\mytextt
  \end{beamercolorbox}%
  \begin{beamercolorbox}[wd=.35\paperwidth,ht=2.25ex,dp=1ex,right]{site in head/foot}%
    \usebeamerfont{site in head/foot}\mytexttt\hspace*{2em}
    \insertframenumber{} / \inserttotalframenumber\hspace*{2ex} 
  \end{beamercolorbox}}%
  \vskip0pt%
}
\makeatother

\providecommand{\arcsin}{} \renewcommand{\arcsin}{\hspace{2pt}\textrm{arcsen}}
\providecommand{\sin}{} \renewcommand{\sin}{\hspace{2pt}\textrm{sen}}
\newcommand{\N}{\rm I\!N}
\newcommand{\I}{\rm I\!I}
\newcommand{\R}{\rm I\!R}
\newcommand{\Sim}{\overset{\text{iid}}{\sim}}
\newcommand{\Lim}{{\displaystyle \lim_{n\to\infty}}}
\newcommand{\LimInf}{{\displaystyle \liminf_{n\to\infty}}}
\newcommand{\rightLim}{\xrightarrow[n\rightarrow\infty]{}}
\newcommand{\Sumi}{{\displaystyle \sum_{i=1}^{n}}}
\newcommand{\Int}{{\displaystyle \int_{-\infty}^{+\infty}}}
\newcommand{\ConvD}{\overset{D}{\rightarrow}}
\newcommand{\ConvP}{\overset{P}{\rightarrow}}
\newcommand{\Prodi}{{\displaystyle \prod_{i=1}^{n}}}
\newcommand{\SetaUP}[2]{\underset{\mathclap{\substack{\xuparrow[30pt] \\ #1}}}{#2}}
%\newcommand{\SetaInclinada}[2]{\underset{\mathclap{\substack{\rotatebox{135}{\xuparrow[30pt] \\ #1}}}}{#2}}
\newcommand{\Home}{\begin{tikzpicture}
\node[scale=2] at (3,4) {\text{Para}~\faHome};
\end{tikzpicture}}
\newcommand{\vecX}{\boldsymbol{X}}
\newcommand{\Implica}[1]{\xRightarrow{#1}}
\newcommand{\SeSe}{\iff}
\newcommand{\EscoreA}{\dfrac{\partial}{\partial\theta}\log{f(x,\theta)}}
\newcommand{\EscoreB}{\dfrac{\partial^{2}}{\partial\theta^{2}}\log{f(x,\theta)}}
\newcommand{\cqd}{\text{cqd}~\blacksquare}
\newcommand{\seqX}{$X_{1},\ldots,X_{n}$}
\newcommand{\seqY}{$Y_{1},\ldots,Y_{n}$}
\newcommand{\tend}[1]{\hbox{\oalign{$\bm{#1}$\crcr\hidewidth$\scriptscriptstyle\bm{\sim}$\hidewidth}}}

%\newtheorem{Teorema}{Teorema}
%\newtheorem{Proposicao}{Proposição}
%\newtheorem{Definicao}{Definição}
%\newtheorem{Corolario}{Corolário}
%\newtheorem{Demonstracao}{Demonstração}

\titlegraphic{\hspace*{8cm}\href{https://fsbmat-ufv.github.io/}{\includegraphics[width=2cm]{figs/mylogo.png}}
}

%Continuar a numeracao em slides diferentes
\newcounter{saveenumi}
\newcommand{\seti}{\setcounter{saveenumi}{\value{enumi}}}
\newcommand{\conti}{\setcounter{enumi}{\value{saveenumi}}}

\resetcounteronoverlays{saveenumi}

% Layout da pagina
\hypersetup{pdfpagelayout=SinglePage}

%Para o \pause funcionar dentro do ambiente align
\makeatletter
\let\save@measuring@true\measuring@true
\def\measuring@true{%
  \save@measuring@true
  \def\beamer@sortzero##1{\beamer@ifnextcharospec{\beamer@sortzeroread{##1}}{}}%
  \def\beamer@sortzeroread##1<##2>{}%
  \def\beamer@finalnospec{}%
}
\makeatother


\title{Inferência Estatística II}
\author{Prof. Fernando de Souza Bastos\texorpdfstring{\\ fernando.bastos@ufv.br}{}}
\institute{Departamento de Estatística\texorpdfstring{\\ Programa de Pós-Graduação em Estatística Aplicada e Biometria}\texorpdfstring{\\ Universidade Federal de Viçosa}{}\texorpdfstring{\\ Campus UFV - Viçosa}{}}
\date{}
\newcommand\mytext{Aula 8}
\newcommand\mytextt{Fernando de Souza Bastos}
\newcommand\mytexttt{\url{https://est711.github.io/}}


\begin{document}
%\SweaveOpts{concordance=TRUE}

\frame{\titlepage}

\begin{frame}{}
\frametitle{\bf Sumário}
\tableofcontents
\end{frame}

\section{Caso Multiparamétrico: Estimação}
\begin{frame}{Caso Multiparamétrico: Estimação}
\begin{block}{}
\justifying
Considere $\theta$ um vetor de $p$ parâmetros. Sejam $X_{1},\ldots,X_{n}$ variáveis aleatórias iid com densidade $f(x,\theta),~\theta\in\Omega\subset\R^{p}.$ A função log-verossimilhança é dada por
\begin{align*}
    \ell(\theta)=\log{L(\theta)}=\Sumi \log{f(x_{i},\theta)}
\end{align*}
A teoria para o caso multiparamétrico requer condições de regularidade adicionais. Considere o apêndice do livro texto \citet{hogg}, página 687.
\end{block}
\end{frame}

\begin{frame}{}
\begin{block}{Observação:}
\justifying
A prova do teorema abaixo não depende se $\theta$ é um escalar ou um vetor. Portanto, o resultado é válido também para o caso multiparamétrico.

\end{block}
\begin{Teorema}\label{Teo1}
\justifying
Seja $\theta_{0}$ o valor verdadeiro de $\theta.$ Sob $(R0)$ e $(R1),$ temos que
\begin{align*}
    \Lim P_{\theta_{0}}(L(\theta_{0},X)>L(\theta,X))=1,~\text{para}~\theta\neq \theta_{0},
\end{align*}
ou seja, $\theta_{0}$ é o ponto de máximo de $L(\theta,X).$
\end{Teorema}

%Seja X uma variável aleatória com função densidade de probabilidade $f(x; \theta)$, onde $\theta \in \Omega \subset \R^p$. Sob essas suposições, $X$ pode ser uma variável aleatória escalar ou um vetor aleatório em $\R^k$. Seja $I(\theta) = [I_{jk}]$ a matriz de informação $p \times p$ dada pela expressão (6.4.4). Além disso, vamos denotar o parâmetro verdadeiro por $\theta_0$.

\end{frame}

\begin{frame}{}
\begin{block}{}
\justifying
Com isso, usaremos como estimador de $\theta$ aquele valor que maximiza $L(\theta),$ ou equivalentemente, o valor que resolve o sistema de equações $\dfrac{\partial\ell(\theta)}{\partial \theta}=0.$ Se esse valor existir, será chamado de estimador de máxima verossimilhança (EMV).
\end{block}
\pause
\begin{block}{}
\justifying
Se o interesse for estimar uma função de $\theta,$ digamos $\eta=g(\theta)$ e se $\hat{\theta}$ é um EMV para $\theta,$ então $\hat{\eta}=g(\hat{\theta})$ é um estimador de máxima verossimilhança para $\eta.$
\end{block}
\end{frame}

\begin{frame}{Exemplo}
\begin{block}{}
\justifying
$X_{1},\ldots,X_{n}\Sim N(\mu,\sigma^{2}).$ Neste caso, $\theta=(\mu, \sigma).$
\begin{align*}
    f(x;\mu,\sigma)=\dfrac{1}{\sqrt{2\pi\sigma^{2}}}e^{-\dfrac{(x-\mu)^{2}}{2\sigma^{2}}}, \theta\in\Omega=\R\times(0,+\infty)~\text{e}~ x\in\R.
\end{align*}
\end{block}
\pause
\begin{block}{}
\justifying
A função de verossimilhança nesse caso é:
\begin{align*}
L(\theta)=\Big(\dfrac{1}{\sqrt{2\pi}\sigma}\Big)^{n}e^{-\dfrac{\Sumi(x_{i}-\mu)^{2}}{2\sigma^{2}}}
\end{align*}
\end{block}
\end{frame}

\begin{frame}{}
\begin{block}{}
\justifying
\begin{align*}
 \ell(\theta)=\log{L(\theta)}=-\frac{n}{2}\log{2\pi}-n\log{\sigma}-\frac{1}{2\sigma^{2}}\Sumi(x_{i}-\mu)^{2}
\end{align*}
fazendo $\dfrac{\partial\ell(\theta)}{\partial\mu}=0$ e $\dfrac{\partial\ell(\theta)}{\partial\sigma}=0,$ pode - se obter:
\begin{align*}
    \hat{\mu}=\Bar{X}~\text{e}~\hat{\sigma}=\sqrt{\dfrac{1}{n-1}\Sumi(x_{i}-\Bar{x})^{2}}.
\end{align*}
Analisando a matriz de segundas derivadas de $\ell,$ pode-se ver que $(\hat{\mu},\hat{\sigma})$ é o ponto de máximo global de $\ell(\mu,\sigma).$ Ou seja, $(\hat{\mu},\hat{\sigma})$ é o EMV de $(\mu,\sigma).$
\end{block}
\end{frame}

\begin{frame}{}
\begin{block}{}
\justifying
No caso univariado, a informação de Fisher é a variância da v.a. $\dfrac{\partial\log{f(x,\theta)}}{\partial\theta}.$ No caso multiparamétrico, a informação de Fisher é definida como a matriz de variância e covariância do vetor aleatório
\begin{align*}
\bigtriangledown\log{f(\Vec{X},\theta)}=\Big(\dfrac{\partial\log{f(x,\theta)}}{\partial\theta_{1}},\ldots,\dfrac{\partial\log{f(x,\theta)}}{\partial\theta_{n}}\Big)^\top
\end{align*}
A informação de Fisher é $I(\theta)=Cov(\bigtriangledown\log{f(\Vec{X},\theta)}),$ em que 
\vspace{1cm}

{\small
\begin{align*}
Cov(Z_{1},Z_{2})=\begin{bmatrix}
       Cov(Z_{1},Z_{1}) & Cov(Z_{1},Z_{2})\\[0.3em]
       Cov(Z_{2},Z_{1}) & Cov(Z_{2},Z_{2})
\end{bmatrix}=\begin{bmatrix}
       Var(Z_{1})      & Cov(Z_{1},Z_{2})\\[0.3em]
       Cov(Z_{2},Z_{1})& Var(Z_{2})
\end{bmatrix}
\end{align*}
}

\end{block}
\end{frame}

\begin{frame}{}
\begin{block}{}
\justifying
O elemento $(j,k)$ de $I(\theta)$ é dado por 
\begin{align*}
    I_{jk}(\theta)=Cov\Big(\dfrac{\partial\log{f(x,\theta)}}{\partial\theta_{j}},\dfrac{\partial\log{f(x,\theta)}}{\partial\theta_{k}}\Big)
\end{align*}
\end{block}
\pause
\begin{block}{}
\justifying
Sob as condições de regularidade, temos que
\begin{align*}
    1&=\Int f(x,\theta)dx\Rightarrow 0=\dfrac{\partial}{\partial\theta}\Int f(x,\theta)dx\\
    \Rightarrow 0&=\Int\dfrac{\partial f(x,\theta)}{\partial\theta} dx=\Int\dfrac{\partial \log{f(x,\theta)}}{\partial\theta}f(x,\theta)dx\\
    &=E\Big[\dfrac{\partial \log{f(x,\theta)}}{\partial\theta}\Big]=E\Big[\bigtriangledown\log{f(x,\theta)}\Big]
\end{align*}
\end{block}
\end{frame}

\begin{frame}{}
\begin{block}{}
\justifying
Tomando mais uma derivada com relação a $\theta,$ obtemos:
{\small
\begin{align*}
    0_{1\times p}^{\top}&=\dfrac{\partial}{\partial\theta^{\top}}\Int\dfrac{\partial \log{f(x,\theta)}}{\partial\theta}f(x,\theta)dx\\
    \\
    &=\int \dfrac{\partial^{2}\log{f(x,\theta)}}{\partial\theta\partial\theta^{\top}}f(x,\theta)dx+\int \dfrac{\partial \log{f(x,\theta)}}{\partial\theta}\dfrac{\partial \log{f(x,\theta)}}{\partial\theta^{\top}}f(x,\theta)dx\\
    \\
    \Rightarrow &- \int \dfrac{\partial^{2}\log{f(x,\theta)}}{\partial\theta\partial\theta^{\top}}f(x,\theta)dx=\int \dfrac{\partial \log{f(x,\theta)}}{\partial\theta}\dfrac{\partial \log{f(x,\theta)}}{\partial\theta^{\top}}f(x,\theta)dx\\
    \\
    \Rightarrow &\phantom{=} E\Big(\dfrac{\partial \log{f(x,\theta)}}{\partial\theta}\dfrac{\partial \log{f(x,\theta)}}{\partial\theta^{\top}} \Big)   =-E\Big(\dfrac{\partial^{2}\log{f(x,\theta)}}{\partial\theta\partial\theta^{\top}}\Big)
\end{align*}
}
\end{block}
\end{frame}

\begin{frame}{}
\begin{block}{}
\justifying
Daí, $$I(\theta)=-E\Big(\dfrac{\partial^{2}\log{f(x,\theta)}}{\partial\theta\partial\theta^{\top}}\Big).$$
\end{block}
\end{frame}

\begin{frame}{Exemplo}
\begin{block}{}
\justifying
$X\sim N(\mu,\sigma),~\theta=(\mu,\sigma).$
\begin{align*}
    \dfrac{\partial\log{f}}{\partial\mu}&=\frac{1}{\sigma^{2}}(x-\mu);~\dfrac{\partial^{2}\log{f}}{\partial\mu^{2}}=-\frac{1}{\sigma^{2}};~\dfrac{\partial\log{f}}{\partial\sigma}=-\frac{1}{\sigma}+\frac{1}{\sigma^{3}}(x-\mu)^{2}\\
    \dfrac{\partial^{2}\log{f}}{\partial\sigma^{2}}&=\frac{1}{\sigma^{2}}-\frac{3}{\sigma^{4}}(x-\mu)^{2};~\dfrac{\partial^{2}\log{f}}{\partial\mu\partial\sigma}=-\frac{2}{\sigma^{3}}(x-\mu)
\end{align*}
Logo,
{\small
\begin{align*}
I(\theta)=I(\mu,\sigma)=\begin{bmatrix}
       \frac{1}{\sigma^{2}} & 0\\[0.3em]
       0 & \frac{2}{\sigma^{2}}
\end{bmatrix}=\begin{bmatrix}
       -E\Big(\dfrac{\partial^{2}\log{f(x,\theta)}}{\partial\mu^{2}}\Big)      & -E\Big(\dfrac{\partial^{2}\log{f(x,\theta)}}{\partial\sigma\partial\mu}\Big)\\[0.3em]
       E\Big(\dfrac{\partial^{2}\log{f(x,\theta)}}{\partial\mu\partial\sigma}\Big)& -E\Big(\dfrac{\partial^{2}\log{f(x,\theta)}}{\partial\sigma^{2}}\Big)
\end{bmatrix}
\end{align*}
}
\end{block}
\end{frame}

\begin{frame}{}
\begin{Teorema}{}
\justifying
Sejam $X_{1},\ldots,X_{n}$ variáveis aleatórias iid com densidade $f(x,\theta),\theta\in\Omega\subset\R^{P}.$ Assuma que valem as condições de regularidade. Então,
\begin{enumerate}
    \item O sistema de equações $\dfrac{\partial\ell(\theta)}{\partial\theta}=0$ tem uma solução $\hat{\theta}_{n}$ tal que $\hat{\theta}_{n}\ConvP \theta_{0}$ ($\theta_{0}$ verdadeiro valor de $\theta$).
    \item Para qualquer sequência satisfazendo 1, temos $$\sqrt{n}(\hat{\theta}_{n}-\theta_{0})\ConvD N_{p}(0,I^{-1}(\theta)).$$
\end{enumerate}
\end{Teorema}
\end{frame}

\begin{frame}{}
\begin{block}{Corolário:}
\justifying
Sob as condições do teorema anterior, temos que $$\sqrt{n}(\hat{\theta}_{in}-\theta_{i0})\ConvD N_{p}(0,I_{ii}^{-1}(\theta)).$$
Seja $g(\theta)=(g_{1}(\theta),\ldots,g_{k}(\theta))^{\top},1\leq k\leq p$ tal que a matriz de derivadas parciais $B=\Big(\dfrac{\partial g_{i}}{\partial \theta_{j}}\Big)_{ij}, i=1,\ldots,k$ e $j=1,\ldots,p$ possuam elementos contínuos e que não se anulem numa vizinhança de $\theta_{0}.$ Seja $\hat{\eta}=g(\hat{\theta}).$ Então, $\hat{\eta}$ é o EMV de $\eta$ e $$\sqrt{n}(\hat{\eta}-\eta_{0})\ConvD N_{k}(0,BI^{-1}(\theta)B^{\top}).$$
\end{block}
\end{frame}


\begin{frame}{}
\begin{block}{\Home}
\justifying
\begin{itemize}
    \item \textbf{Exercícios da seção 6.4:} Todos, exceto o 5 e o 8.
\end{itemize}
\nocite{hogg, casella2021statistical, bolfarine}
\end{block}
\end{frame}

\begin{frame}[allowframebreaks]
\frametitle{\bf Referências}
\printbibliography
\end{frame}


\end{document}
