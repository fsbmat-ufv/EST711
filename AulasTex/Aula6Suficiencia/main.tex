\documentclass[12pt]{beamer}

\usepackage[brazil]{babel}
\usepackage[utf8]{inputenc}
\usepackage[T1]{fontenc}
\usepackage{animate}
\usepackage{amsbsy}
\usepackage{amsfonts}
\usepackage{amsmath}
\usepackage{amssymb}
\usepackage{amsthm}
\setbeamertemplate{theorems}[numbered] % to number
\usepackage[toc,page,title,titletoc]{appendix}
\usepackage{dsfont}
\usepackage{esvect}
\usepackage[labelfont=bf]{caption}
%\usepackage{subcaption}
\usepackage{float}
\usepackage[Glenn]{fncychap}%Sonny %Conny %Lenny %Glenn %Renje %Bjarne %Bjornstrup
\usepackage{graphicx}
\usepackage{subfig}
\usepackage{indentfirst}%Para indentar os paragrafos automáticamente
\usepackage{lipsum}
\usepackage{longtable}
\usepackage{mathtools}
\usepackage{listings}%Inserir codigo do R no latex
\usepackage{multirow}
\usepackage{multicol}
\usepackage{csquotes}
\usepackage[citestyle=authoryear,maxcitenames=2,terseinits=true,natbib=true, style=abnt, maxbibnames=99]{biblatex}
\addbibresource{../Referencias/Referencias.bib}
\usepackage[figuresright]{rotating}
\usepackage{spalign}
\usepackage{pgfplots}
\pgfplotsset{compat=1.17}
\usepackage{tikz}
\usepackage{fontawesome}
\usepackage{color, colortbl}
\usepackage{url}
\usepackage{cancel}
\usepackage{accents}
\usepackage{bm}
\usepackage{ragged2e}%para justificar o texto dentro de algum ambiente
\definecolor{Gray}{gray}{0.9}
\definecolor{LightCyan}{rgb}{0.88,1,1}


\usepackage[all]{xy}
\usepackage{hyperref,bookmark}
\hypersetup{
  colorlinks=true,
  linkcolor=blue,
  citecolor=red,
  filecolor=blue,
  urlcolor=blue,
}

\usetheme{Madrid}
\usecolortheme[RGB={193,0,0}]{structure}

%\setbeamertemplate{footline}[frame number]
%\setbeamertemplate{footline}[text line]{%
%  \parbox{\linewidth}{\vspace*{-8pt}\hfill\date{}\hfill\insertshortauthor\hfill\insertpagenumber}}
\beamertemplatenavigationsymbolsempty
\renewcommand{\vec}[1]{\mbox{\boldmath$#1$}}
\newtheorem{Teorema}{Teorema}
\newtheorem{Proposicao}{Proposição}
\newtheorem{definicao}{Definição}
\newtheorem{Corolario}{Corolário}
\newtheorem{Demonstracao}{Demonstração}
\newcommand{\bx}{\ensuremath{\bar{x}}}
\newcommand{\Ho}{\ensuremath{H_{0}}}
\newcommand{\Hi}{\ensuremath{H_{1}}}
\newcommand{\at}[2][]{#1|_{#2}}
\newcommand\xuparrow[1][2ex]{%
   \mathrel{\rotatebox{-90}{$\xleftarrow{\rule{#1}{0pt}}$}}
}
\apptocmd{\frame}{}{\justifying}{} % Allow optional arguments after frame.

\makeatletter
\setbeamertemplate{footline}
{
  \leavevmode%
  \hbox{%
  \begin{beamercolorbox}[wd=.3\paperwidth,ht=2.25ex,dp=1ex,center]{author in head/foot}%
    \usebeamerfont{author in head/foot}\mytext
  \end{beamercolorbox}%
  \begin{beamercolorbox}[wd=.3\paperwidth,ht=2.25ex,dp=1ex,center]{title in head/foot}%
    \usebeamerfont{title in head/foot}\mytextt
  \end{beamercolorbox}%
  \begin{beamercolorbox}[wd=.35\paperwidth,ht=2.25ex,dp=1ex,right]{site in head/foot}%
    \usebeamerfont{site in head/foot}\mytexttt\hspace*{2em}
    \insertframenumber{} / \inserttotalframenumber\hspace*{2ex} 
  \end{beamercolorbox}}%
  \vskip0pt%
}
\makeatother

\providecommand{\arcsin}{} \renewcommand{\arcsin}{\hspace{2pt}\textrm{arcsen}}
\providecommand{\sin}{} \renewcommand{\sin}{\hspace{2pt}\textrm{sen}}
\newcommand{\N}{\rm I\!N}
\newcommand{\I}{\rm I\!I}
\newcommand{\R}{\rm I\!R}
\newcommand{\Sim}{\overset{\text{iid}}{\sim}}
\newcommand{\Lim}{{\displaystyle \lim_{n\to\infty}}}
\newcommand{\LimInf}{{\displaystyle \liminf_{n\to\infty}}}
\newcommand{\rightLim}{\xrightarrow[n\rightarrow\infty]{}}
\newcommand{\Sumi}{{\displaystyle \sum_{i=1}^{n}}}
\newcommand{\Int}{{\displaystyle \int_{-\infty}^{+\infty}}}
\newcommand{\ConvD}{\overset{D}{\rightarrow}}
\newcommand{\ConvP}{\overset{P}{\rightarrow}}
\newcommand{\Prodi}{{\displaystyle \prod_{i=1}^{n}}}
\newcommand{\SetaUP}[2]{\underset{\mathclap{\substack{\xuparrow[30pt] \\ #1}}}{#2}}
%\newcommand{\SetaInclinada}[2]{\underset{\mathclap{\substack{\rotatebox{135}{\xuparrow[30pt] \\ #1}}}}{#2}}
\newcommand{\Home}{\begin{tikzpicture}
\node[scale=2] at (3,4) {\text{Para}~\faHome};
\end{tikzpicture}}
\newcommand{\vecX}{\boldsymbol{X}}
\newcommand{\Implica}[1]{\xRightarrow{#1}}
\newcommand{\SeSe}{\iff}
\newcommand{\EscoreA}{\dfrac{\partial}{\partial\theta}\log{f(x,\theta)}}
\newcommand{\EscoreB}{\dfrac{\partial^{2}}{\partial\theta^{2}}\log{f(x,\theta)}}
\newcommand{\cqd}{\text{cqd}~\blacksquare}
\newcommand{\seqX}{$X_{1},\ldots,X_{n}$}
\newcommand{\seqY}{$Y_{1},\ldots,Y_{n}$}
\newcommand{\tend}[1]{\hbox{\oalign{$\bm{#1}$\crcr\hidewidth$\scriptscriptstyle\bm{\sim}$\hidewidth}}}

%\newtheorem{Teorema}{Teorema}
%\newtheorem{Proposicao}{Proposição}
%\newtheorem{Definicao}{Definição}
%\newtheorem{Corolario}{Corolário}
%\newtheorem{Demonstracao}{Demonstração}

\titlegraphic{\hspace*{8cm}\href{https://fsbmat-ufv.github.io/}{\includegraphics[width=2cm]{figs/mylogo.png}}
}

%Continuar a numeracao em slides diferentes
\newcounter{saveenumi}
\newcommand{\seti}{\setcounter{saveenumi}{\value{enumi}}}
\newcommand{\conti}{\setcounter{enumi}{\value{saveenumi}}}

\resetcounteronoverlays{saveenumi}

% Layout da pagina
\hypersetup{pdfpagelayout=SinglePage}

%Para o \pause funcionar dentro do ambiente align
\makeatletter
\let\save@measuring@true\measuring@true
\def\measuring@true{%
  \save@measuring@true
  \def\beamer@sortzero##1{\beamer@ifnextcharospec{\beamer@sortzeroread{##1}}{}}%
  \def\beamer@sortzeroread##1<##2>{}%
  \def\beamer@finalnospec{}%
}
\makeatother


\title{Inferência Estatística II}
\author{Prof. Fernando de Souza Bastos\texorpdfstring{\\ fernando.bastos@ufv.br}{}}
\institute{Departamento de Estatística\texorpdfstring{\\ Programa de Pós-Graduação em Estatística Aplicada e Biometria}\texorpdfstring{\\ Universidade Federal de Viçosa}{}\texorpdfstring{\\ Campus UFV - Viçosa}{}}
\date{}
\newcommand\mytext{Aula 6}
\newcommand\mytextt{Fernando de Souza Bastos}
\newcommand\mytexttt{\url{https://est711.github.io/}}


\begin{document}
%\SweaveOpts{concordance=TRUE}

\frame{\titlepage}

\begin{frame}{}
\frametitle{\bf Sumário}
\tableofcontents
\end{frame}
\section{Medidas da Qualidade de um Estimador}
\subsection{Estimador ENVVUM}
\begin{frame}{}
\begin{block}{}
\justifying
Suponha que $f(x; \theta),$ $\theta \in \Omega,$ seja uma função densidade ou função de probabilidade. Considere $Y_n = u(X_1, \ldots, X_n)$ baseado em uma amostra aleatória $X_1, \ldots, X_n,$ com densidade ou função de probabilidade $f(x; \theta).$
\end{block}
\pause
\begin{block}{Definição:}
\justifying
Para um dado inteiro positivo $n$, $Y = u(X_1,X_2, \ldots, X_n)$ é um estimador não viesado de variância uniformemente mínima(ENVVUM) para $\theta$ se $Y$ for não viesado, ou seja, $\mathbb{E}(Y) = \theta$, e se a variância de $Y$ for menor ou igual à variância de qualquer outro estimador não viesado de $\theta$.
\end{block}
\end{frame}

\begin{frame}{}
\begin{block}{Observação Importante:}
\justifying
Vamos agora discutir o problema da estimação pontual de um parâmetro a partir de uma perspectiva ligeiramente diferente.
\end{block}
\end{frame}

\subsection{Função de Decisão}
\begin{frame}{Função de Decisão}
\begin{block}{}
\justifying
Seja $Y = u(X_1,X_2, \ldots, X_n)$ uma estatística com valor observado $y = u(x_1,x_2, \ldots, x_n)$. Seja $\delta(y)$ uma função da estatística observada $y,$ uma estimativa pontual para $\theta$. Assim, a função $\delta(y)$ decide o valor de $\theta.$  $\delta$ é chamada de função de decisão ou regra de decisão.
\end{block}
\pause
\begin{block}{}
\justifying
Um valor da função de decisão, digamos $\delta(y)$, é chamado de decisão. Assim, uma estimativa pontual numericamente determinada de um parâmetro $\theta$ é uma decisão. Uma decisão pode estar correta ou pode estar errada. Seria útil ter uma medida da diferença, se houver, entre o valor verdadeiro de $\theta$ e a estimativa pontual $\delta(y)$.
\end{block}
\end{frame}

\begin{frame}{}
\begin{block}{}
\justifying
Associamos a cada par $(\theta, \delta(y))$, $\theta \in \Omega$, um número não negativo $L(\theta, \delta(y))$ que reflete o quanto $\delta(y)$ está afastado de $\theta.$ 
\begin{itemize}
    \item Chamamos a função $L$ de função perda. 
\end{itemize}
\end{block}
\pause
\begin{block}{}
Para a variável aleatória contínua $Y,$ podemos calcular a função Risco:
$$R(\theta, \delta) =\mathbb{E}\Big({L(\theta, \delta(y))}\Big),$$ 
em que,
\begin{itemize}
    \item $\mathbb{E}\Big({L(\theta, \delta(y))}\Big) = \int_{-\infty}^{\infty} L(\theta, \delta(y)) \cdot f_Y(y)dy$
\end{itemize}
\end{block}
\end{frame}

\begin{frame}{Exemplo}
\begin{block}{}
\justifying
$X_1, X_2, \ldots, X_{25}\overset{\text{iid}}{\sim} N(\theta, 1), ~ \theta \in \R$. Seja $Y = \bar{X}\overset{\text{iid}}{\sim} N(\theta, \dfrac{1}{25}).$ Considere, $\delta_1(y) = y, ~\delta_2(y) = 0$ e tome $L(\theta, \delta(y)) = (\theta - \delta(y))^2.$ As funções de risco correspondentes são:
\begin{itemize}
    \item $R(\theta, \delta_1) = E[(\theta - Y)^2] = Var(Y) = \frac{1}{25};$
    \item $R(\theta, \delta_2) = E[(\theta - 0)^2] = \theta^2$.
\end{itemize}
\end{block}
\pause
\begin{block}{}
\justifying
Se nosso critério for selecionar o estimador com menor risco, temos:
$R(\theta, \delta_2)\leq R(\theta, \delta_1) \iff \theta^2\leq \dfrac{1}{25}\iff -\dfrac{1}{5}\leq \theta \leq \dfrac{1}{5}.$
\end{block}
\pause
\begin{block}{}
\justifying
Neste caso, se $-\dfrac{1}{5}\leq \theta \leq \dfrac{1}{5}$ o estimador $\delta_{2}$ será melhor que o estimador $\delta_{1}.$ Caso contrário, ou seja, para $\theta \not \in [-\dfrac{1}{5},\dfrac{1}{5}],~\delta_{1} $ será melhor que $\delta_{2}.$
\end{block}
\end{frame}

\begin{frame}{}
\begin{block}{}
\justifying
Agora, vamos nos restringir a classe de estimadores não viesados $(\mathbb{E}[\delta(Y)] = \theta).$ Utilizando a perda quadrática $L(\theta, \delta(y))=(\theta-\delta(y)^{2}),$ escolher um estimador não viesado para $\theta$ que tem o menor risco dentre todos os estimadores não viesados para $\theta.$ Nos deparamos com o problema de encontrar o ENVVUM (estimador não tendencioso de variância uniformemente mínima).
\end{block}
\end{frame}

\begin{frame}{}
\begin{block}{}
\justifying
Poderíamos também adotar outro critério. Em vez de nos restringir a classe dos estimadores não viesados, poderíamos escolher o estimador que minimiza o máximo risco $(\max_\theta R(\theta, \delta)),$ conhecido como estimador Minimax. Com esse critério, nosso exemplo, $\max_\theta R(\theta, \delta_1) = \max_\theta \left(\frac{1}{25}\right) = \frac{1}{25}$ e $\max_\theta R(\theta, \delta_2) = +\infty$ e, portanto, o melhor estimador seria o estimador $\delta_{1}.$
\end{block}
\pause
\begin{block}{As funções perda mais utilizadas:}
\justifying
\begin{itemize}
    \item $L(\theta, \delta(y))=(\theta-\delta(y))^{2},$ conhecida como perda quadrática.
    \item $L(\theta, \delta(y))=|\theta-\delta(y)|,$ conhecida como erro absoluto.
\end{itemize}
\end{block}
\end{frame}

\begin{frame}{}
\begin{block}{Neste exemplo, ilustramos o seguinte:}
\justifying
\begin{enumerate}
    \item Sem alguma restrição sob a função de decisão, é difícil encontrar uma função de decisão que tem risco uniformemente menor que outras funções de decisão.\pause
    \item Um princípio de seleção da melhor função de decisão é chamado de \textbf{princípio minimax}. Esse princípio diz que: 
\end{enumerate}
\end{block}

\begin{block}{}
\justifying
Se a função de decisão dada por $\delta_{0}(y)$ é tal que, para todo $\theta \in \Omega$,
$\max_\theta R[\theta, \delta_0(y)] \leq \max_\theta R[\theta, \delta(y)]$
para qualquer outra função de decisão $\delta(y)$, então $\delta_0(y)$ é chamada de função de decisão minimax.
\end{block}
\end{frame}

\begin{frame}{Exercícios}
\begin{block}{\Home}
\justifying
Exercícios 7.1: 1,2,4 ao 9
\end{block}
\end{frame}

\section{Estatística Suficiente para um Parâmetro}
\begin{frame}{Estatística Suficiente para um Parâmetro}
\begin{definicao}\label{def1}
\justifying
    Sejam $X_{1},X_{2}, \ldots , X_{n}$ uma amostra aleatória com densidade ou função de probabilidade $f(x; \theta)$, $\theta \in \Omega$. Seja $Y_{1} = u_{1}(X_{1},X_{2}, \ldots , X_{n})$ uma estatística cuja função de densidade ou função de probabilidade é $f_{Y_1}(y_1; \theta)$. Então, $Y_1$ é uma estatística suficiente para $\theta$ se, e somente se,
\begin{equation}
\dfrac{{f(x_1; \theta)f(x_2; \theta) \cdots f(x_n; \theta)}}{{f_{Y_1}(y_{1}; \theta)}} = H(x_1, x_2, \ldots , x_n),
\end{equation}

em que $H(x_1, x_2, \ldots , x_n)$ não depende de $\theta \in \Omega$.
\end{definicao}
\end{frame}

\begin{frame}{Exemplo}
\begin{block}{}
\justifying
$X_1,X_2, \ldots,X_n\Sim \Gamma(2,\theta),~\theta>0.$ Seja $y={\displaystyle \sum_{i=1}^{n}X_{i}}.$ Mostre que $Y\Sim \Gamma(2n,\theta)$ usando função geradora de momentos.
\end{block}
\pause
\begin{block}{}
\justifying
Como a função geradora de momentos associada a essa distribuição é dada por $M(t) = (1 - \theta t)^{-2}, t < \frac{1}{\theta}$, a função geradora de momentos de $Y= X_1 + X_2 + \ldots + X_n$ é:
\begin{align*}
E[e^{t(X_1+X_2+\ldots+X_n)}] &= E[e^{tX_1}e^{tX_2} \ldots e^{tX_n}] \\
&= [(1-\theta t)^{-2}]^n \\
&= (1 - \theta t)^{-2n}.
\end{align*}

Logo, $Y\Sim \Gamma(2n,\theta).$ 
\end{block}
\end{frame}

\begin{frame}{}
\begin{block}{}
\justifying
Notem que, as funções de densidade $f_{X_{1}}(x_{1};\theta), f_{Y}(y;\theta)$ são definidas como:

\begin{align*}
f_{X_{1}}(x_{1};\theta) &= 
\frac{1}{\Gamma(2)\theta^{2}}x_1^{2-1}e^{-\frac{x}{\theta}},~x_{1}>0.\\
f_{Y}(y;\theta) &= 
\frac{1}{\Gamma(2n)\theta^{2n}}y^{2n-1}e^{-\frac{y}{\theta}}, ~y={\displaystyle \sum_{i=1}^{n}X_{i}}.
\end{align*}
\end{block}
\end{frame}

\begin{frame}{}
\begin{block}{}
\justifying
Segue que,
\begin{equation*}
\dfrac{\frac{x_1^{2-1}e^{-x_1/\theta}}{\Gamma(2)\theta^2} \cdot \frac{x_2^{2-1}e^{-x_2/\theta}}{\Gamma(2)\theta^2} \cdot \ldots \cdot \frac{x_n^{2-1}e^{-x_n/\theta}}{\Gamma(2)\theta^2}}{\frac{(x_1 + x_2 + \ldots + x_n)^{2n-1}e^{-(x_1+x_2+\ldots+x_n)/\theta}}{\Gamma(2n)\theta^{2n}}}=\frac{\Gamma(2n)}{(\Gamma(2))^n} \cdot \frac{x_1x_2\cdot\ldots\cdot x_n}{(x_1 + x_2 + \ldots + x_n)^{2n-1}},
\end{equation*}
Como, a expressão final, não depende de $\theta,~Y$ é uma estatística suficiente para $\theta.$ 
\end{block}
\end{frame}

\subsection{Teorema de Neyman (Critério de Fatoração)}
\begin{frame}{Teorema de Neyman (Critério de Fatoração)}
\begin{block}{}
\begin{Teorema}
\justifying
    Seja $X_1, X_2, \ldots, X_n$ uma amostra aleatória com densidade de probabilidade ou função de probabilidade $f(x; \theta), ~\theta \in \Omega$. A estatística $Y_1 = u(X_1, \ldots, X_n)$ é uma estatística suficiente para $\theta$ se, e somente se, existir duas funções não negativas, $k_1$ e $k_2$, tais que

\begin{equation*}
f(x_1; \theta)f(x_2; \theta) \ldots f(x_n; \theta) = k_1[u(x_1, x_2, \ldots, x_n); \theta]k_2(x_1, x_2, \ldots, x_n)
\end{equation*}

em que $k_2(x_1, x_2, \ldots, x_n)$ não depende de $\theta$.
\end{Teorema}
\end{block}
\end{frame}

\begin{frame}{Observação (Jacobiano):}
	\begin{block}{}
		\justifying
Seja $(X_{1},X_{2})$ um par aleatório absolutamente contínuo com densidade de probabilidade conjunta $f_{X_{1},X_{2}}(x_{1},x_{2})$. Seja também $U: \mathbb{R}^2 \rightarrow \mathbb{R}^2$ uma função injetiva (portanto com inversa) com dois componentes $U(x_{1},x_{2}) = (y_{1},y_{2})$. Cada um destes componentes é função de duas variáveis reais, tal que:

\[
\begin{cases}
	y_{1} = u_1(x_{1},x_{2}) \\
	y_{2} = u_2(x_{1},x_{2})
\end{cases}
\]

sendo que $u_1$ e $u_2$ possuem derivadas parciais em relação a $x_{1}$ e $x_{2}$.

Portanto, podemos definir o par aleatório $(Y_{1},Y_{2}) = U(X_{1},X_{2})$. Como determinar a densidade de probabilidade conjunta do par $(Y_{1},Y_{2})$ a partir da densidade conjunta de $(X_{1},X_{2})$?
	\end{block}
\end{frame}

\begin{frame}{Observação (Jacobiano):}
	\begin{block}{}
		\justifying
		Como $U$ tem inversa, podemos escrever:
		
		\[
		\begin{cases}
			x_{1} = w_1(y_{1},y_{2}) \\
			x_{2} = w_2(y_{1},y_{2})
		\end{cases}
		\]
A densidade conjunta de $(Y_{1},Y_{2})$ será:

\[
f_{Y_{1},Y_{2}}(y_{1},y_{2}) = \left| J \right| f_{X_{1},X_{2}}\left(w_1(y_{1},y_{2}), w_2(y_{1},y_{2})\right)
\]

em que $\left| J \right|$ representa o módulo do determinante jacobiano, isto é, o módulo de:

\[
\left| 
\begin{vmatrix}
	\frac{\partial w_1(y_{1},y_{2})}{\partial y_{1}} & \frac{\partial w_1(y_{1},y_{2})}{\partial y_{2}} \\
	\frac{\partial w_2(y_{1},y_{2})}{\partial y_{1}} & \frac{\partial w_2(y_{1},y_{2})}{\partial y_{2}}
\end{vmatrix}
\right|.
\]		
	\end{block}
\end{frame}

\begin{frame}{Demonstração}
	\begin{block}{Considere o caso contínuo!}
		\justifying
		Faremos a ``volta'' do teorema primeiro $(\Leftarrow{})$. Por hipótese, existem duas funções não negativas, $k_1$ e $k_2$, tais que
		
		\begin{equation*}
			f(x_1; \theta)f(x_2; \theta) \ldots f(x_n; \theta) = k_1[u_1(x_1, x_2, \ldots, x_n); \theta]k_2(x_1, x_2, \ldots, x_n),
		\end{equation*}
		ou seja, 
		\begin{align*}
			\dfrac{f(x_1; \theta)f(x_2; \theta) \ldots f(x_n; \theta)}{f_{Y_{1}}(y_{1},\theta)}=\dfrac{k_1[u_1(x_1, x_2, \ldots, x_n); \theta]k_2(x_1, x_2, \ldots, x_n)}{f_{Y_{1}}(y_{1},\theta)}
		\end{align*}
	\end{block}
	\pause
	\begin{block}{}
		Precisamos encontrar $f_{Y_{1}}(y_{1},\theta).$
	\end{block}
\end{frame}

\begin{frame}{Demonstração}
\begin{block}{Considere o caso contínuo!}
\justifying
Considere $$y_1 = u_1(x_1, x_2, \ldots, x_n), \ldots, y_n = u_n(x_1, x_2, \ldots, x_n)$$ tendo as funções inversas, $$x_1 = w_1(y_1, y_2, \ldots, y_n), \ldots, x_n = w_n(y_1, y_2, \ldots, y_n)$$ com Jacobiano $J$; Sabemos que, 
\begin{align*}
    f_{X_{1}, X_{2}, \ldots, X_{n}}(x_{1}, x_{2}, \ldots, x_{n})&=f_{X_{1}}(x_{1})f_{X_{2}}(x_{2})\ldots f_{X_{n}}(x_{n}).
\end{align*}
Para simplificar a notação considere $w_{i}=w_{i}(y_1, y_2, \ldots, y_n),\forall i=1,2,\cdots,n.$ 

\end{block}
\end{frame}

\begin{frame}{}
\begin{block}{}
\justifying
Com isso, 
\begin{align*}
	f_{Y_{1}, Y_{2}, \ldots, Y_{n}}(y_{1}, y_{2}, \ldots, y_{n})&=f_{X_{1}}(w_{1})f_{X_{2}}(w_{2})\ldots f_{X_{n}}(w_{n})|J|.
\end{align*}
Logo, por hipótese, existem duas funções não negativas, $k_1$ e $k_2$, tais que
\begin{align*}
    f_{Y_{1}, Y_{2}, \ldots, Y_{n}}(y_{1}, y_{2}, \ldots, y_{n})&=|J|k_1[y_{1}; \theta]k_2(w_1, w_2, \ldots, w_n)
\end{align*}
Daí,
\begin{align*}
    f_{Y_{1}}(y_{1})&=\int_{-\infty}^{\infty}\cdots \int_{-\infty}^{\infty} k_1[y_{1}; \theta]k_2(w_1, w_2, \ldots, w_n)|J|dy_{2}dy_{3}\cdots dy_{n}\\
    &=k_{1}(y_{1},\theta)\underbrace{\int_{-\infty}^{\infty}\cdots \int_{-\infty}^{\infty} k_2(w_1, w_2, \ldots, w_n)|J|dy_{2}dy_{3}\cdots dy_{n}}_{\text{Não depende de $\theta$, somente de $y_{1}$(=$m(y_{1})$)}}
\end{align*}
\end{block}
\pause
\begin{block}{}
\justifying
Então, $f_{Y_{1}}(y_{1})=k_{1}(y_{1},\theta)m(y_{1})$
\end{block}
\end{frame}

\begin{frame}{}
\begin{block}{}
\justifying
Com isso, 
\begin{align*}
    \dfrac{f(x_{1}; \theta) \cdot f(x_{2}; \theta) \ldots f(x_{n}; \theta)}{f_{Y_{1}}(y_{1},\theta)}&=\dfrac{\cancel{k_{1}(y_{1},\theta)}k_{2}(x_1, x_2, \ldots, x_n)}{\cancel{k_{1}(y_{1},\theta)}m(y_{1})}
\end{align*}
que não depende de $\theta,$ portanto, $Y$ é suficiente para $\theta.~\blacksquare$
\end{block}
\end{frame}

\begin{frame}{}
\begin{block}{$(\Rightarrow{})$}
\justifying
Por suposição,
\begin{align*}
    f(x_{1}; \theta) \cdot f(x_{2}; \theta) \ldots f(x_{n}; \theta)=\underbrace{H(x_{1},\ldots, x_{n})}_{k_{2}}\underbrace{f_{Y_{1}}(y_{1},\theta)}_{k_{1}}.~\blacksquare
\end{align*}
\end{block}
\end{frame}

\begin{frame}{Exemplo}
\begin{block}{}
\justifying
Seja $X_1, X_2, \ldots, X_n$ amostra aleatória com densidade 
$f(x; \theta) =\theta x^{\theta-1},~x\in (0,1)$ e $\theta>0.$

\begin{align*}
f(x_{1}; \theta) f(x_{2}; \theta) \ldots f(x_{n}; \theta)&=\theta^{n}x_{1}^{\theta-1} \ldots x_{n}^{\theta-1}\\
&=\theta^{n}\left(\displaystyle{\prod_{i=1}^{n}} x_i\right)^{\theta}\dfrac{1}{\displaystyle{\prod_{i=1}^{n}} x_i}\\
&=k_{1}(\theta,\displaystyle{\prod_{i=1}^{n}} x_i)k_{2}(x_{1},\ldots, x_{n})
\end{align*}

Uma vez que $k_2(x_1, x_2, \ldots, x_n)$ não depende de $\theta$, o produto $\prod_{i=1}^n X_i,$ pelo critério da fatoração, é uma estatística suficiente para $\theta$.
\end{block}
\end{frame}

\begin{frame}{Vantagens de uma Estatística Suficiente}
	\begin{block}{Saber que uma estatística é suficiente oferece as seguintes vantagens:}
		\justifying
	\begin{itemize}
		\item Reduz a quantidade de dados que precisamos manipular, sem perda de informação.\pause
		\item Melhora a eficiência dos estimadores, ajudando a encontrar estimadores de menor variância.\pause
		\item Facilita a fatoração da função de verossimilhança, simplificando cálculos.\pause
		\item Auxilia na construção de testes e intervalos de confiança ótimos.\pause
		\item Garante a melhor inferência possível em termos de uso da informação dos dados.
	\end{itemize}
	\end{block}
\end{frame}

\begin{frame}{Exercícios}
\begin{block}{\Home}
\justifying
Exercícios 7.2: 1,2,4 ao 8
\nocite{hogg, casella2021statistical, bolfarine}
\end{block}
\end{frame}

\begin{frame}[allowframebreaks]
\frametitle{\bf Referências}
\printbibliography
\end{frame}


\end{document}

\begin{frame}{}
\begin{block}{}
\justifying

\end{block}
\end{frame}

\begin{frame}{}
\begin{block}{}
\justifying

\end{block}
\end{frame}

\begin{frame}{}
\begin{block}{}
\justifying

\end{block}
\end{frame}

\begin{frame}{}
\begin{block}{}
\justifying

\end{block}
\end{frame}

\section{Estatística Suficiente e Minimal}
\begin{frame}{}
\begin{block}{}
\justifying

\end{block}
\end{frame}

\begin{frame}{}
\begin{block}{}
\justifying

\end{block}
\end{frame}

\begin{frame}{}
\begin{block}{}
\justifying

\end{block}
\end{frame}

\begin{frame}{}
\begin{block}{}
\justifying

\end{block}
\end{frame}

\begin{frame}{}
\begin{block}{}
\justifying

\end{block}
\end{frame}

\begin{frame}{}
\begin{block}{}
\justifying

\end{block}
\end{frame}

\begin{frame}{}
\begin{block}{}
\justifying

\end{block}
\end{frame}

\section{Familia Exponencial}
\begin{frame}{}
\begin{block}{}
\justifying

\end{block}
\end{frame}

\begin{frame}{}
\begin{block}{}
\justifying

\end{block}
\end{frame}

\begin{frame}{}
\begin{block}{}
\justifying

\end{block}
\end{frame}

\begin{frame}{}
\begin{block}{}
\justifying

\end{block}
\end{frame}

\begin{frame}{}
\begin{block}{}
\justifying

\end{block}
\end{frame}

\begin{frame}{}
\begin{block}{}
\justifying

\end{block}
\end{frame}

\begin{frame}{}
\begin{block}{}
\justifying

\end{block}
\end{frame}