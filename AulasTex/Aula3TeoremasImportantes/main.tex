
\documentclass[12pt]{beamer}

%\usepackage[brazil]{babel}
\usepackage[utf8]{inputenc}
\usepackage[T1]{fontenc}
\usepackage{animate}
\usepackage{amsbsy}
\usepackage{amsfonts}
\usepackage{amsmath}
\usepackage{amssymb}
\usepackage{amsthm}
\setbeamertemplate{theorems}[numbered] % to number
\usepackage[toc,page,title,titletoc]{appendix}
\usepackage{dsfont}
\usepackage{esvect}
\usepackage[labelfont=bf]{caption}
%\usepackage{subcaption}
\usepackage{float}
\usepackage[Glenn]{fncychap}%Sonny %Conny %Lenny %Glenn %Renje %Bjarne %Bjornstrup
\usepackage{graphicx}
\usepackage{subfig}
\usepackage{indentfirst}%Para indentar os paragrafos automáticamente
\usepackage{lipsum}
\usepackage{longtable}
\usepackage{mathtools}
\usepackage{listings}%Inserir codigo do R no latex
\usepackage{multirow}
\usepackage{multicol}
\usepackage{csquotes}
\usepackage[citestyle=authoryear,maxcitenames=2,terseinits=true,natbib=true, style=abnt, maxbibnames=99]{biblatex}
\addbibresource{../Referencias/Referencias.bib}
\usepackage[figuresright]{rotating}
\usepackage{spalign}
\usepackage{pgfplots}
\pgfplotsset{compat=1.17}
\usepackage{tikz}
\usepackage{fontawesome}
\usepackage{color, colortbl}
\usepackage{url}
\usepackage{cancel}
\usepackage{accents}
\usepackage{bm}
\usepackage{ragged2e}%para justificar o texto dentro de algum ambiente
\definecolor{Gray}{gray}{0.9}
\definecolor{LightCyan}{rgb}{0.88,1,1}


\usepackage[all]{xy}
\usepackage{hyperref,bookmark}
\hypersetup{
  colorlinks=true,
  linkcolor=blue,
  citecolor=red,
  filecolor=blue,
  urlcolor=blue,
}

\usetheme{Madrid}
\usecolortheme[RGB={193,0,0}]{structure}

%\setbeamertemplate{footline}[frame number]
%\setbeamertemplate{footline}[text line]{%
%  \parbox{\linewidth}{\vspace*{-8pt}\hfill\date{}\hfill\insertshortauthor\hfill\insertpagenumber}}
\beamertemplatenavigationsymbolsempty
\renewcommand{\vec}[1]{\mbox{\boldmath$#1$}}
\newtheorem{Teorema}{Teorema}
\newtheorem{Proposicao}{Proposição}
\newtheorem{definicao}{Definição}
\newtheorem{Corolario}{Corolário}
\newtheorem{Demonstracao}{Demonstração}
\newcommand{\bx}{\ensuremath{\bar{x}}}
\newcommand{\Ho}{\ensuremath{H_{0}}}
\newcommand{\Hi}{\ensuremath{H_{1}}}
\newcommand{\at}[2][]{#1|_{#2}}
\newcommand\xuparrow[1][2ex]{%
   \mathrel{\rotatebox{-90}{$\xleftarrow{\rule{#1}{0pt}}$}}
}
\apptocmd{\frame}{}{\justifying}{} % Allow optional arguments after frame.

\makeatletter
\setbeamertemplate{footline}
{
  \leavevmode%
  \hbox{%
  \begin{beamercolorbox}[wd=.3\paperwidth,ht=2.25ex,dp=1ex,center]{author in head/foot}%
    \usebeamerfont{author in head/foot}\mytext
  \end{beamercolorbox}%
  \begin{beamercolorbox}[wd=.3\paperwidth,ht=2.25ex,dp=1ex,center]{title in head/foot}%
    \usebeamerfont{title in head/foot}\mytextt
  \end{beamercolorbox}%
  \begin{beamercolorbox}[wd=.35\paperwidth,ht=2.25ex,dp=1ex,right]{site in head/foot}%
    \usebeamerfont{site in head/foot}\mytexttt\hspace*{2em}
    \insertframenumber{} / \inserttotalframenumber\hspace*{2ex} 
  \end{beamercolorbox}}%
  \vskip0pt%
}
\makeatother

\providecommand{\arcsin}{} \renewcommand{\arcsin}{\hspace{2pt}\textrm{arcsen}}
\providecommand{\sin}{} \renewcommand{\sin}{\hspace{2pt}\textrm{sen}}
\newcommand{\N}{\rm I\!N}
\newcommand{\I}{\rm I\!I}
\newcommand{\R}{\rm I\!R}
\newcommand{\Sim}{\overset{\text{iid}}{\sim}}
\newcommand{\Lim}{{\displaystyle \lim_{n\to\infty}}}
\newcommand{\LimInf}{{\displaystyle \liminf_{n\to\infty}}}
\newcommand{\rightLim}{\xrightarrow[n\rightarrow\infty]{}}
\newcommand{\Sumi}{{\displaystyle \sum_{i=1}^{n}}}
\newcommand{\Int}{{\displaystyle \int_{-\infty}^{+\infty}}}
\newcommand{\ConvD}{\overset{D}{\rightarrow}}
\newcommand{\ConvP}{\overset{P}{\rightarrow}}
\newcommand{\Prodi}{{\displaystyle \prod_{i=1}^{n}}}
\newcommand{\SetaUP}[2]{\underset{\mathclap{\substack{\xuparrow[30pt] \\ #1}}}{#2}}
%\newcommand{\SetaInclinada}[2]{\underset{\mathclap{\substack{\rotatebox{135}{\xuparrow[30pt] \\ #1}}}}{#2}}
\newcommand{\Home}{\begin{tikzpicture}
\node[scale=2] at (3,4) {\text{Para}~\faHome};
\end{tikzpicture}}
\newcommand{\vecX}{\boldsymbol{X}}
\newcommand{\Implica}[1]{\xRightarrow{#1}}
\newcommand{\SeSe}{\iff}
\newcommand{\EscoreA}{\dfrac{\partial}{\partial\theta}\log{f(x,\theta)}}
\newcommand{\EscoreB}{\dfrac{\partial^{2}}{\partial\theta^{2}}\log{f(x,\theta)}}
\newcommand{\cqd}{\text{cqd}~\blacksquare}
\newcommand{\seqX}{$X_{1},\ldots,X_{n}$}
\newcommand{\seqY}{$Y_{1},\ldots,Y_{n}$}
\newcommand{\tend}[1]{\hbox{\oalign{$\bm{#1}$\crcr\hidewidth$\scriptscriptstyle\bm{\sim}$\hidewidth}}}

%\newtheorem{Teorema}{Teorema}
%\newtheorem{Proposicao}{Proposição}
%\newtheorem{Definicao}{Definição}
%\newtheorem{Corolario}{Corolário}
%\newtheorem{Demonstracao}{Demonstração}

\titlegraphic{\hspace*{8cm}\href{https://fsbmat-ufv.github.io/}{\includegraphics[width=2cm]{figs/mylogo.png}}
}

%Continuar a numeracao em slides diferentes
\newcounter{saveenumi}
\newcommand{\seti}{\setcounter{saveenumi}{\value{enumi}}}
\newcommand{\conti}{\setcounter{enumi}{\value{saveenumi}}}

\resetcounteronoverlays{saveenumi}

% Layout da pagina
\hypersetup{pdfpagelayout=SinglePage}

%Para o \pause funcionar dentro do ambiente align
\makeatletter
\let\save@measuring@true\measuring@true
\def\measuring@true{%
  \save@measuring@true
  \def\beamer@sortzero##1{\beamer@ifnextcharospec{\beamer@sortzeroread{##1}}{}}%
  \def\beamer@sortzeroread##1<##2>{}%
  \def\beamer@finalnospec{}%
}
\makeatother
\usepackage[brazil]{babel}
\usepackage[utf8]{inputenc}
\usepackage[T1]{fontenc}
\usepackage{animate}
\usepackage{amsbsy}
\usepackage{amsfonts}
\usepackage{amsmath}
\usepackage{amssymb}
\usepackage{amsthm}
\setbeamertemplate{theorems}[numbered] % to number
\usepackage[toc,page,title,titletoc]{appendix}
\usepackage{dsfont}
\usepackage{esvect}
\usepackage[labelfont=bf]{caption}
%\usepackage{subcaption}
\usepackage{float}
\usepackage[Glenn]{fncychap}%Sonny %Conny %Lenny %Glenn %Renje %Bjarne %Bjornstrup
\usepackage{graphicx}
\usepackage{subfig}
\usepackage{indentfirst}%Para indentar os paragrafos automáticamente
\usepackage{lipsum}
\usepackage{longtable}
\usepackage{mathtools}
\usepackage{listings}%Inserir codigo do R no latex
\usepackage{multirow}
\usepackage{multicol}
\usepackage{csquotes}
\usepackage[citestyle=authoryear,maxcitenames=2,terseinits=true,natbib=true, style=abnt, maxbibnames=99]{biblatex}
\addbibresource{../Referencias/Referencias.bib}
\usepackage[figuresright]{rotating}
\usepackage{spalign}
\usepackage{pgfplots}
\pgfplotsset{compat=1.17}
\usepackage{tikz}
\usepackage{fontawesome}
\usepackage{color, colortbl}
\usepackage{url}
\usepackage{cancel}
\usepackage{accents}
\usepackage{bm}
\usepackage{ragged2e}%para justificar o texto dentro de algum ambiente
\definecolor{Gray}{gray}{0.9}
\definecolor{LightCyan}{rgb}{0.88,1,1}


\usepackage[all]{xy}
\usepackage{hyperref,bookmark}
\hypersetup{
  colorlinks=true,
  linkcolor=blue,
  citecolor=red,
  filecolor=blue,
  urlcolor=blue,
}

\usetheme{Madrid}
\usecolortheme[RGB={193,0,0}]{structure}

%\setbeamertemplate{footline}[frame number]
%\setbeamertemplate{footline}[text line]{%
%  \parbox{\linewidth}{\vspace*{-8pt}\hfill\date{}\hfill\insertshortauthor\hfill\insertpagenumber}}
\beamertemplatenavigationsymbolsempty
\renewcommand{\vec}[1]{\mbox{\boldmath$#1$}}
\newtheorem{Teorema}{Teorema}
\newtheorem{Proposicao}{Proposição}
\newtheorem{definicao}{Definição}
\newtheorem{Corolario}{Corolário}
\newtheorem{Demonstracao}{Demonstração}
\newcommand{\bx}{\ensuremath{\bar{x}}}
\newcommand{\Ho}{\ensuremath{H_{0}}}
\newcommand{\Hi}{\ensuremath{H_{1}}}
\newcommand{\at}[2][]{#1|_{#2}}
\newcommand\xuparrow[1][2ex]{%
   \mathrel{\rotatebox{-90}{$\xleftarrow{\rule{#1}{0pt}}$}}
}
\apptocmd{\frame}{}{\justifying}{} % Allow optional arguments after frame.

\makeatletter
\setbeamertemplate{footline}
{
  \leavevmode%
  \hbox{%
  \begin{beamercolorbox}[wd=.3\paperwidth,ht=2.25ex,dp=1ex,center]{author in head/foot}%
    \usebeamerfont{author in head/foot}\mytext
  \end{beamercolorbox}%
  \begin{beamercolorbox}[wd=.3\paperwidth,ht=2.25ex,dp=1ex,center]{title in head/foot}%
    \usebeamerfont{title in head/foot}\mytextt
  \end{beamercolorbox}%
  \begin{beamercolorbox}[wd=.35\paperwidth,ht=2.25ex,dp=1ex,right]{site in head/foot}%
    \usebeamerfont{site in head/foot}\mytexttt\hspace*{2em}
    \insertframenumber{} / \inserttotalframenumber\hspace*{2ex} 
  \end{beamercolorbox}}%
  \vskip0pt%
}
\makeatother

\providecommand{\arcsin}{} \renewcommand{\arcsin}{\hspace{2pt}\textrm{arcsen}}
\providecommand{\sin}{} \renewcommand{\sin}{\hspace{2pt}\textrm{sen}}
\newcommand{\N}{\rm I\!N}
\newcommand{\I}{\rm I\!I}
\newcommand{\R}{\rm I\!R}
\newcommand{\Sim}{\overset{\text{iid}}{\sim}}
\newcommand{\Lim}{{\displaystyle \lim_{n\to\infty}}}
\newcommand{\LimInf}{{\displaystyle \liminf_{n\to\infty}}}
\newcommand{\rightLim}{\xrightarrow[n\rightarrow\infty]{}}
\newcommand{\Sumi}{{\displaystyle \sum_{i=1}^{n}}}
\newcommand{\Int}{{\displaystyle \int_{-\infty}^{+\infty}}}
\newcommand{\ConvD}{\overset{D}{\rightarrow}}
\newcommand{\ConvP}{\overset{P}{\rightarrow}}
\newcommand{\Prodi}{{\displaystyle \prod_{i=1}^{n}}}
\newcommand{\SetaUP}[2]{\underset{\mathclap{\substack{\xuparrow[30pt] \\ #1}}}{#2}}
%\newcommand{\SetaInclinada}[2]{\underset{\mathclap{\substack{\rotatebox{135}{\xuparrow[30pt] \\ #1}}}}{#2}}
\newcommand{\Home}{\begin{tikzpicture}
\node[scale=2] at (3,4) {\text{Para}~\faHome};
\end{tikzpicture}}
\newcommand{\vecX}{\boldsymbol{X}}
\newcommand{\Implica}[1]{\xRightarrow{#1}}
\newcommand{\SeSe}{\iff}
\newcommand{\EscoreA}{\dfrac{\partial}{\partial\theta}\log{f(x,\theta)}}
\newcommand{\EscoreB}{\dfrac{\partial^{2}}{\partial\theta^{2}}\log{f(x,\theta)}}
\newcommand{\cqd}{\text{cqd}~\blacksquare}
\newcommand{\seqX}{$X_{1},\ldots,X_{n}$}
\newcommand{\seqY}{$Y_{1},\ldots,Y_{n}$}
\newcommand{\tend}[1]{\hbox{\oalign{$\bm{#1}$\crcr\hidewidth$\scriptscriptstyle\bm{\sim}$\hidewidth}}}

%\newtheorem{Teorema}{Teorema}
%\newtheorem{Proposicao}{Proposição}
%\newtheorem{Definicao}{Definição}
%\newtheorem{Corolario}{Corolário}
%\newtheorem{Demonstracao}{Demonstração}

\titlegraphic{\hspace*{8cm}\href{https://fsbmat-ufv.github.io/}{\includegraphics[width=2cm]{figs/mylogo.png}}
}

%Continuar a numeracao em slides diferentes
\newcounter{saveenumi}
\newcommand{\seti}{\setcounter{saveenumi}{\value{enumi}}}
\newcommand{\conti}{\setcounter{enumi}{\value{saveenumi}}}

\resetcounteronoverlays{saveenumi}

% Layout da pagina
\hypersetup{pdfpagelayout=SinglePage}

%Para o \pause funcionar dentro do ambiente align
\makeatletter
\let\save@measuring@true\measuring@true
\def\measuring@true{%
  \save@measuring@true
  \def\beamer@sortzero##1{\beamer@ifnextcharospec{\beamer@sortzeroread{##1}}{}}%
  \def\beamer@sortzeroread##1<##2>{}%
  \def\beamer@finalnospec{}%
}
\makeatother

\title{Inferência Estatística II}
\author{Prof. Fernando de Souza Bastos\texorpdfstring{\\ fernando.bastos@ufv.br}{}}
\institute{Departamento de Estatística\texorpdfstring{\\ Programa de Pós-Graduação em Estatística Aplicada e Biometria}\texorpdfstring{\\ Universidade Federal de Viçosa}{}\texorpdfstring{\\ Campus UFV - Viçosa}{}}
\date{}
\newcommand\mytext{Aula 3}
\newcommand\mytextt{Fernando de Souza Bastos}
\newcommand\mytexttt{\url{https://est711.github.io/}}


\begin{document}
%\SweaveOpts{concordance=TRUE}

\frame{\titlepage}

\begin{frame}{}
\frametitle{\bf Sumário}
\tableofcontents
\end{frame}

\section{Teoremas Sobre Convergência}
\begin{frame}{}
\begin{Teorema}
\justifying
Se $X_{n} \overset{P}{\rightarrow} X,~\text{então}~X_{n} \overset{D}{\rightarrow} X.$
\end{Teorema}
\pause
\begin{block}{Demonstração do Teorema 1}
	Seja \( x \) um ponto de continuidade de \( F_X(x) \), a função de distribuição acumulada (FDA) de \( X \). Queremos mostrar que \( F_{X_n}(x) \to F_X(x) \) à medida que \( n \to \infty \), onde \( F_{X_n}(x) \) é a FDA de \( X_n \).
	
	%Para isso, partimos da definição de \( F_{X_n}(x) = P(X_n \leq x) \). Usaremos uma técnica dividindo a probabilidade em dois pedaços.
\end{block}
\end{frame}

\begin{frame}{}
	\begin{block}{Demonstração do Teorema 1}
	Dividimos o evento \( \{X_n \leq x\} \) em dois subconjuntos: um onde \( |X_n - X| < \varepsilon \) e outro onde \( |X_n - X| \geq \varepsilon \). Assim, podemos reescrever:
	
	\begin{align*}
		F_{X_n}(x) &= P(X_n \leq x)\\ 
		&= P(\{X_n \leq x\} \cap \{|X_n - X| < \varepsilon\})\\ 
		&+ P(\{X_n \leq x\} \cap \{|X_n - X| \geq \varepsilon\})\\
		&\leq P(X \leq x + \varepsilon) + P(|X_n - X| \geq \varepsilon)
	\end{align*}
	
	Essa é uma decomposição da probabilidade em duas partes: uma onde \( X_n \) está ``perto'' de \( X \) (a diferença é menor que \( \varepsilon \)) e outra onde \( X_n \) está ``longe'' de \( X \) (a diferença é maior ou igual a \( \varepsilon \)).	
	\end{block}
\end{frame}


\begin{frame}{}
	\begin{block}{Demonstração do Teorema 1}
	%A probabilidade \( P(\{X_n \leq x\} \cap \{|X_n - X| < \varepsilon\}) \) pode ser estimada por \( P(X \leq x + \varepsilon) \).
	
	%\[
	%P(\{X_n \leq x\} \cap \{|X_n - X| < \varepsilon\}) \leq P(X \leq x + \varepsilon)
	%\]
	
	%Isso porque, 
	\begin{itemize}
		\item Quando \( |X_n - X| < \varepsilon \), sabemos que \( X_n \) está perto de \( X \), então \( X_n \leq x \) implica que \( X \leq x + \varepsilon \).	\pause
		\item O segundo termo, \( P(\{X_n \leq x\} \cap \{|X_n - X| \geq \varepsilon\}) \), é menor ou igual a \( P(|X_n - X| \geq \varepsilon) \), que é simplesmente a probabilidade de $X_{n}$ estar longe de $X.$
	\end{itemize}
	\end{block}
\end{frame}


%\begin{frame}{}
%	\begin{block}{Demonstração do Teorema 1}
%	O segundo termo, \( P(\{X_n \leq x\} \cap \{|X_n - X| \geq \varepsilon\}) \), é menor ou igual a \( P(|X_n - X| \geq \varepsilon) \), que é simplesmente a probabilidade de $X_{n}$ estar longe de $X.$ Essa probabilidade tende a 0 quando \( X_n \to X \) em probabilidade, mas por enquanto, deixamos essa expressão como está:
%	
%	\[
%	P(\{X_n \leq x\} \cap \{|X_n - X| \geq \varepsilon\}) \leq P(|X_n - X| \geq \varepsilon)
%	\]	
%	\end{block}
%\end{frame}

%\begin{frame}{}
%	\begin{block}{Demonstração do Teorema 1}
%	Juntando as duas estimativas, temos:
%	
%	\[
%	F_{X_n}(x) \leq P(X \leq x + \varepsilon) + P(|X_n - X| \geq \varepsilon)
%	\]
%	
%	Esta é a estimativa superior para \( F_{X_n}(x) \).
%	
%	
%	\begin{itemize}
%		\item O primeiro termo, \( P(X \leq x + \varepsilon) \), representa o evento de que \( X \) está um pouco acima de \( x \). Isso é um "ajuste", pois estamos lidando com \( X_n \) próximo de \( X \).\pause
%		\item O segundo termo, \( P(|X_n - X| \geq \varepsilon) \), é a probabilidade de que \( X_n \) esteja muito distante de \( X \), ou seja, mais de \( \varepsilon \) de diferença.	
%	\end{itemize}
%	\end{block}
%\end{frame}

\begin{frame}{}
	\begin{block}{Demonstração do Teorema 1}
	\( P(|X_n - X| \geq \varepsilon) \to 0 \) conforme \( n \to \infty \), pois \( X_n \to X \) em probabilidade. Portanto, podemos concluir:
		
		\[
		\lim_{n \to \infty} F_{X_n}(x) \leq F_X(x + \varepsilon)
		\]
		
		Isso nos dá a estimativa superior (upper bound) da função de distribuição acumulada de \( X_n \).
	\end{block}
\end{frame}


\begin{frame}{}
	\begin{block}{Demonstração do Teorema 1}
	Agora, para obter a \textbf{estimativa inferior}, começamos reescrevendo \( P(X_n \leq x) \) utilizando o complemento:
	
	\[
	P(X_n \leq x) = 1 - P(X_n > x)
	\]
	
	Dividimos a probabilidade \( P(X_n > x) \) em dois pedaços:
	
\begin{align*}
	P(X_n > x) &= P(\{X_n > x\} \cap \{|X_n - X| < \varepsilon\})\\
	& + P(\{X_n > x\} \cap \{|X_n - X| \geq \varepsilon\})
\end{align*}
	\end{block}
\end{frame}


%\begin{frame}{}
%	\begin{block}{Demonstração do Teorema 1}
%	\begin{itemize}
%		\item A primeira parte \( P(\{X_n > x\} \cap \{|X_n - X| < \varepsilon\}) \) considera os casos em que \( X_n \) está próximo de \( X \) (a diferença é menor que \( \varepsilon \)) e, ao mesmo tempo, \( X_n > x \).
%		\pause
%		\item A segunda parte \( P(\{X_n > x\} \cap \{|X_n - X| \geq \varepsilon\}) \) considera os casos em que \( X_n \) e \( X \) estão distantes mais de \( \varepsilon \).
%	\end{itemize}	
%	\end{block}
%\end{frame}

\begin{frame}{}
	\begin{block}{Demonstração do Teorema 1}
	Como \( P(\{X_n > x\} \cap \{|X_n - X| < \varepsilon\}) \) é menor que \( P(X > x - \varepsilon) \), podemos usar a seguinte desigualdade:
	
	\[
	P(X_n > x) \leq P(X \geq x - \varepsilon) + P(|X_n - X| \geq \varepsilon)
	\]
	
	\begin{itemize}
		\item O primeiro termo, \( P(X \geq x - \varepsilon) \), é a probabilidade de \( X \) ser maior ou igual a \( x - \varepsilon \). Isso é uma aproximação para lidar com o fato de que \( X_n \) está próximo de \( X \).\pause
		\item O segundo termo, \( P(|X_n - X| \geq \varepsilon) \), representa a probabilidade de \( X_n \) estar distante de \( X \) (mais de \( \varepsilon \)).
	\end{itemize}	
	\end{block}
\end{frame}

\begin{frame}{}
	\begin{block}{Demonstração do Teorema 1}
	Assim, podemos expressar \( P(X_n \leq x) \) como:
	
	\[
	P(X_n \leq x) = 1 - P(X_n > x)
	\]
	
	Substituímos o limite que encontramos para \( P(X_n > x) \):
	
	\[
	P(X_n \leq x) \geq 1 - P(X \geq x - \varepsilon) - P(|X_n - X| \geq \varepsilon)
	\]
	
	Ou, de forma mais compacta:
	
	\[
	F_{X_n}(x) \geq F_X(x - \varepsilon) - P(|X_n - X| \geq \varepsilon)
	\]	
	\end{block}
\end{frame}

\begin{frame}{}
	\begin{block}{Demonstração do Teorema 1}
	Sabemos que, como \( X_n \to X \) em probabilidade, temos \( P(|X_n - X| \geq \varepsilon) \to 0 \) conforme \( n \to \infty \). Assim, no limite:
	
	\[
	\lim_{n \to \infty} F_{X_n}(x) \geq F_X(x - \varepsilon)
	\]
	
	
	Agora, combinamos as duas estimativas (superior e inferior) que obtivemos:
	
	\[
	F_X(x - \varepsilon) \leq \lim_{n \to \infty} F_{X_n}(x) \leq F_X(x + \varepsilon)
	\]
	
	Finalmente, fazendo \( \varepsilon \to 0 \), chegamos à conclusão desejada:
	
	\[
	\lim_{n \to \infty} F_{X_n}(x) = F_X(x)
	\]	
	\end{block}
\end{frame}


\begin{frame}{Outros Teoremas Importantes:}
\begin{Teorema}
\justifying
Se $X_{n} \overset{D}{\rightarrow} a,~\text{então}~X_{n} \overset{P}{\rightarrow} a,~a$ constante.
\end{Teorema}
\pause
\begin{Teorema}
\justifying
Se $X_{n} \overset{D}{\rightarrow} X~\text{e}~Y_{n} \overset{P}{\rightarrow} 0~\text{então}~X_{n}+Y_{n} \overset{D}{\rightarrow} X.$
\end{Teorema}
\pause
\begin{Teorema}
\justifying
Se $X_{n} \overset{D}{\rightarrow} X~\text{e}~g~\text{é uma função contínua no suporte de}~X,~\text{então}~$ $$g(X_{n}) \overset{D}{\rightarrow} g(X).$$
\end{Teorema}
\end{frame}

\subsection{Teorema de Slutsky}
\begin{frame}{Teorema de Slutsky}
\begin{Teorema}
\justifying
Sejam $X_{n},~A_{n}$ e $B_{n},$ variáveis aleatórias com $X_{n} \overset{D}{\rightarrow} X,~A_{n} \overset{P}{\rightarrow} a~\text{e}~B_{n} \overset{P}{\rightarrow} b,~a,b$~constantes reais. Então, $$A_{n}X_{n}+B_{n} \overset{D}{\rightarrow} aX+b.$$
\end{Teorema}
\end{frame}

\begin{frame}{}
\begin{block}{\Home}
	\nocite{hogg}
\justifying
Exercícios 5.2.2, 5.2.3, 5.2.6, 5.2.12, 5.2.15, 5.2.17, 5.2.19 e 5.2.20
\end{block}
\end{frame}

\section{Momentos de Uma Variável Aleatória}
\begin{frame}{Momentos de Ordem $k$}
	\begin{definicao}\label{def4}
		\justifying
		Para $k=1,2,\dots,$ o momento de ordem $k$ da variável $X$ é definido por $E(X^{k}),$ desde que essa quantidade exista. Se $E(X)=\mu<\infty$, definimos o momento central de ordem $k$ por $E[(X-\mu)^{k}],$ sempre que essa quantidade exista. De modo similar, o momento absoluto de ordem $k$ da variável aleatória $X$ é definido por $E(|X|^{k}).$
	\end{definicao}
\end{frame}

\begin{frame}{Exemplo 1}
	\begin{block}{}
		\justifying
		Considerando \( X \sim \Gamma(\alpha, \beta) \), calcule seus momentos.
		
		\begin{align*}
			E(X^k) &= \int_{0}^{\infty} x^k \cdot \frac{\beta^{\alpha}x^{\alpha-1} \cdot e^{-\beta x}}{\Gamma(\alpha)} \, dx \\
			&=\frac{\beta^{\alpha}}{\Gamma(\alpha)} \int_{0}^{\infty} x^{\alpha + k - 1} \cdot e^{-\beta x} \, dx
		\end{align*}
		Sabe-se que
		
		\[ \int_{0}^{\infty} x^{p-1} \cdot e^{-qx} \, dx = \Gamma(p) \cdot q^{-p} \]
		
	\end{block}
\end{frame}

\begin{frame}{Exemplo 1}
	\begin{block}{}
		\justifying
		Aplicando essa propriedade à integral, obtemos:
		
		\[ \int_{0}^{\infty} x^{\alpha + k - 1} \cdot e^{-\beta x} \, dx = \Gamma(\alpha + k) \cdot (\beta)^{-(\alpha + k)} \]
		
		Substituindo na expressão original:
		
		\[E(X^k) = \frac{\beta^{\alpha}}{\Gamma(\alpha)} \cdot \Gamma(\alpha + k) \cdot (\beta)^{-(\alpha + k)} \]
		
		\[E(X^k) = \frac{\Gamma(\alpha + k)}{\Gamma(\alpha)} \cdot \beta^{-k} = \dfrac{\alpha(\alpha+1)\cdots(\alpha+k-1)}{\beta^{k}}\]
		
		Esse é o valor esperado \(E(X^k)\) em termos das funções Gama e do parâmetro \(\beta\).
		
	\end{block}
\end{frame}

\begin{frame}{Exemplo 1}
	\begin{block}{}
		\justifying
		Observe que se $\alpha=1,~X$ tem distribuição exponencial de parâmetro $\beta>0$ e, assim, 
		\begin{align*}
			E(X^k) = \dfrac{k!}{\beta^{k}}   
		\end{align*}
		Se fizermos $k=1,$ obtemos a média desta variável que é $\dfrac{1}{\beta}.$
	\end{block}
\end{frame}

\begin{frame}{Exemplo 2}
	\begin{block}{}
		\justifying
		Sabendo que $X\Sim B(n,p).$ Encontre o momento central de ordem 1 e 2 desta variável.
		\begin{align*}
			E(X^{2})&={\displaystyle \sum_{i=0}^{n}i^{2}\binom{n}{i}p^{i}(1-p)^{n-i}}\\
			&={\displaystyle \sum_{i=0}^{n}[i(i-1)+i]\binom{n}{i}p^{i}(1-p)^{n-i}}\\
			&={\displaystyle \sum_{i=2}^{n}i(i-1)\binom{n}{i}p^{i}(1-p)^{n-i}}+E(X)\\
			&={\displaystyle n(n-1)\sum_{i=2}^{n}\binom{n-2}{i-2}p^{i}(1-p)^{n-i}}+E(X)
		\end{align*}
	\end{block}
\end{frame}

\begin{frame}{Exemplo 2}
	\begin{block}{}
		\justifying
		\begin{align*}
			E(X^{2})&={\displaystyle n(n-1)p^{2}\sum_{j=0}^{n-2}\binom{n-2}{j}p^{j}(1-p)^{(n-2)-j}}+E(X)\\
			&=n(n-1)p^{2}[p+(1-p)]^{(n-2)}+np\\
			&=n^{2}p^{2}+np(1-p)
		\end{align*}
	\end{block}
	\pause
	\begin{block}{}
		\justifying
		Assim, $Var(X)=E{(X-E(X))^{2}}=E(X^{2})-(E(X))^{2}=np(1-p)$
	\end{block}
\end{frame}

\section{Função Geradora de Momentos}
\begin{frame}{Função Geradora de Momentos}
	\begin{definicao}
		\justifying
		A função geradora de momentos de uma variável aleatória $X$ é definida por $M_{X}(t)=E(e^{tX}),~t\in \R$
	\end{definicao}
	\pause
	\begin{block}{Observação Importante:}
		\justifying
		O momento de ordem \( k \) de uma variável aleatória \( X \) pode ser encontrado utilizando a função geradora de momentos. Para encontrar o momento de ordem \( k \), derivamos \( k \) vezes em relação a \( t \) a função geradora de momentos \( M_X(t) \) e então avaliamos em \( t = 0 \):
		
		\[
		\mathbb{E}[X^k] = \frac{d^k}{dt^k} M_X(t) \bigg|_{t=0}
		\]
	\end{block}
\end{frame}

\begin{frame}{}
	\begin{Teorema}
		\justifying
		Seja $\{X_{n}\}_{n\geq 1}$ uma sequência de variáveis aleatórias com fgm $M_{X_n}(t)$ que existe para $|t|<h$ para todo $n$. Seja $X$ uma variável aleatória com fgm $M_{X}(t)$, que existe para $|t| \leq h_1 \leq h$. Se $\lim_{n \to \infty} M_{X_n}(t) = M_{X}(t)$ para $|t| \leq h_1$, então $X_{n} \overset{D}{\rightarrow} X.$
	\end{Teorema}
	\pause
	\begin{block}{Observação importante na resolução de exercícios:}
		\justifying
		Se ${\displaystyle \lim_{n \to \infty} \left(1 + \frac{b}{n} + \frac{\psi(n)}{n}\right)^{cn}}$, em que $b$ e $c$ não dependem de $n$ e, em que, ${\displaystyle\lim_{n \to \infty} \psi(n) = 0}$. Então,
		${\displaystyle \lim_{n \to \infty} \left(1 + \frac{b}{n} + \frac{\psi(n)}{n}\right)^{cn} = \lim_{n \to \infty} \left(1 + \frac{b}{n}\right)^{cn} = e^{bc}}.$
	\end{block}
\end{frame}

\begin{frame}{Exemplo 1}
	\begin{block}{}
		\justifying
		
		\begin{align*}
			{\displaystyle \lim_{{n \to \infty}} \left(1 - \frac{{t^2}}{{n}} + \frac{{t^2}}{{n^{3/2}}}\right)^{-n/2}} &= {\displaystyle \lim_{{n \to \infty}} \left(1 - \frac{{t^2}}{{n}} + \frac{{t^2}/\sqrt{n}}{{n}}\right)^{-n/2}}\\ \pause
			&={\displaystyle \lim_{{n \to \infty}} \left(1 - \frac{{t^2}}{{n}}\right)^{-n/2}}\\
			&= e^{t^2/2}
		\end{align*}
		
%		${\displaystyle \lim_{{n \to \infty}} \left(1 - \frac{{t^2}}{{n}} + \frac{{t^2}}{{n^{3/2}}}\right)^{-n/2}} = {\displaystyle \lim_{{n \to \infty}} \left(1 - \frac{{t^2}}{{n}} + \frac{{t^2}/\sqrt{n}}{{n}}\right)^{-n/2}}$.
		
		Aqui, $b = -t^2$, $c = -\frac{1}{2}$ e $\psi(n) = \frac{{t^2}}{{\sqrt{n}}}$. %Consequentemente, para cada valor fixo de $t$, o limite é $e^{t^2/2}$.
	\end{block}
\end{frame}

\begin{frame}{Exemplo 2}
	\vspace{-0.5cm}
	\begin{block}{}
		\justifying
		Considere $X_{n}\sim Binomial(n,p_{n})$ e suponha ${\displaystyle \Lim np_{n}=\lambda>0}$ (por exemplo, $p_{n}=\dfrac{1}{n+1},~\Lim np_{n}=1$). Então, $X_{n} \overset{D}{\rightarrow} X,$ em que $X\sim$Poisson($\lambda$).
	\end{block}
	\pause
	\vspace{-0.3cm}
	\begin{block}{Demonstração}
		\justifying
		Temos que,
		\begin{align*}
			M_{X_{n}}(t)&=E(e^{tX_{n}})={\displaystyle \sum_{k=0}^{n}e^{tk}\binom{n}{k}p_{n}^{k}(1-p_{n})^{n-k}}\\
			&=\Big(1-p_{n}+p_{n}e^{t}\Big)^{n}=\Big(1+\dfrac{np_{n}}{n}(e^{t}-1)\Big)^{n}\\
			\text{(para n grande)}    &=\Big(1+\dfrac{\lambda}{n}(e^{t}-1)\Big)^{n}\rightLim \exp{\{\lambda(e^{t}-1)\}}
		\end{align*}
		Logo, $X_{n} \overset{D}{\rightarrow} X\sim$Poisson($\lambda$).
	\end{block}
\end{frame}

\begin{frame}
	\begin{block}{}
		\justifying
		Quando a quantidade \( np_n \) se estabiliza em um valor \( \lambda > 0 \), estamos essencialmente controlando a média da binomial. À medida que \( n \to \infty \) e \( p_n \) diminui de forma controlada, mantemos \( np_n \) constante, aproximando o comportamento da binomial ao de uma distribuição Poisson com parâmetro \( \lambda \). A essência é que estamos explorando o comportamento assintótico da binomial, com \( p_n \) diminuindo à medida que \( n \) cresce, mas de modo que \( np_n \) permaneça fixo e igual a \( \lambda \). Isso faz com que a média e variância da binomial ``convirjam'' para os parâmetros de uma Poisson.
		%\textbf{Transformação da expressão}
		%Quando reescrevemos o momento gerador:
		%\[
		%\left(1 - p_n + p_n e^t\right)^n = \left(1 + \frac{np_n}{n}(e^t - 1)\right)^n
		%\]
		%fazemos isso porque \( np_n \) tende a \( \lambda \). Ou seja, estamos preparando a fórmula para o limite \( n \to \infty \).
		
		%No limite:
		%\[
		%np_n \approx \lambda \quad \text{para grandes } n
		%\]
		%Assim, substituímos \( \frac{np_n}{n} \) por \( \frac{\lambda}{n} \) e fazemos \( n \to \infty \). Isso nos leva ao limite exponencial:
		%\[
		%\left(1 + \frac{\lambda}{n}(e^t - 1)\right)^n \to \exp\left(\lambda (e^t - 1)\right)
		%\]
		%Essa é a razão pela qual manipulamos a expressão de modo a colocar \( np_n \) em evidência.
		
	\end{block}
\end{frame}

\begin{frame}[allowframebreaks]
\frametitle{\bf Referências}
\printbibliography
\end{frame}


\end{document}
