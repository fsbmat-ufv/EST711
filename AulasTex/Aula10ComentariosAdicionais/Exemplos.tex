\documentclass[12pt]{beamer}

\usepackage[brazil]{babel}
\usepackage[utf8]{inputenc}
\usepackage[T1]{fontenc}
\usepackage{animate}
\usepackage{amsbsy}
\usepackage{amsfonts}
\usepackage{amsmath}
\usepackage{amssymb}
\usepackage{amsthm}
\setbeamertemplate{theorems}[numbered] % to number
\usepackage[toc,page,title,titletoc]{appendix}
\usepackage{dsfont}
\usepackage{esvect}
\usepackage[labelfont=bf]{caption}
%\usepackage{subcaption}
\usepackage{float}
\usepackage[Glenn]{fncychap}%Sonny %Conny %Lenny %Glenn %Renje %Bjarne %Bjornstrup
\usepackage{graphicx}
\usepackage{subfig}
\usepackage{indentfirst}%Para indentar os paragrafos automáticamente
\usepackage{lipsum}
\usepackage{longtable}
\usepackage{mathtools}
\usepackage{listings}%Inserir codigo do R no latex
\usepackage{multirow}
\usepackage{multicol}
\usepackage{csquotes}
\usepackage[citestyle=authoryear,maxcitenames=2,terseinits=true,natbib=true, style=abnt, maxbibnames=99]{biblatex}
\addbibresource{../Referencias/Referencias.bib}
\usepackage[figuresright]{rotating}
\usepackage{spalign}
\usepackage{pgfplots}
\pgfplotsset{compat=1.17}
\usepackage{tikz}
\usepackage{fontawesome}
\usepackage{color, colortbl}
\usepackage{url}
\usepackage{cancel}
\usepackage{accents}
\usepackage{bm}
\usepackage{ragged2e}%para justificar o texto dentro de algum ambiente
\definecolor{Gray}{gray}{0.9}
\definecolor{LightCyan}{rgb}{0.88,1,1}


\usepackage[all]{xy}
\usepackage{hyperref,bookmark}
\hypersetup{
  colorlinks=true,
  linkcolor=blue,
  citecolor=red,
  filecolor=blue,
  urlcolor=blue,
}

\usetheme{Madrid}
\usecolortheme[RGB={193,0,0}]{structure}

%\setbeamertemplate{footline}[frame number]
%\setbeamertemplate{footline}[text line]{%
%  \parbox{\linewidth}{\vspace*{-8pt}\hfill\date{}\hfill\insertshortauthor\hfill\insertpagenumber}}
\beamertemplatenavigationsymbolsempty
\renewcommand{\vec}[1]{\mbox{\boldmath$#1$}}
\newtheorem{Teorema}{Teorema}
\newtheorem{Proposicao}{Proposição}
\newtheorem{definicao}{Definição}
\newtheorem{Corolario}{Corolário}
\newtheorem{Demonstracao}{Demonstração}
\newcommand{\bx}{\ensuremath{\bar{x}}}
\newcommand{\Ho}{\ensuremath{H_{0}}}
\newcommand{\Hi}{\ensuremath{H_{1}}}
\newcommand{\at}[2][]{#1|_{#2}}
\newcommand\xuparrow[1][2ex]{%
   \mathrel{\rotatebox{-90}{$\xleftarrow{\rule{#1}{0pt}}$}}
}
\apptocmd{\frame}{}{\justifying}{} % Allow optional arguments after frame.

\makeatletter
\setbeamertemplate{footline}
{
  \leavevmode%
  \hbox{%
  \begin{beamercolorbox}[wd=.3\paperwidth,ht=2.25ex,dp=1ex,center]{author in head/foot}%
    \usebeamerfont{author in head/foot}\mytext
  \end{beamercolorbox}%
  \begin{beamercolorbox}[wd=.3\paperwidth,ht=2.25ex,dp=1ex,center]{title in head/foot}%
    \usebeamerfont{title in head/foot}\mytextt
  \end{beamercolorbox}%
  \begin{beamercolorbox}[wd=.35\paperwidth,ht=2.25ex,dp=1ex,right]{site in head/foot}%
    \usebeamerfont{site in head/foot}\mytexttt\hspace*{2em}
    \insertframenumber{} / \inserttotalframenumber\hspace*{2ex} 
  \end{beamercolorbox}}%
  \vskip0pt%
}
\makeatother

\providecommand{\arcsin}{} \renewcommand{\arcsin}{\hspace{2pt}\textrm{arcsen}}
\providecommand{\sin}{} \renewcommand{\sin}{\hspace{2pt}\textrm{sen}}
\newcommand{\N}{\rm I\!N}
\newcommand{\I}{\rm I\!I}
\newcommand{\R}{\rm I\!R}
\newcommand{\Sim}{\overset{\text{iid}}{\sim}}
\newcommand{\Lim}{{\displaystyle \lim_{n\to\infty}}}
\newcommand{\LimInf}{{\displaystyle \liminf_{n\to\infty}}}
\newcommand{\rightLim}{\xrightarrow[n\rightarrow\infty]{}}
\newcommand{\Sumi}{{\displaystyle \sum_{i=1}^{n}}}
\newcommand{\Int}{{\displaystyle \int_{-\infty}^{+\infty}}}
\newcommand{\ConvD}{\overset{D}{\rightarrow}}
\newcommand{\ConvP}{\overset{P}{\rightarrow}}
\newcommand{\Prodi}{{\displaystyle \prod_{i=1}^{n}}}
\newcommand{\SetaUP}[2]{\underset{\mathclap{\substack{\xuparrow[30pt] \\ #1}}}{#2}}
%\newcommand{\SetaInclinada}[2]{\underset{\mathclap{\substack{\rotatebox{135}{\xuparrow[30pt] \\ #1}}}}{#2}}
\newcommand{\Home}{\begin{tikzpicture}
\node[scale=2] at (3,4) {\text{Para}~\faHome};
\end{tikzpicture}}
\newcommand{\vecX}{\boldsymbol{X}}
\newcommand{\Implica}[1]{\xRightarrow{#1}}
\newcommand{\SeSe}{\iff}
\newcommand{\EscoreA}{\dfrac{\partial}{\partial\theta}\log{f(x,\theta)}}
\newcommand{\EscoreB}{\dfrac{\partial^{2}}{\partial\theta^{2}}\log{f(x,\theta)}}
\newcommand{\cqd}{\text{cqd}~\blacksquare}
\newcommand{\seqX}{$X_{1},\ldots,X_{n}$}
\newcommand{\seqY}{$Y_{1},\ldots,Y_{n}$}
\newcommand{\tend}[1]{\hbox{\oalign{$\bm{#1}$\crcr\hidewidth$\scriptscriptstyle\bm{\sim}$\hidewidth}}}

%\newtheorem{Teorema}{Teorema}
%\newtheorem{Proposicao}{Proposição}
%\newtheorem{Definicao}{Definição}
%\newtheorem{Corolario}{Corolário}
%\newtheorem{Demonstracao}{Demonstração}

\titlegraphic{\hspace*{8cm}\href{https://fsbmat-ufv.github.io/}{\includegraphics[width=2cm]{figs/mylogo.png}}
}

%Continuar a numeracao em slides diferentes
\newcounter{saveenumi}
\newcommand{\seti}{\setcounter{saveenumi}{\value{enumi}}}
\newcommand{\conti}{\setcounter{enumi}{\value{saveenumi}}}

\resetcounteronoverlays{saveenumi}

% Layout da pagina
\hypersetup{pdfpagelayout=SinglePage}

%Para o \pause funcionar dentro do ambiente align
\makeatletter
\let\save@measuring@true\measuring@true
\def\measuring@true{%
  \save@measuring@true
  \def\beamer@sortzero##1{\beamer@ifnextcharospec{\beamer@sortzeroread{##1}}{}}%
  \def\beamer@sortzeroread##1<##2>{}%
  \def\beamer@finalnospec{}%
}
\makeatother

\title{Inferência Estatística II}
\author{Prof. Fernando de Souza Bastos\texorpdfstring{\\ fernando.bastos@ufv.br}{}}
\institute{Departamento de Estatística\texorpdfstring{\\ Programa de Pós-Graduação em Estatística Aplicada e Biometria}\texorpdfstring{\\ Universidade Federal de Viçosa}{}\texorpdfstring{\\ Campus UFV - Viçosa}{}}
\date{}
\newcommand\mytext{Exemplos de TH}
\newcommand\mytextt{Fernando de Souza Bastos}
\newcommand\mytexttt{\url{https://est711.github.io/}}


\begin{document}
%\SweaveOpts{concordance=TRUE}

\frame{\titlepage}

\begin{frame}{}
\frametitle{\bf Sumário}
\tableofcontents
\end{frame}

\section{Exemplo 1: Teste unilateral  a direita para a Média Baseado em Grandes Amostras}
\begin{frame}{Teste unilateral para a Média Baseado em Grandes Amostras}
\begin{block}{}
\justifying
Vamos testar a seguinte hipótese nula e alternativa:

\[
H_0: \mu \leq \mu_0 \quad \text{contra} \quad H_1: \mu > \mu_0
\]

Em que $\mu_0 = 100$ e $\mu$ é a média populacional desconhecida. Suponha que temos uma amostra de tamanho $n = 36$, com desvio padrão populacional conhecido $\sigma = 12$, e o nível de significância $\alpha = 0,05$.

\end{block}
\pause
\begin{block}{}
	\justifying
Vamos rejeitar $H_0$ se $\bar{X}>x_{c}>100,$ tal que $\alpha=P_{\text{Sob}~H_{0}}(\bar{X}>x_{c})$
\[
Z_{\alpha} = 1,645
\]	
\end{block}
\end{frame}

\begin{frame}{}
\begin{block}{Cálculo do Valor Crítico sob $H_{0}$}
\justifying
\begin{align*}
	0,05&=P_{\text{Sob}~H_{0}}(\bar{X}>x_{c})=P(Z>\dfrac{(x_{c}-100)\times6}{12})\\
	&\Rightarrow \dfrac{(x_{c}-100)\times6}{12}=1,645\\
	&\Rightarrow x_{c}=103,29
\end{align*}
\end{block}
\pause
\begin{block}{Cálculo do Poder do Teste}
	\justifying
	O poder do teste é a probabilidade de rejeitar $H_0$ quando a verdadeira média $\mu$ é maior que $\mu_0$. Assumindo que a verdadeira média é $\mu = 105:$
	\begin{align*}
\gamma(\mu=105)&=P_{\mu=105}(\bar{X}>x_{c})=P(\bar{X}>103,29)\\
	           &=P(Z>\dfrac{(103,29-105)\times6}{12})=P(Z>-0,855)\\
    	       &=0,804
	\end{align*}
\end{block}
\end{frame}

\begin{frame}{}
	\begin{block}{}
Vejam os gráficos de poder \href{https://est711.shinyapps.io/FuncaoPoder/}{cliquem aqui!}
	\end{block}
\end{frame}

\section{Exemplo 2: Teste unilateral a esquerda para a Média Baseado em Grandes Amostras}
\begin{frame}{Teste unilateral para a Média Baseado em Grandes Amostras}
	\begin{block}{}
		\justifying
	Vamos testar a seguinte hipótese nula e alternativa:
	
	\[
	H_0: \mu = \mu_0 = 50 \quad \text{contra} \quad H_1: \mu < 50
	\]
	
	Em que \( \mu_0 = 50 \) é a média sob \( H_0 \), e \( \mu \) é a média populacional desconhecida. Temos uma amostra de tamanho \( n = 36 \), com desvio padrão populacional \( \sigma = 8 \), e o nível de significância \( \alpha = 0.05 \).
	
		
	\end{block}
	\pause
	\begin{block}{}
		\justifying
		Vamos rejeitar $H_0$ se $\bar{X}<x_{c}<50,$ tal que $\alpha=P_{\text{Sob}~H_{0}}(\bar{X}<x_{c})$
		\[
		Z_{\alpha} = -1,645
		\]	
	\end{block}
\end{frame}

\begin{frame}{}
	\begin{block}{Cálculo do Valor Crítico sob $H_{0}$}
		\justifying
		\begin{align*}
			0,05&=P_{\text{Sob}~H_{0}}(\bar{X}<x_{c})=P(Z<\dfrac{(x_{c}-50)\times6}{8})\\
			&\Rightarrow \dfrac{(x_{c}-50)\times6}{8}=-1,645\\
			&\Rightarrow x_{c}=47,81
		\end{align*}
	\end{block}
	\pause
	\begin{block}{Cálculo do Poder do Teste}
		\justifying
		O poder do teste é a probabilidade de rejeitar $H_0$ quando a verdadeira média $\mu$ é menor que $\mu_0$. Vamos assumir que a verdadeira média é $\mu = 47:$
		\begin{align*}
			\gamma(47)&=P_{\mu=47}(\bar{X}<x_{c})=P(\bar{X}<47,81)\\
			&=P(Z<\dfrac{(47,81-47)\times6}{8})=P(Z<0,6075)\\
			&=0,728
		\end{align*}
	\end{block}
\end{frame}

\begin{frame}{}
	\begin{block}{}
		Vejam os gráficos de poder \href{https://est711.shinyapps.io/FuncaoPoder/}{cliquem aqui!}
	\end{block}
\end{frame}

\section{Exemplo 3: Teste unilateral para a Proporção Binomial}
\begin{frame}{}
\begin{block}{}
\justifying
Vamos testar a seguinte hipótese nula e alternativa:

\[
H_0: p = p_0 = 0,4 \quad \text{contra} \quad H_1: p > 0,4
\]

O tamanho da amostra é \(n = 100\), o nível de significância é \(\alpha = 0,01\), e assumimos que a proporção real sob \(H_1\) é \(p = 0,55\).
\end{block}
\pause
\begin{block}{}
	\justifying
	Vamos rejeitar $H_0$ se $\bar{p}>p_{c}>0,4,$ tal que $\alpha=P_{\text{Sob}~H_{0}}(\bar{p}>p_{c})$
	\[
	Z_{\alpha} = 2,33
	\]	
\end{block}
\end{frame}

\begin{frame}{}
	\begin{block}{Cálculo do Valor Crítico sob $H_{0}$}
		\justifying
		\begin{align*}
			0,01&=P_{\text{Sob}~H_{0}}(\bar{p}>p_{c})=P(Z>\dfrac{(p_{c}-0,4)\times10}{\sqrt{0,04\times(1-0,04)}})\\
			&\Rightarrow \dfrac{(p_{c}-0,4)\times10}{\sqrt{0,4\times(1-0,4)}}=2,33\Rightarrow p_{c}=0,5141
		\end{align*}
	\end{block}
	\pause
	\begin{block}{Cálculo do Poder do Teste}
		\justifying
		O poder do teste é a probabilidade de rejeitar $H_0$ quando a verdadeira proporção $p$ é maior que $0,4$. Assumindo que a verdadeira proporção é $p=0,55:$
		\begin{align*}
			\gamma(p=0,55)&=P_{p=0,55}(\bar{p}>p_{c})=P(\bar{p}>0,5141)\\
			&=P(Z>\dfrac{(0,5141-0,55)\times10}{\sqrt{0,55\times(1-0,55)}})=P(Z>-0,722)\\
			&=0,7648
		\end{align*}
	\end{block}
\end{frame}

\begin{frame}{}
	\begin{block}{}
		Vejam os gráficos de poder \href{https://est711.shinyapps.io/FuncaoPoder/}{cliquem aqui!}
	\end{block}
\end{frame}

\section{Exemplo 4: Teste Bilateral para a Média}
\begin{frame}{Teste Bilateral para a Média Baseado em Grandes Amostras}
	\begin{block}{}
		\justifying
		Vamos testar a seguinte hipótese nula e alternativa:
		
		\[
		H_0: \mu = \mu_0 = 50 \quad \text{contra} \quad H_1: \mu \neq 50
		\]
		
		Onde \( \mu_0 = 50 \) é a média sob \( H_0 \) e \( \mu \) é a média populacional desconhecida. Temos uma amostra de tamanho \( n = 36 \), com desvio padrão populacional \( \sigma = 10 \), e o nível de significância \( \alpha = 0.05 \).
		
	\end{block}
	\pause
	\begin{block}{}
		\justifying
		Vamos rejeitar $H_0$ se $\bar{X}>k>50,$ ou se $\bar{X}<h<50,$ tal que $\alpha=P_{\text{Sob}~H_{0}}(\bar{X}>k)+P_{\text{Sob}~H_{0}}(\bar{X}<h),$ assim,
		\[
		\dfrac{\alpha}{2}=P_{\text{Sob}~H_{0}}(\bar{X}>k)=P_{\text{Sob}~H_{0}}(\bar{X}<h)\Rightarrow Z_{\frac{\alpha}{2}} = 1,96
		\]	
	\end{block}
\end{frame}

\begin{frame}{}
	\begin{block}{Cálculo do Valor Crítico sob $H_{0}$ para o lado direito}
		\justifying
		\begin{align*}
			0,025&=P_{\text{Sob}~H_{0}}(\bar{X}>k)=P(Z>\dfrac{(k-50)\times6}{10})\\
			&\Rightarrow \dfrac{(k-50)\times6}{10}=1,96\\
			&\Rightarrow x_{c}=53,27
		\end{align*}
	\end{block}
	\pause
	\begin{block}{Cálculo do Valor Crítico sob $H_{0}$ para o lado esquerdo}
		\justifying
		\begin{align*}
			0,025&=P_{\text{Sob}~H_{0}}(\bar{X}<h)=P(Z<\dfrac{(h-50)\times6}{10})\\
			&\Rightarrow \dfrac{(h-50)\times6}{10}=-1,96\\
			&\Rightarrow x_{c}=46,73
		\end{align*}
	\end{block}
\end{frame}

\begin{frame}{}
	\begin{block}{Cálculo do Poder do Teste}
		\justifying
A função poder é a probabilidade de rejeitar \( H_0 \) para $\mu\in \mathbb{R}.$
		\begin{align*}
			\gamma(\mu)&=P_{\mu}(\bar{X}>53,27)=P_{\mu}(\bar{X}<46,73)\\
		\end{align*}
	\end{block}
	\pause
	\begin{block}{Cálculo do Poder do Teste}
		\justifying
		O poder do Teste é a probabilidade de rejeitar $H_{0}$ quando a média verdadeira não é $50.$ Se $\mu=54,$ o poder do teste é:
		\begin{align*}
			\gamma(\mu)&=P_{\mu = 54}(\bar{X}>53,27)=P_{\mu = 54}(\bar{X}<46,73)\\
			&=P(Z < -4,36) + P(Z > -0,438)\\
			&\approx P(Z > -0,438) = 0,67
		\end{align*}
	\end{block}
	\nocite{hogg}
\end{frame}

\begin{frame}{}
	\begin{block}{}
		Vejam os gráficos de poder \href{https://est711.shinyapps.io/FuncaoPoder/}{cliquem aqui!}
	\end{block}
\end{frame}

\begin{frame}[allowframebreaks]
\frametitle{\bf Referências}
\printbibliography
\end{frame}


\end{document}
