\documentclass[12pt]{beamer}

\usepackage[brazil]{babel}
\usepackage[utf8]{inputenc}
\usepackage[T1]{fontenc}
\usepackage{animate}
\usepackage{amsbsy}
\usepackage{amsfonts}
\usepackage{amsmath}
\usepackage{amssymb}
\usepackage{amsthm}
\setbeamertemplate{theorems}[numbered] % to number
\usepackage[toc,page,title,titletoc]{appendix}
\usepackage{dsfont}
\usepackage{esvect}
\usepackage[labelfont=bf]{caption}
%\usepackage{subcaption}
\usepackage{float}
\usepackage[Glenn]{fncychap}%Sonny %Conny %Lenny %Glenn %Renje %Bjarne %Bjornstrup
\usepackage{graphicx}
\usepackage{subfig}
\usepackage{indentfirst}%Para indentar os paragrafos automáticamente
\usepackage{lipsum}
\usepackage{longtable}
\usepackage{mathtools}
\usepackage{listings}%Inserir codigo do R no latex
\usepackage{multirow}
\usepackage{multicol}
\usepackage{csquotes}
\usepackage[citestyle=authoryear,maxcitenames=2,terseinits=true,natbib=true, style=abnt, maxbibnames=99]{biblatex}
\addbibresource{../Referencias/Referencias.bib}
\usepackage[figuresright]{rotating}
\usepackage{spalign}
\usepackage{pgfplots}
\pgfplotsset{compat=1.17}
\usepackage{tikz}
\usepackage{fontawesome}
\usepackage{color, colortbl}
\usepackage{url}
\usepackage{cancel}
\usepackage{accents}
\usepackage{bm}
\usepackage{ragged2e}%para justificar o texto dentro de algum ambiente
\definecolor{Gray}{gray}{0.9}
\definecolor{LightCyan}{rgb}{0.88,1,1}


\usepackage[all]{xy}
\usepackage{hyperref,bookmark}
\hypersetup{
  colorlinks=true,
  linkcolor=blue,
  citecolor=red,
  filecolor=blue,
  urlcolor=blue,
}

\usetheme{Madrid}
\usecolortheme[RGB={193,0,0}]{structure}

%\setbeamertemplate{footline}[frame number]
%\setbeamertemplate{footline}[text line]{%
%  \parbox{\linewidth}{\vspace*{-8pt}\hfill\date{}\hfill\insertshortauthor\hfill\insertpagenumber}}
\beamertemplatenavigationsymbolsempty
\renewcommand{\vec}[1]{\mbox{\boldmath$#1$}}
\newtheorem{Teorema}{Teorema}
\newtheorem{Proposicao}{Proposição}
\newtheorem{definicao}{Definição}
\newtheorem{Corolario}{Corolário}
\newtheorem{Demonstracao}{Demonstração}
\newcommand{\bx}{\ensuremath{\bar{x}}}
\newcommand{\Ho}{\ensuremath{H_{0}}}
\newcommand{\Hi}{\ensuremath{H_{1}}}
\newcommand{\at}[2][]{#1|_{#2}}
\newcommand\xuparrow[1][2ex]{%
   \mathrel{\rotatebox{-90}{$\xleftarrow{\rule{#1}{0pt}}$}}
}
\apptocmd{\frame}{}{\justifying}{} % Allow optional arguments after frame.

\makeatletter
\setbeamertemplate{footline}
{
  \leavevmode%
  \hbox{%
  \begin{beamercolorbox}[wd=.3\paperwidth,ht=2.25ex,dp=1ex,center]{author in head/foot}%
    \usebeamerfont{author in head/foot}\mytext
  \end{beamercolorbox}%
  \begin{beamercolorbox}[wd=.3\paperwidth,ht=2.25ex,dp=1ex,center]{title in head/foot}%
    \usebeamerfont{title in head/foot}\mytextt
  \end{beamercolorbox}%
  \begin{beamercolorbox}[wd=.35\paperwidth,ht=2.25ex,dp=1ex,right]{site in head/foot}%
    \usebeamerfont{site in head/foot}\mytexttt\hspace*{2em}
    \insertframenumber{} / \inserttotalframenumber\hspace*{2ex} 
  \end{beamercolorbox}}%
  \vskip0pt%
}
\makeatother

\providecommand{\arcsin}{} \renewcommand{\arcsin}{\hspace{2pt}\textrm{arcsen}}
\providecommand{\sin}{} \renewcommand{\sin}{\hspace{2pt}\textrm{sen}}
\newcommand{\N}{\rm I\!N}
\newcommand{\I}{\rm I\!I}
\newcommand{\R}{\rm I\!R}
\newcommand{\Sim}{\overset{\text{iid}}{\sim}}
\newcommand{\Lim}{{\displaystyle \lim_{n\to\infty}}}
\newcommand{\LimInf}{{\displaystyle \liminf_{n\to\infty}}}
\newcommand{\rightLim}{\xrightarrow[n\rightarrow\infty]{}}
\newcommand{\Sumi}{{\displaystyle \sum_{i=1}^{n}}}
\newcommand{\Int}{{\displaystyle \int_{-\infty}^{+\infty}}}
\newcommand{\ConvD}{\overset{D}{\rightarrow}}
\newcommand{\ConvP}{\overset{P}{\rightarrow}}
\newcommand{\Prodi}{{\displaystyle \prod_{i=1}^{n}}}
\newcommand{\SetaUP}[2]{\underset{\mathclap{\substack{\xuparrow[30pt] \\ #1}}}{#2}}
%\newcommand{\SetaInclinada}[2]{\underset{\mathclap{\substack{\rotatebox{135}{\xuparrow[30pt] \\ #1}}}}{#2}}
\newcommand{\Home}{\begin{tikzpicture}
\node[scale=2] at (3,4) {\text{Para}~\faHome};
\end{tikzpicture}}
\newcommand{\vecX}{\boldsymbol{X}}
\newcommand{\Implica}[1]{\xRightarrow{#1}}
\newcommand{\SeSe}{\iff}
\newcommand{\EscoreA}{\dfrac{\partial}{\partial\theta}\log{f(x,\theta)}}
\newcommand{\EscoreB}{\dfrac{\partial^{2}}{\partial\theta^{2}}\log{f(x,\theta)}}
\newcommand{\cqd}{\text{cqd}~\blacksquare}
\newcommand{\seqX}{$X_{1},\ldots,X_{n}$}
\newcommand{\seqY}{$Y_{1},\ldots,Y_{n}$}
\newcommand{\tend}[1]{\hbox{\oalign{$\bm{#1}$\crcr\hidewidth$\scriptscriptstyle\bm{\sim}$\hidewidth}}}

%\newtheorem{Teorema}{Teorema}
%\newtheorem{Proposicao}{Proposição}
%\newtheorem{Definicao}{Definição}
%\newtheorem{Corolario}{Corolário}
%\newtheorem{Demonstracao}{Demonstração}

\titlegraphic{\hspace*{8cm}\href{https://fsbmat-ufv.github.io/}{\includegraphics[width=2cm]{figs/mylogo.png}}
}

%Continuar a numeracao em slides diferentes
\newcounter{saveenumi}
\newcommand{\seti}{\setcounter{saveenumi}{\value{enumi}}}
\newcommand{\conti}{\setcounter{enumi}{\value{saveenumi}}}

\resetcounteronoverlays{saveenumi}

% Layout da pagina
\hypersetup{pdfpagelayout=SinglePage}

%Para o \pause funcionar dentro do ambiente align
\makeatletter
\let\save@measuring@true\measuring@true
\def\measuring@true{%
  \save@measuring@true
  \def\beamer@sortzero##1{\beamer@ifnextcharospec{\beamer@sortzeroread{##1}}{}}%
  \def\beamer@sortzeroread##1<##2>{}%
  \def\beamer@finalnospec{}%
}
\makeatother

\title{Inferência Estatística II}
\author{Prof. Fernando de Souza Bastos\texorpdfstring{\\ fernando.bastos@ufv.br}{}}
\institute{Departamento de Estatística\texorpdfstring{\\ Programa de Pós-Graduação em Estatística Aplicada e Biometria}\texorpdfstring{\\ Universidade Federal de Viçosa}{}\texorpdfstring{\\ Campus UFV - Viçosa}{}}
\date{}
\newcommand\mytext{Aula 13}
\newcommand\mytextt{Fernando de Souza Bastos}
\newcommand\mytexttt{\url{https://est711.github.io/}}


\begin{document}
%\SweaveOpts{concordance=TRUE}

\frame{\titlepage}

\begin{frame}{}
\frametitle{\bf Sumário}
\tableofcontents
\end{frame}

\section{Testes de Hipóteses - Caso Multiparamétrico}
\begin{frame}{}
\begin{block}{}
\justifying
\seqX~variáveis aleatórias iid com densidade $f(x,\theta), \theta\in \Omega\subset \R^{p}.$ Assumiremos todas as condições de regularidade. As hipóteses de interesse são: 
$H_{0}:\theta\in W$ contra $H_{1}:\theta\in \Omega\cap W^{c},$ em que $W\subset\Omega.$
\end{block}
\pause 
\begin{block}{}
\justifying
Assim, um teste intuitivo (baseado no teorema 6.1.1 do livro texto \citet{hogg}) é baseado na estatística de teste dada pela razão de verossimilhança,
\begin{align*}
    \Lambda=\dfrac{{\displaystyle \max_{\theta\in W}L(\theta)}}{{\displaystyle \max_{\theta\in \Omega}L(\theta)}}.
\end{align*}
Valores grandes de $\Lambda,$ ou seja, próximos de 1, sugerem que $H_{0}$ é verdadeira, enquanto que valores pequenos sugerem que $H_{1}$ é verdadeira.
\end{block}
\end{frame}

\begin{frame}{}
\begin{block}{}
\justifying
Para um nível de significância $\alpha\in(0,1)$ especificado, isto sugere a seguinte regra de decisão:
\begin{itemize}
    \item Rejeite $H_{0}$ em favor de $H_{1}$ se $\Lambda\leq c,$ em que $c$ é tal que ${\displaystyle \max_{\theta\in W}P_{\theta}(\Lambda\leq c)=\alpha}$
\end{itemize}
\end{block}
\end{frame}

\begin{frame}{Exemplo}
\vspace{-0.2cm}
\begin{block}{}
\justifying
Sejam \seqX$\Sim N(\mu,\sigma^{2}).$
\begin{align*}
H_{0}&:\mu=\mu_{0}\\
H_{1}&:\mu\neq\mu_{0},
\end{align*}
$$\theta=(\mu,\sigma^{2})^{\top},~W=\{(\mu_{0},\sigma^{2});\sigma^{2}>0\}$$
$$\Omega=\{(\mu,\sigma^{2});\mu\in\R~\text{e}~\sigma^{2}>0\}$$
\end{block}
\pause
\vspace{-0.1cm}
\begin{block}{}
\justifying
\begin{itemize}
    \item Para $\theta\in\Omega,$ já vimos que $\hat{\mu}=\Bar{X}$ e $\hat{\sigma}^{2}=\dfrac{1}{n}\Sumi (X_{i}-\Bar{X})^{2}$ são os EMV de $\mu$ e $\sigma^{2},$ respectivamente. Neste caso, 
\end{itemize}
\begin{align*}
    {\displaystyle \max_{\theta\in\Omega}L(\theta)=\dfrac{1}{(2\pi)^{\frac{n}{2}}}\dfrac{1}{(\hat{\sigma}^{2})^{\frac{n}{2}}}\exp{(-\frac{n}{2})}}
\end{align*}
\end{block}
\end{frame}

\begin{frame}{}
\begin{block}{}
\justifying
\begin{itemize}
    \item Para $\theta\in W,$ temos 
    $\hat{\sigma}_{0}^{2}=\dfrac{1}{n}\Sumi (X_{i}-\mu_{0})^{2}.$ Neste caso, 
\end{itemize}
\begin{align*}
    {\displaystyle \max_{\theta\in W}L(\theta)=\dfrac{1}{(2\pi)^{\frac{n}{2}}}\dfrac{1}{(\hat{\sigma}_{0}^{2})^{\frac{n}{2}}}\exp{(-\frac{n}{2})}}
\end{align*}
\end{block}
\end{frame}

\begin{frame}{}
\begin{block}{}
\justifying
Daí,
\begin{align*}
    \Lambda=\left(\dfrac{\hat{\sigma}^{2}}{\hat{\sigma}_{0}^{2}}\right)^{\frac{n}{2}}=\left(\dfrac{\Sumi (X_{i}-\Bar{X})^{2}}{\Sumi (X_{i}-\mu_{0})^{2}}\right)^{\frac{n}{2}}
\end{align*}
\end{block}
\pause
\begin{block}{}
\justifying
$\Lambda\leq c$ é equivalente a
\begin{align*}
    \dfrac{\Sumi (X_{i}-\mu_{0})^{2}}{\Sumi (X_{i}-\Bar{X})^{2}}\geq c^{'}\qquad (\star)
\end{align*}
\end{block}
\end{frame}

\begin{frame}{}
\begin{block}{}
\justifying
Além disso. considerem a identidade
\begin{align*}
    \Sumi (X_{i}-\mu_{0})^{2}=\Sumi (X_{i}-\Bar{X})^{2}+n(\Bar{X}-\mu_{0})^{2}\qquad (\star\star)
\end{align*}
\end{block}
\pause
\begin{block}{}
\justifying
Substituindo $(\star\star)$ em $(\star),$ segue que 
\begin{align*}
    1+\dfrac{n(\Bar{X}-\mu_{0})^{2}}{\Sumi (X_{i}-\Bar{X})^{2}}\geq c^{'}&\SeSe 
    \left(\underbrace{\dfrac{\sqrt{n}(\Bar{X}-\mu_{0})}{\sqrt{\dfrac{(X_{i}-\Bar{X})^{2}}{n-1}}}}_{T\sim t_{n-1}(\text{sob}~H_{0})}\right)^{2}\geq c^{*}\\
    &\SeSe |T|\geq c^{''}
\end{align*}
\end{block}
\end{frame}

\begin{frame}{}
\begin{block}{}
\justifying
Podemos tomar $c^{''}=t_{1-\frac{\alpha}{2}}(n-1).$
\begin{itemize}
    \item Rejeite $H_{0}$ em favor de $H_{1}$ se $|T|\geq t_{1-\frac{\alpha}{2}}(n-1)$
\end{itemize}
\end{block}
\end{frame}

\begin{frame}{}
\begin{Teorema}
\justifying
Sejam \seqX variáveis aleatórias iid com densidade $f(x,\theta),\theta\in\Omega\subset\R^{p}.$ Assuma que as condições de regularidade valem. Seja $\hat{\theta}_{n}$ uma sequência de soluções consistentes da equação de verossimilhança sob $\theta\in \Omega$ (dimensão $p$). Seja $\hat{\theta}_{0n}$ uma sequência de soluções consistentes da equação de verossimilhança sob $\theta\in W$ (dimensão $q<p$). Sob $H_{0},$
\begin{align*}
    -2\log{\Lambda}\ConvD \chi^{2}(q)
\end{align*}
\end{Teorema}
\end{frame}

\begin{frame}{\Home}
\begin{block}{}
\justifying

\begin{itemize}
    \item \textbf{Exercícios da seção 6.5:} 2 ao 5, 7 e 9 ao 11.
\end{itemize}
\end{block}
\end{frame}

\begin{frame}[allowframebreaks]
\frametitle{\bf Referências}
\printbibliography
\end{frame}


\end{document}

\begin{frame}{}
\begin{block}{}
\justifying

\end{block}
\end{frame}

\begin{frame}{}
\begin{block}{}
\justifying

\end{block}
\end{frame}

\begin{frame}{}
\begin{block}{}
\justifying

\end{block}
\end{frame}

\begin{frame}{}
\begin{block}{}
\justifying

\end{block}
\end{frame}

\begin{frame}{}
\begin{block}{}
\justifying

\end{block}
\end{frame}

\begin{frame}{}
\begin{block}{}
\justifying

\end{block}
\end{frame}

\begin{frame}{}
\begin{block}{}
\justifying

\end{block}
\end{frame}

\begin{frame}{}
\begin{block}{}
\justifying

\end{block}
\end{frame}

\begin{frame}{}
\begin{block}{}
\justifying

\end{block}
\end{frame}