\documentclass[12pt]{beamer}

\usepackage[brazil]{babel}
\usepackage[utf8]{inputenc}
\usepackage[T1]{fontenc}
\usepackage{animate}
\usepackage{amsbsy}
\usepackage{amsfonts}
\usepackage{amsmath}
\usepackage{amssymb}
\usepackage{amsthm}
\setbeamertemplate{theorems}[numbered] % to number
\usepackage[toc,page,title,titletoc]{appendix}
\usepackage{dsfont}
\usepackage{esvect}
\usepackage[labelfont=bf]{caption}
%\usepackage{subcaption}
\usepackage{float}
\usepackage[Glenn]{fncychap}%Sonny %Conny %Lenny %Glenn %Renje %Bjarne %Bjornstrup
\usepackage{graphicx}
\usepackage{subfig}
\usepackage{indentfirst}%Para indentar os paragrafos automáticamente
\usepackage{lipsum}
\usepackage{longtable}
\usepackage{mathtools}
\usepackage{listings}%Inserir codigo do R no latex
\usepackage{multirow}
\usepackage{multicol}
\usepackage{csquotes}
\usepackage[citestyle=authoryear,maxcitenames=2,terseinits=true,natbib=true, style=abnt, maxbibnames=99]{biblatex}
\addbibresource{../Referencias/Referencias.bib}
\usepackage[figuresright]{rotating}
\usepackage{spalign}
\usepackage{pgfplots}
\pgfplotsset{compat=1.17}
\usepackage{tikz}
\usepackage{fontawesome}
\usepackage{color, colortbl}
\usepackage{url}
\usepackage{cancel}
\usepackage{accents}
\usepackage{bm}
\usepackage{ragged2e}%para justificar o texto dentro de algum ambiente
\definecolor{Gray}{gray}{0.9}
\definecolor{LightCyan}{rgb}{0.88,1,1}


\usepackage[all]{xy}
\usepackage{hyperref,bookmark}
\hypersetup{
  colorlinks=true,
  linkcolor=blue,
  citecolor=red,
  filecolor=blue,
  urlcolor=blue,
}

\usetheme{Madrid}
\usecolortheme[RGB={193,0,0}]{structure}

%\setbeamertemplate{footline}[frame number]
%\setbeamertemplate{footline}[text line]{%
%  \parbox{\linewidth}{\vspace*{-8pt}\hfill\date{}\hfill\insertshortauthor\hfill\insertpagenumber}}
\beamertemplatenavigationsymbolsempty
\renewcommand{\vec}[1]{\mbox{\boldmath$#1$}}
\newtheorem{Teorema}{Teorema}
\newtheorem{Proposicao}{Proposição}
\newtheorem{definicao}{Definição}
\newtheorem{Corolario}{Corolário}
\newtheorem{Demonstracao}{Demonstração}
\newcommand{\bx}{\ensuremath{\bar{x}}}
\newcommand{\Ho}{\ensuremath{H_{0}}}
\newcommand{\Hi}{\ensuremath{H_{1}}}
\newcommand{\at}[2][]{#1|_{#2}}
\newcommand\xuparrow[1][2ex]{%
   \mathrel{\rotatebox{-90}{$\xleftarrow{\rule{#1}{0pt}}$}}
}
\apptocmd{\frame}{}{\justifying}{} % Allow optional arguments after frame.

\makeatletter
\setbeamertemplate{footline}
{
  \leavevmode%
  \hbox{%
  \begin{beamercolorbox}[wd=.3\paperwidth,ht=2.25ex,dp=1ex,center]{author in head/foot}%
    \usebeamerfont{author in head/foot}\mytext
  \end{beamercolorbox}%
  \begin{beamercolorbox}[wd=.3\paperwidth,ht=2.25ex,dp=1ex,center]{title in head/foot}%
    \usebeamerfont{title in head/foot}\mytextt
  \end{beamercolorbox}%
  \begin{beamercolorbox}[wd=.35\paperwidth,ht=2.25ex,dp=1ex,right]{site in head/foot}%
    \usebeamerfont{site in head/foot}\mytexttt\hspace*{2em}
    \insertframenumber{} / \inserttotalframenumber\hspace*{2ex} 
  \end{beamercolorbox}}%
  \vskip0pt%
}
\makeatother

\providecommand{\arcsin}{} \renewcommand{\arcsin}{\hspace{2pt}\textrm{arcsen}}
\providecommand{\sin}{} \renewcommand{\sin}{\hspace{2pt}\textrm{sen}}
\newcommand{\N}{\rm I\!N}
\newcommand{\I}{\rm I\!I}
\newcommand{\R}{\rm I\!R}
\newcommand{\Sim}{\overset{\text{iid}}{\sim}}
\newcommand{\Lim}{{\displaystyle \lim_{n\to\infty}}}
\newcommand{\LimInf}{{\displaystyle \liminf_{n\to\infty}}}
\newcommand{\rightLim}{\xrightarrow[n\rightarrow\infty]{}}
\newcommand{\Sumi}{{\displaystyle \sum_{i=1}^{n}}}
\newcommand{\Int}{{\displaystyle \int_{-\infty}^{+\infty}}}
\newcommand{\ConvD}{\overset{D}{\rightarrow}}
\newcommand{\ConvP}{\overset{P}{\rightarrow}}
\newcommand{\Prodi}{{\displaystyle \prod_{i=1}^{n}}}
\newcommand{\SetaUP}[2]{\underset{\mathclap{\substack{\xuparrow[30pt] \\ #1}}}{#2}}
%\newcommand{\SetaInclinada}[2]{\underset{\mathclap{\substack{\rotatebox{135}{\xuparrow[30pt] \\ #1}}}}{#2}}
\newcommand{\Home}{\begin{tikzpicture}
\node[scale=2] at (3,4) {\text{Para}~\faHome};
\end{tikzpicture}}
\newcommand{\vecX}{\boldsymbol{X}}
\newcommand{\Implica}[1]{\xRightarrow{#1}}
\newcommand{\SeSe}{\iff}
\newcommand{\EscoreA}{\dfrac{\partial}{\partial\theta}\log{f(x,\theta)}}
\newcommand{\EscoreB}{\dfrac{\partial^{2}}{\partial\theta^{2}}\log{f(x,\theta)}}
\newcommand{\cqd}{\text{cqd}~\blacksquare}
\newcommand{\seqX}{$X_{1},\ldots,X_{n}$}
\newcommand{\seqY}{$Y_{1},\ldots,Y_{n}$}
\newcommand{\tend}[1]{\hbox{\oalign{$\bm{#1}$\crcr\hidewidth$\scriptscriptstyle\bm{\sim}$\hidewidth}}}

%\newtheorem{Teorema}{Teorema}
%\newtheorem{Proposicao}{Proposição}
%\newtheorem{Definicao}{Definição}
%\newtheorem{Corolario}{Corolário}
%\newtheorem{Demonstracao}{Demonstração}

\titlegraphic{\hspace*{8cm}\href{https://fsbmat-ufv.github.io/}{\includegraphics[width=2cm]{figs/mylogo.png}}
}

%Continuar a numeracao em slides diferentes
\newcounter{saveenumi}
\newcommand{\seti}{\setcounter{saveenumi}{\value{enumi}}}
\newcommand{\conti}{\setcounter{enumi}{\value{saveenumi}}}

\resetcounteronoverlays{saveenumi}

% Layout da pagina
\hypersetup{pdfpagelayout=SinglePage}

%Para o \pause funcionar dentro do ambiente align
\makeatletter
\let\save@measuring@true\measuring@true
\def\measuring@true{%
  \save@measuring@true
  \def\beamer@sortzero##1{\beamer@ifnextcharospec{\beamer@sortzeroread{##1}}{}}%
  \def\beamer@sortzeroread##1<##2>{}%
  \def\beamer@finalnospec{}%
}
\makeatother

\title{Inferência Estatística II}
\author{Prof. Fernando de Souza Bastos\texorpdfstring{\\ fernando.bastos@ufv.br}{}}
\institute{Departamento de Estatística\texorpdfstring{\\ Programa de Pós-Graduação em Estatística Aplicada e Biometria}\texorpdfstring{\\ Universidade Federal de Viçosa}{}\texorpdfstring{\\ Campus UFV - Viçosa}{}}
\date{}
\newcommand\mytext{Aula 12}
\newcommand\mytextt{Fernando de Souza Bastos}
\newcommand\mytexttt{\url{https://est711.github.io/}}


\begin{document}
%\SweaveOpts{concordance=TRUE}

\frame{\titlepage}

\begin{frame}{}
\frametitle{\bf Sumário}
\tableofcontents
\end{frame}

\section{Exemplo 1: Teste Bilateral para a Média Baseado em Grandes Amostras}
\begin{frame}{Teste Bilateral para a Média Baseado em Grandes Amostras}
\begin{block}{}
\justifying
Considere $X$ uma variável aleatória com média $\mu$ e variância finita $\sigma^2$. Queremos testar

\begin{align}\label{t1}
H_0 : \mu = \mu_0 \quad \text{contra} \quad H_1 : \mu \neq \mu_0
\end{align}

onde $\mu_0$ é especificado. Sejam $X_1, \ldots, X_n$ uma amostra aleatória da distribuição de $X$ e denotem a média e a variância da amostra por $\bar{X}$ e $S^2$, respectivamente.
\end{block}
\end{frame}

\begin{frame}{}
\begin{block}{}
\justifying
Para o teste unilateral, rejeitamos $H_0$ se $\bar{X}$ for muito grande. Portanto, para as hipóteses (\ref{t1}), usamos a regra de decisão

\begin{align}\label{t2}
\text{Rejeitar } H_0 \text{ em favor de } H_1 \text{ se } \bar{X} \leq h \text{ ou } \bar{X} \geq k
\end{align}

onde $h$ e $k$ são tais que $\alpha = P_{H_0} [\bar{X} \leq h \text{ ou } \bar{X} \geq k]$. Claramente, $h < k$; portanto, temos

\begin{align}
\alpha = P_{H_0} [\bar{X} \leq h \text{ ou } \bar{X} \geq k] = P_{H_0} [\bar{X} \leq h] + P_{H_0} [\bar{X} \geq k].
\end{align}
\end{block}
\end{frame}

\begin{frame}{}
\begin{block}{}
\justifying
Uma vez que, pelo menos para amostras grandes, a distribuição de $X$ é simétrica em torno de $\mu_0$, sob $H_0$, uma regra intuitiva é dividir $\alpha$ igualmente entre os dois termos do lado direito da expressão acima; isto é, $h$ e $k$ são escolhidos de forma que

\begin{align}\label{t3}
P_{H_0} [\bar{X} \leq h] = \frac{\alpha}{2} \quad \text{e} \quad P_{H_0} [\bar{X} \geq k] = \frac{\alpha}{2}.
\end{align}

Sabemos que, para amostras grandes, $(\bar{X} - \mu_0) / (S / \sqrt{n})$ é aproximadamente $N(0, 1)$. Isso e (\ref{t3}) levam à regra de decisão aproximada

\begin{align}\label{t4}
\text{Rejeitar } H_0 \text{ em favor de } H_1 \text{ se } \frac{\bar{X} - \mu_0}{S / \sqrt{n}} \geq z_{\alpha/2}.
\end{align}
\end{block}
\end{frame}

\begin{frame}{}
\begin{block}{}
\justifying
Substituindo $S$ por $\sigma$, segue facilmente que a função poder aproximada é
\begin{align*}\label{t5}
\gamma(\mu) &= P_{\mu}(\bar{X} \leq \mu_0 - z_{\alpha/2} \sigma / \sqrt{n}) + P_{\mu}(\bar{X} \geq \mu_0 + z_{\alpha/2} \sigma / \sqrt{n})\\
&= \Phi\left(\dfrac{\sqrt{n}(\mu_0 - \mu)}{\sigma} - z_{\alpha/2}\right) + 1 - \Phi\left(\dfrac{\sqrt{n}(\mu_0 - \mu)}{\sigma} + z_{\alpha/2}\right), 
\end{align*}

em que $\Phi(z)$ é a função de distribuição acumulada de uma variável aleatória normal padrão. Observe que a derivada da função poder é

\begin{align*}
\gamma'(\mu) = \dfrac{\sqrt{n}}{\sigma} \left(\phi\left(\dfrac{\sqrt{n}(\mu_0 - \mu)}{\sigma} + z_{\alpha/2}\right) - \phi\left(\dfrac{\sqrt{n}(\mu_0 - \mu)}{\sigma} - z_{\alpha/2}\right)\right),
\end{align*}

em que $\phi(z)$ é a função de densidade de probabilidade de uma variável aleatória normal padrão. 
\end{block}
\end{frame}

\section{Exemplo 2: Exercício 4.6.2}
\begin{frame}{Exemplo 2: Exercício 4.6.2}
\begin{block}{}
\justifying
Considere $a=\dfrac{\sqrt{n}(\mu_0 - \mu)}{\sigma}$ e notem que,
\begin{itemize}
    \item Se $\mu<\mu_{0},$ então $a>0;$
    \pause
    \item Se $\mu>\mu_{0},$ então $a<0;$
\end{itemize}
\end{block}
\pause 
\begin{block}{}
\justifying
Podemos reescrever então a derivada da função poder como
\begin{align*}
\gamma'(\mu) = \dfrac{\sqrt{n}}{\sigma} \left(\phi\left( z_{\alpha/2}+a\right) - \phi\left(z_{\alpha/2}-a\right)\right),
\end{align*}
uma vez que $\phi(x)=\phi(-x).$ 
\end{block}
\end{frame}

\begin{frame}{}
\begin{block}{Suponha $\mu<\mu_{0}$}
\justifying
Nesse caso, 
\begin{align*}
    z_{\alpha/2}+a>z_{\alpha/2}-a&\Rightarrow -\dfrac{(z_{\alpha/2}+a)^{2}}{2}<-\dfrac{(z_{\alpha/2}-a)^{2}}{2}\\
    &\Rightarrow e^{-\dfrac{(z_{\alpha/2}+a)^{2}}{2}} < e^{-\dfrac{(z_{\alpha/2}-a)^{2}}{2}}\\
    &\Rightarrow \dfrac{\sqrt{n}}{\sigma\sqrt{2\pi}}\left[e^{-\dfrac{(z_{\alpha/2}+a)^{2}}{2}}-e^{-\dfrac{(z_{\alpha/2}-a)^{2}}{2}}\right]<0\\
    &\Rightarrow \gamma'(\mu)<0
\end{align*}
\end{block}
\end{frame}


\begin{frame}{}
\begin{block}{Suponha $\mu>\mu_{0}$}
\justifying
Nesse caso, 
\begin{align*}
    z_{\alpha/2}+a<z_{\alpha/2}-a&\Rightarrow -\dfrac{(z_{\alpha/2}+a)^{2}}{2}>-\dfrac{(z_{\alpha/2}-a)^{2}}{2}\\
    &\Rightarrow e^{-\dfrac{(z_{\alpha/2}+a)^{2}}{2}} > e^{-\dfrac{(z_{\alpha/2}-a)^{2}}{2}}\\
    &\Rightarrow \dfrac{\sqrt{n}}{\sigma\sqrt{2\pi}}\left[e^{-\dfrac{(z_{\alpha/2}+a)^{2}}{2}}-e^{-\dfrac{(z_{\alpha/2}-a)^{2}}{2}}\right]>0\\
    &\Rightarrow \gamma'(\mu)>0
\end{align*}
\end{block}
\end{frame}

\section{Exemplo 3}
\begin{frame}{}
\begin{block}{}
\justifying
Suponha que desejamos testar
\begin{align}
H_0 : \mu = 30,000 \text{ versus } H_1 : \mu \neq 30,000. 
\end{align}

Suponha que $n = 20$ e $\alpha = 0.01$. Então, a regra de rejeição se torna

\begin{align}
\text{Rejeitar } H_0 \text{ em favor de } H_1 \text{ se } \frac{\Bar{X} - 30,000}{S/\sqrt{20}} \geq z_{\frac{0.01}{2}}.
\end{align}

A próxima Figura exibe a curva da função poder para este teste quando $S$ é substituído por $\sigma = 5000$. Para comparação, a curva da função poder para o teste com nível $\alpha = 0.05$ também é apresentada. 
\end{block}
\end{frame}


\begin{frame}{}
\begin{block}{}
\justifying
\begin{figure}
    \centering
    \includegraphics[scale=0.6]{figs/FunctionPower.png}
    \caption{Função Poder para o teste de hipótese do exemplo}
    \label{fig:enter-label}
\end{figure}
\end{block}
\end{frame}

\begin{frame}{}
\begin{block}{}
\justifying
Se assumirmos que $X$ tem uma distribuição normal, então, o seguinte teste tem tamanho exato $\alpha$ para testar $H_0 : \mu = \mu_0$ versus $H_1 : \mu \neq \mu_0$:

\begin{align}
\text{Rejeitar } H_0 \text{ em favor de } H_1 \text{ se } \frac{\Bar{X} - \mu_0}{S/\sqrt{n}} \geq t_{\alpha/2, n-1}. 
\end{align}

Ele também possui uma curva da função poder em forma de ``bacia'' semelhante à Figura anterior, embora não seja tão fácil de mostrar; veja Lehmann (1986).
\end{block}
\end{frame}

\section{Outros Exemplos de Cálculo da função poder}
\begin{frame}{}
\begin{block}{Distribuição Normal:}
\justifying

\begin{itemize}
\item \textbf{Hipótese Nula (H0):} A média de uma população é igual a 100.
\item \textbf{Hipótese Alternativa (H1):} A média de uma população é maior que 100.
\item \textbf{Tamanho da amostra:} 30
\item \textbf{Desvio padrão populacional conhecido:} 15
\item \textbf{Nível de significância:} 0,05
\item \textbf{Média real sob H1 (Suposição):} 105
\end{itemize}

Para calcular a função poder, usamos a distribuição normal padrão ($Z$) e a fórmula:

\[ \text{Poder} = P(Z > Z_{\alpha} - \frac{\mu - \mu_0}{\sigma/\sqrt{n}}) \]

em que $Z_{\alpha}$ é o valor crítico para o nível de significância $\alpha$.
\end{block}
\end{frame}

\begin{frame}{}
\begin{block}{}
\justifying
Para $\alpha = 0,05$, $Z_{0,05} \approx 1,645.$

Agora, substituindo os valores:
\begin{align*}
    \text{Poder} &= P(Z > 1,645 - \frac{105 - 100}{15/\sqrt{30}})\\
    &= P(Z > 1,645 - 2,738)\\
    &= P(Z > -1,093)
\end{align*}

A probabilidade de $Z$ ser maior que $-1,093$ é aproximadamente $0,8628$. Portanto, o poder do teste é de aproximadamente $0,8628$.
\end{block}
\end{frame}

\begin{frame}{}
\begin{block}{Distribuição Binomial}
\justifying
\begin{itemize}
\item \textbf{Hipótese Nula (H0):} A proporção de sucesso em um experimento binomial é de 0,4.
\item \textbf{Hipótese Alternativa (H1):} A proporção de sucesso em um experimento binomial é maior que 0,4.
\item \textbf{Tamanho da amostra:} 100
\item \textbf{Nível de significância:} 0,01
\item \textbf{Proporção real sob H1 (Suposição):} 0,55
\end{itemize}

\[ \text{Poder} = 1 - P(X \leq x) \]

Onde $x$ é o valor crítico que corresponde ao nível de significância $\alpha$. Para $\alpha = 0,01$, $x$ é aproximadamente 52.
\end{block}
\end{frame}

\begin{frame}{}
\begin{block}{}
\justifying
Agora, substituindo os valores:

\[ \text{Poder} = 1 - P(X \leq 52) \]

Usando uma calculadora ou software estatístico, encontramos $P(X \leq 52) \approx 0,307$. Portanto, o poder do teste é aproximadamente $1 - 0,307 = 0,693$.
\end{block}
\end{frame}

\begin{frame}{}
\begin{block}{}
\justifying
\begin{figure}
    \centering
    \includegraphics[scale=0.7]{figs/binomPower.png}
    \caption{Função Poder de um teste unilateral usando a distribuição binomial}
    \label{fig:enter-label}
\end{figure}
\end{block}
\end{frame}

\section{Relação entre Testes de Hipóteses e IC}
\begin{frame}{Relação entre Testes de Hipóteses e IC}
\begin{block}{}
\justifying
Existe uma relação entre testes bilaterais e intervalos de confiança. Considere o teste $t$ bilateral. Aqui, usamos a regra de rejeição com ``se e somente se'' substituindo ``se''. Portanto, em termos de aceitação, temos Aceitar $H_0,$ se, e somente se,

\begin{align*}
\mu_0 - t_{\alpha/2, n-1}S/\sqrt{n} < \Bar{X} < \mu_0 + t_{\alpha/2, n-1}S/\sqrt{n}.
\end{align*}
\end{block}
\pause
\begin{block}{}
Isso pode ser facilmente demonstrado como ``Aceitar $H_0$ se, e somente se'',
\begin{align*}
\mu_0 \in \left(\Bar{X} - t_{\alpha/2, n-1}\frac{S}{\sqrt{n}}, \Bar{X} + t_{\alpha/2, n-1}\frac{S}{\sqrt{n}}\right).
\end{align*}
\end{block}
\end{frame}

\begin{frame}{}
\begin{block}{}
\justifying
Ou seja, aceitamos $H_0$ ao nível de significância $\alpha$ se e somente se $\mu_0$ está no intervalo de confiança de $(1 - \alpha)100\%$ para $\mu$. De forma equivalente, rejeitamos $H_0$ ao nível de significância $\alpha$ se, e somente se, $\mu_0$ não está no intervalo de confiança de $(1 - \alpha)100\%$ para $\mu$. Isso é válido para todos os testes de hipóteses bilaterais.
\end{block}
\end{frame}

\section{Exemplo 3}
\begin{frame}{Exemplo 3}
\vspace{-0.2cm}
\begin{block}{}
\justifying
Considere amostras aleatórias independentes de $N(\mu_1, \sigma^2)$ e $N(\mu_2, \sigma^2)$, respectivamente. Definimos $n = n_1 + n_2$ como o tamanho combinado da amostra e $S_{p}^2$ como o estimador combinado da variância comum, dado por
\begin{align*}
S_{p}^2 = \frac{(n_1 - 1)S_1^2 + (n_2 - 1)S_2^2}{n - 2}.
\end{align*}
A um nível de significância $\alpha = 0.05$, rejeitamos $H_0 : \mu_1 = \mu_2$ em favor da alternativa unilateral $H_1 : \mu_1 > \mu_2$ se
\begin{align*}
T = \frac{\bar{X} - \bar{Y} - 0}{S_p\sqrt{\frac{1}{n_1} + \frac{1}{n_2}}}  \geq t_{0.05, n-2},
\end{align*}
pois, sob $H_0$, $T$ segue uma distribuição t com $n - 2$ graus de liberdade.
\end{block}
\end{frame}

\section{Exemplo 4 - Distribuição Binomial}
\begin{frame}{Exemplo 4}
\begin{block}{}
\justifying
Suponha que $X$ segue uma distribuição binomial com parâmetros $1$ e $p$. Considere o teste de hipótese $H_0 : p = p_0$ contra $H_1 : p < p_0$. Seja $X_1, \ldots, X_n$ uma amostra aleatória da distribuição de $X$, e seja $\hat{p} = \frac{X}{n}$. Para testar $H_0$ versus $H_1$, utilizamos uma das seguintes estatísticas:

\begin{align*}
Z_1 = \frac{\hat{p} - p_0}{\sqrt{p_0(1 - p_0)/n}} \leq c~\text{ou}~Z_2 = \frac{\hat{p}-p_0}{\sqrt{\hat{p}(1 - \hat{p})/n}} \leq c.
\end{align*}

Se o tamanho da amostra $n$ for grande, tanto $Z_1$ quanto $Z_2$ têm distribuições normais aproximadas, desde que $H_0 : p = p_0$ seja verdadeira. Portanto, se $c$ for definido como $-1.645$, o nível de significância aproximado é $\alpha = 0.05$. Ambos os métodos fornecem resultados numéricos semelhantes.
\end{block}
\end{frame}


\begin{frame}{}
\begin{block}{}
\justifying
O uso de $Z_1$ fornece melhores probabilidades para cálculos de poder se o verdadeiro valor de $p$ estiver próximo de $p_0$, enquanto $Z_2$ é melhor quando $H_0$ for claramente falsa. No entanto, com uma hipótese alternativa bilateral, $Z_2$ fornece uma melhor relação com o intervalo de confiança para $p$. Ou seja, $|Z_2| < z_{\alpha/2}$ é equivalente a $p_0$ estar no intervalo

\begin{align*}
\left(\hat{p} - z_{\alpha/2}\sqrt{\frac{\hat{p}(1 - \hat{p})}{n}}, \hat{p} + z_{\alpha/2}\sqrt{\frac{\hat{p}(1 - \hat{p})}{n}}\right),
\end{align*}

que é o intervalo que fornece um intervalo de confiança aproximado de $(1 - \alpha)100\%$ para $p$, conforme discutido na aula de Intervalos de Confiança.
\end{block}
\end{frame}

\section{Exemplo 5 - Distribuição Poisson}
\begin{frame}{Exemplo 5 - Distribuição Poisson}
\begin{block}{}
\justifying
Seja $X_1, X_2, \ldots, X_{10}$ uma amostra aleatória de tamanho $n = 10$ de uma distribuição de Poisson com média $\theta$. A região crítica para testar $H_0 : \theta = 0.1$ contra $H_1 : \theta > 0.1$ é dada por $Y = {\displaystyle \sum_{i=1}^{10}} X_i \geq 3$. A estatística $Y$ segue uma distribuição de Poisson com média $10\theta$. Portanto, com $\theta = 0.1$, de modo que a média de $Y$ seja igual a 1, o nível de significância do teste é

\begin{align*}
P(Y \geq 3) = 1 - P(Y \leq 2) = 1 - 0.920 = 0.080.
\end{align*}
\end{block}
\end{frame}

\begin{frame}{}
\begin{block}{}
\justifying
Por outro lado, se a região crítica definida por $\sum_{i=1}^{10} X_i \geq 4$ for usada, o nível de significância é

\begin{align*}
\alpha = P(Y \geq 4) = 1 - P(Y \leq 3) = 1 - 0.981 = 0.019.
\end{align*}

Por exemplo, se um nível de significância de aproximadamente $\alpha = 0.05$ for desejado, a maioria dos estatísticos usaria um desses testes, ou seja, eles ajustariam o nível de significância para o teste mais conveniente. 
\end{block}
\end{frame}

\begin{frame}{}
\begin{block}{}
\justifying
No entanto, um nível de significância de $\alpha = 0.05$ pode ser alcançado da seguinte maneira. Seja $W$ uma variável aleatória com distribuição de Bernoulli com probabilidade de sucesso igual a

\begin{align*}
P(W = 1) = \frac{\alpha-0.019}{0.08-0.019}=\frac{0.050 - 0.019}{0.080 - 0.019} = \frac{31}{61}.
\end{align*}

Suponha que $W$ seja selecionado independentemente da amostra. Considere a regra de rejeição: rejeitar $H_0$ se $\sum_{i=1}^{10} X_i \geq 4$ ou se $\sum_{i=1}^{10} X_i = 3$ e $W = 1$. O nível de significância dessa regra é
{\small 
\begin{align*}
P_{H_0}(Y \geq 4) + P_{H_0}(\{Y = 3\} \cap \{W = 1\})&= P_{H_0}(Y \geq 4)\\ + P_{H_0}(Y = 3)P(W = 1)\\ &= 0.019 + 0.061 \times \frac{31}{61}\\ &= 0.05.
\end{align*}}
\end{block}
\end{frame}

\begin{frame}{}
\begin{block}{}
\justifying
Portanto, a regra de decisão tem exatamente um nível de 0.05. O processo de realizar o experimento auxiliar para decidir se rejeita ou não quando $Y = 3$ é às vezes referido como um teste randomizado.
\end{block}
\end{frame}

\section{Nível de Significância Observado (p-valor)}
\begin{frame}{Nível de Significância Observado (p-valor)}
\begin{block}{}
\justifying
Muitos estatísticos não gostam de testes randomizados na prática, pois o uso deles implica que dois estatísticos podem fazer as mesmas suposições, observar os mesmos dados, aplicar o mesmo teste e, no entanto, tomar decisões diferentes. Portanto, eles costumam ajustar seu nível de significância para evitar a aleatoriedade. Na verdade, muitos estatísticos relatam o que são comumente chamados de níveis de significância observados ou valores-p (para valores de probabilidade).

Um exemplo geral é suficiente para explicar os níveis de significância observados.  
\end{block}
\end{frame}

\begin{frame}{}
\begin{block}{}
\justifying
Seja $X_1, \ldots, X_n$ uma amostra aleatória de uma distribuição $N(\mu, \sigma^2)$, em que tanto $\mu$ quanto $\sigma^2$ são desconhecidos. Considere, primeiro, as hipóteses unilaterais $H_0 : \mu = \mu_0$ versus $H_1 : \mu > \mu_0$, em que $\mu_0$ é especificado. Escreva a regra de rejeição como

\begin{align}
\text{Rejeitar } H_0 \text{ em favor de } H_1, \text{ se } \Bar{X} \geq k, 
\end{align}
em que $\Bar{X}$ é a média da amostra.
\end{block}
\pause
\begin{block}{}
\justifying
Anteriormente, especificamos um nível e, em seguida, resolvemos para $k$. Na prática, no entanto, o nível não é especificado. Em vez disso, uma vez que a amostra é observada, o valor realizado $\Bar{x}$ de $\Bar{X}$ é calculado e fazemos a pergunta: O valor $\Bar{x}$ é suficientemente grande para rejeitar $H_0$ em favor de $H_1$? 
\end{block}
\end{frame}

\begin{frame}{}
\begin{block}{}
\justifying
Para responder a isso, calculamos o valor-p, que é a probabilidade,

\begin{align*}
\text{valor-p} = P(H_0 (\Bar{X} \geq \Bar{x})).
\end{align*}

Observe que este é um ``nível de significância'' baseado nos dados, e chamamos isso de nível de significância observado ou valor-p. 
\end{block}
\end{frame}

\begin{frame}{}
\begin{block}{}
\justifying
A hipótese $H_0$ é rejeitada em todos os níveis maiores ou iguais ao valor-p. Por exemplo, se o valor-p for $0,048$ e o nível nominal $\alpha$ for $0,05$, então $H_0$ será rejeitada; no entanto, se o nível nominal $\alpha$ for $0,01$, então $H_0$ não será rejeitada. Em resumo, o experimentador define as hipóteses; o estatístico seleciona a estatística de teste e a regra de rejeição; os dados são observados e o estatístico relata o valor-p para o experimentador; e o experimentador decide se o valor-p é suficientemente pequeno para justificar a rejeição de $H_0$ em favor de $H_1$. O próximo exemplo fornece uma ilustração numérica.
\end{block}
\end{frame}

\section{Exemplo 6 (Valor - p)}
\begin{frame}{Exemplo 6 (Valor - p)}
\begin{block}{}
\justifying
Considere os dados de Darwin do Exemplo 4.5.5 do livro do Hogg (Edição 8). Os dados são um design emparelhado sobre as alturas de plantas de Zea mays cruzadas e autofertilizadas. Em cada um dos 15 vasos, uma planta cruzada e uma autofertilizada foram cultivadas. Os dados de interesse são as 15 diferenças emparelhadas, (cruzada - autofertilizada). Como no Exemplo, deixe $X_i$ denotar a diferença emparelhada para o i-ésimo vaso. Deixe $\mu$ ser a verdadeira diferença média. As hipóteses de interesse são $H_0 : \mu = 0$ versus $H_1 : \mu > 0$. A regra de rejeição padronizada é

\begin{align}
\text{Rejeitar } H_0 \text{ em favor de } H_1 \text{ se } T \geq k,
\end{align}

em que $T = \frac{\Bar{X}}{S/\sqrt{15}}$, $\Bar{X}$ e $S$ são, respectivamente, a média amostral e o desvio padrão das diferenças. 
\end{block}
\end{frame}

\begin{frame}{}
\begin{block}{}
\justifying
A hipótese alternativa afirma que, em média, as plantas cruzadas são mais altas do que as plantas autofertilizadas. Do Exemplo 4.5.5, a estatística do teste t tem o valor 2,15. Deixando $t(14)$ denotar uma variável aleatória com a distribuição t com 14 graus de liberdade e usando R, o valor-p para o experimento é

\begin{align*}
P[t(14) > 2.15] = 1 - pt(2.15, 14) = 1 - 0.9752 = 0.0248. 
\end{align*}

Na prática, com esse valor-p, $H_0$ seria rejeitada em todos os níveis maiores ou iguais a 0,0248. .

Suponha que as hipóteses sejam $H_0 : \mu = \mu_0$ versus $H_1 : \mu < \mu_0$. Obviamente, o valor-p observado neste caso é

\begin{align*}
\text{valor-p} = P(H_0 (\Bar{X} \leq \Bar{x})).
\end{align*}
\end{block}
\end{frame}

\begin{frame}{}
\begin{block}{}
\justifying
Para a hipótese bilateral $H_0 : \mu = \mu_0$ versus $H_1 : \mu \neq \mu_0$, nossa regra de rejeição ``não especificada'' é

\begin{align}
\text{Rejeitar } H_0 \text{ em favor de } H_1 \text{ se } \Bar{X} \leq l \text{ ou } \Bar{X} \geq k.
\end{align}

Para o valor-p, calculamos cada um dos valores-p de um lado, pegamos o menor valor-p e o dobramos. Como ilustração, no exemplo de Darwin, suponha que as hipóteses sejam $H_0 : \mu = 0$ versus $H_1 : \mu \neq 0$. Então, o valor-p é $2 \times (0,0248) = 0,0496$. 
\end{block}
\end{frame}

\begin{frame}{}
\begin{block}{}
\justifying
Como nota final sobre valores-p para hipóteses bilaterais, suponha que a estatística de teste possa ser expressa em termos de uma estatística de teste t. Nesse caso, o valor-p pode ser encontrado de forma equivalente da seguinte maneira. Se $d$ for o valor realizado da estatística de teste t, então o valor-p é

\begin{align}
\text{valor-p} = P(H_0 [|t| \geq |d|]), 
\end{align}

em que, sob $H_0$, $t$ tem uma distribuição $t$ com $n - 1$ graus de liberdade.
\end{block}
\pause
\begin{block}{}
\justifying
Nessa discussão sobre valores-p, lembre-se de que a boa ciência dita que as hipóteses devem ser conhecidas antes que os dados sejam coletados.
\end{block}
\end{frame}


\begin{frame}{}
\begin{block}{Explicação Detalhada:}
\justifying
O valor-p é uma medida estatística que ajuda a avaliar a força das evidências contra uma hipótese nula em um teste de hipóteses. Ele é calculado com base nos dados observados e na distribuição da estatística de teste sob a suposição de que a hipótese nula é verdadeira. 
\end{block}
\pause
\begin{block}{}
\justifying
Ele não é estritamente uma probabilidade no sentido tradicional. Embora seja comumente interpretado como uma probabilidade, é uma medida de evidência estatística em vez de uma probabilidade direta de um evento.
\end{block}
\pause
\begin{block}{}
\justifying
O valor-p é, portanto, uma medida de quão consistentes os dados observados são com a hipótese nula. É a probabilidade de obter resultados tão extremos quanto os observados, assumindo que a hipótese nula seja verdadeira.
\end{block}
\end{frame}

\begin{frame}{}
\begin{block}{Aqui estão as etapas para entender o valor-p:}

\begin{itemize}
\justifying
\item Formulação das Hipóteses:

A primeira etapa é estabelecer duas hipóteses: a hipótese nula ($H_0$) e a hipótese alternativa ($H_1$). \pause

\item Coleta de Dados:

Em seguida, você coleta dados relevantes para o teste de hipóteses.\pause

\item Cálculo da Estatística de Teste:

Você calcula uma estatística de teste com base nos dados. A escolha da estatística depende do tipo de teste e da pergunta de pesquisa.
\end{itemize}
\end{block}
\end{frame}

\begin{frame}{}
\begin{block}{}
\justifying
\begin{itemize}
\justifying
\item Determinação do Valor-p 

Com a estatística de teste em mãos, você calcula o Valor-p. Esse cálculo envolve a probabilidade de obter uma estatística de teste tão extrema quanto a observada, supondo que a hipótese nula seja verdadeira. Essa probabilidade é chamada de Valor-p.\pause

\item Interpretação do Valor-p 

O valor $p$ é interpretado comparando-o a um nível de significância pré-definido, geralmente denotado como $\alpha$. Se o valor $p$ for menor ou igual a $\alpha$, é comum rejeitar a hipótese nula em favor da hipótese alternativa. Quanto menor o valor $p,$ mais fortes são as evidências contra a hipótese nula.
\end{itemize}
\end{block}
\end{frame}

\section{Exemplo 7 sobre Valor-p}
\begin{frame}{Exemplo 7}
\begin{block}{}
\justifying

Suponha que um fabricante afirma que a média de vida útil de suas lâmpadas é de $1000$ horas. Você suspeita que as lâmpadas, na verdade, têm uma vida útil média diferente.

\begin{itemize}
    \item Hipóteses
\end{itemize}

\begin{align*}
H_0 &: \mu = 1000~\text{horas} \\
H_1 &: \mu \neq 1000~\text{horas}
\end{align*}
\pause
\begin{itemize}
    \item Coleta de Dados
\end{itemize}

Você coleta uma amostra de $30$ lâmpadas e calcula a média da vida útil das lâmpadas da amostra. Suponha que você obtém uma média amostral de $980$ horas.

\end{block}
\end{frame}

\begin{frame}{}
\begin{block}{}
\justifying
\begin{itemize}
    \item Cálculo da Estatística de Teste
\end{itemize}

Para este exemplo, a estatística de teste é o valor da média amostral subtraído do valor hipotético ($1000$) dividido pelo erro padrão. Neste caso:

\[
\text{Estatística de Teste} = \dfrac{980 - 1000}{\text{erro padrão}}
\]
\pause
\begin{itemize}
    \item Determinação do Valor-p
\end{itemize}


Você calcula o Valor-p, que é a probabilidade de obter uma estatística de teste tão extrema quanto a observada (no caso, $-20$) se a média real fosse $1000$ horas, sob a suposição da distribuição de médias amostrais.

\end{block}
\end{frame}

\begin{frame}{}
\begin{block}{}
\justifying
\begin{itemize}
    \item Interpretação do Valor-p
\end{itemize}

Suponhamos que o Valor-p calculado seja $0,03$. Isso significa que, se a hipótese nula ($\mu = 1000$) fosse verdadeira, haveria uma probabilidade de $3\%$ de obter uma média amostral tão extrema quanto $980$ horas. Como esse Valor-p é menor que um nível de significância ($\alpha$) típico de $0,05$, você pode optar por rejeitar a hipótese nula. Isso sugere que as lâmpadas podem ter uma vida útil média diferente de $1000$ horas.
\end{block}
\pause
\begin{block}{}
\justifying
Lembre-se de que o Valor-p não fornece a magnitude da diferença; ele simplesmente indica se as evidências observadas são consistentes com a hipótese nula. A interpretação do Valor-p deve ser feita em conjunto com o contexto e a pergunta de pesquisa. Quanto menor o Valor-p, mais forte é a evidência contra a hipótese nula.
\end{block}
\end{frame}

\begin{frame}[allowframebreaks]
\frametitle{\bf Referências}
\printbibliography
\end{frame}


\end{document}
