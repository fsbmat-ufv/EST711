\documentclass[12pt]{beamer}

\usepackage[brazil]{babel}
\usepackage[utf8]{inputenc}
\usepackage[T1]{fontenc}
\usepackage{animate}
\usepackage{amsbsy}
\usepackage{amsfonts}
\usepackage{amsmath}
\usepackage{amssymb}
\usepackage{amsthm}
\setbeamertemplate{theorems}[numbered] % to number
\usepackage[toc,page,title,titletoc]{appendix}
\usepackage{dsfont}
\usepackage{esvect}
\usepackage[labelfont=bf]{caption}
%\usepackage{subcaption}
\usepackage{float}
\usepackage[Glenn]{fncychap}%Sonny %Conny %Lenny %Glenn %Renje %Bjarne %Bjornstrup
\usepackage{graphicx}
\usepackage{subfig}
\usepackage{indentfirst}%Para indentar os paragrafos automáticamente
\usepackage{lipsum}
\usepackage{longtable}
\usepackage{mathtools}
\usepackage{listings}%Inserir codigo do R no latex
\usepackage{multirow}
\usepackage{multicol}
\usepackage{csquotes}
\usepackage[citestyle=authoryear,maxcitenames=2,terseinits=true,natbib=true, style=abnt, maxbibnames=99]{biblatex}
\addbibresource{../Referencias/Referencias.bib}
\usepackage[figuresright]{rotating}
\usepackage{spalign}
\usepackage{pgfplots}
\pgfplotsset{compat=1.17}
\usepackage{tikz}
\usepackage{fontawesome}
\usepackage{color, colortbl}
\usepackage{url}
\usepackage{cancel}
\usepackage{accents}
\usepackage{bm}
\usepackage{ragged2e}%para justificar o texto dentro de algum ambiente
\definecolor{Gray}{gray}{0.9}
\definecolor{LightCyan}{rgb}{0.88,1,1}


\usepackage[all]{xy}
\usepackage{hyperref,bookmark}
\hypersetup{
  colorlinks=true,
  linkcolor=blue,
  citecolor=red,
  filecolor=blue,
  urlcolor=blue,
}

\usetheme{Madrid}
\usecolortheme[RGB={193,0,0}]{structure}

%\setbeamertemplate{footline}[frame number]
%\setbeamertemplate{footline}[text line]{%
%  \parbox{\linewidth}{\vspace*{-8pt}\hfill\date{}\hfill\insertshortauthor\hfill\insertpagenumber}}
\beamertemplatenavigationsymbolsempty
\renewcommand{\vec}[1]{\mbox{\boldmath$#1$}}
\newtheorem{Teorema}{Teorema}
\newtheorem{Proposicao}{Proposição}
\newtheorem{definicao}{Definição}
\newtheorem{Corolario}{Corolário}
\newtheorem{Demonstracao}{Demonstração}
\newcommand{\bx}{\ensuremath{\bar{x}}}
\newcommand{\Ho}{\ensuremath{H_{0}}}
\newcommand{\Hi}{\ensuremath{H_{1}}}
\newcommand{\at}[2][]{#1|_{#2}}
\newcommand\xuparrow[1][2ex]{%
   \mathrel{\rotatebox{-90}{$\xleftarrow{\rule{#1}{0pt}}$}}
}
\apptocmd{\frame}{}{\justifying}{} % Allow optional arguments after frame.

\makeatletter
\setbeamertemplate{footline}
{
  \leavevmode%
  \hbox{%
  \begin{beamercolorbox}[wd=.3\paperwidth,ht=2.25ex,dp=1ex,center]{author in head/foot}%
    \usebeamerfont{author in head/foot}\mytext
  \end{beamercolorbox}%
  \begin{beamercolorbox}[wd=.3\paperwidth,ht=2.25ex,dp=1ex,center]{title in head/foot}%
    \usebeamerfont{title in head/foot}\mytextt
  \end{beamercolorbox}%
  \begin{beamercolorbox}[wd=.35\paperwidth,ht=2.25ex,dp=1ex,right]{site in head/foot}%
    \usebeamerfont{site in head/foot}\mytexttt\hspace*{2em}
    \insertframenumber{} / \inserttotalframenumber\hspace*{2ex} 
  \end{beamercolorbox}}%
  \vskip0pt%
}
\makeatother

\providecommand{\arcsin}{} \renewcommand{\arcsin}{\hspace{2pt}\textrm{arcsen}}
\providecommand{\sin}{} \renewcommand{\sin}{\hspace{2pt}\textrm{sen}}
\newcommand{\N}{\rm I\!N}
\newcommand{\I}{\rm I\!I}
\newcommand{\R}{\rm I\!R}
\newcommand{\Sim}{\overset{\text{iid}}{\sim}}
\newcommand{\Lim}{{\displaystyle \lim_{n\to\infty}}}
\newcommand{\LimInf}{{\displaystyle \liminf_{n\to\infty}}}
\newcommand{\rightLim}{\xrightarrow[n\rightarrow\infty]{}}
\newcommand{\Sumi}{{\displaystyle \sum_{i=1}^{n}}}
\newcommand{\Int}{{\displaystyle \int_{-\infty}^{+\infty}}}
\newcommand{\ConvD}{\overset{D}{\rightarrow}}
\newcommand{\ConvP}{\overset{P}{\rightarrow}}
\newcommand{\Prodi}{{\displaystyle \prod_{i=1}^{n}}}
\newcommand{\SetaUP}[2]{\underset{\mathclap{\substack{\xuparrow[30pt] \\ #1}}}{#2}}
%\newcommand{\SetaInclinada}[2]{\underset{\mathclap{\substack{\rotatebox{135}{\xuparrow[30pt] \\ #1}}}}{#2}}
\newcommand{\Home}{\begin{tikzpicture}
\node[scale=2] at (3,4) {\text{Para}~\faHome};
\end{tikzpicture}}
\newcommand{\vecX}{\boldsymbol{X}}
\newcommand{\Implica}[1]{\xRightarrow{#1}}
\newcommand{\SeSe}{\iff}
\newcommand{\EscoreA}{\dfrac{\partial}{\partial\theta}\log{f(x,\theta)}}
\newcommand{\EscoreB}{\dfrac{\partial^{2}}{\partial\theta^{2}}\log{f(x,\theta)}}
\newcommand{\cqd}{\text{cqd}~\blacksquare}
\newcommand{\seqX}{$X_{1},\ldots,X_{n}$}
\newcommand{\seqY}{$Y_{1},\ldots,Y_{n}$}
\newcommand{\tend}[1]{\hbox{\oalign{$\bm{#1}$\crcr\hidewidth$\scriptscriptstyle\bm{\sim}$\hidewidth}}}

%\newtheorem{Teorema}{Teorema}
%\newtheorem{Proposicao}{Proposição}
%\newtheorem{Definicao}{Definição}
%\newtheorem{Corolario}{Corolário}
%\newtheorem{Demonstracao}{Demonstração}

\titlegraphic{\hspace*{8cm}\href{https://fsbmat-ufv.github.io/}{\includegraphics[width=2cm]{figs/mylogo.png}}
}

%Continuar a numeracao em slides diferentes
\newcounter{saveenumi}
\newcommand{\seti}{\setcounter{saveenumi}{\value{enumi}}}
\newcommand{\conti}{\setcounter{enumi}{\value{saveenumi}}}

\resetcounteronoverlays{saveenumi}

% Layout da pagina
\hypersetup{pdfpagelayout=SinglePage}

%Para o \pause funcionar dentro do ambiente align
\makeatletter
\let\save@measuring@true\measuring@true
\def\measuring@true{%
  \save@measuring@true
  \def\beamer@sortzero##1{\beamer@ifnextcharospec{\beamer@sortzeroread{##1}}{}}%
  \def\beamer@sortzeroread##1<##2>{}%
  \def\beamer@finalnospec{}%
}
\makeatother


\title{Inferência Estatística II}
\author{Prof. Fernando de Souza Bastos\texorpdfstring{\\ fernando.bastos@ufv.br}{}}
\institute{Departamento de Estatística\texorpdfstring{\\ Programa de Pós-Graduação em Estatística Aplicada e Biometria}\texorpdfstring{\\ Universidade Federal de Viçosa}{}\texorpdfstring{\\ Campus UFV - Viçosa}{}}
\date{}
\newcommand\mytext{Aula 4}
\newcommand\mytextt{Fernando de Souza Bastos}
\newcommand\mytexttt{\url{https://est711.github.io/}}


\begin{document}
%\SweaveOpts{concordance=TRUE}

\frame{\titlepage}

\begin{frame}{}
\frametitle{\bf Sumário}
\tableofcontents
\end{frame}

\section{Teorema Central do Limite}
\begin{frame}{Introdução}
	\begin{block}{}
		\justifying
O Teorema Central do Limite (TCL) é um pilar fundamental na Estatística, Probabilidade e Ciência de Dados, pois estabelece que a soma de um grande número de variáveis aleatórias independentes e identicamente distribuídas tende a se aproximar de uma distribuição normal, independentemente da forma original da distribuição dessas variáveis. Este teorema não apenas justifica a aplicação de métodos estatísticos que assumem normalidade, mas também permite a utilização de técnicas analíticas e inferenciais robustas para uma ampla gama de problemas práticos. 		
	\end{block}
\end{frame}

\begin{frame}{Introdução}
	\begin{block}{}
		\justifying
Em Estatística, o TCL fundamenta a construção de intervalos de confiança e a realização de testes de hipóteses, possibilitando a generalização de resultados e a tomada de decisões com base em amostras. Na Ciência de Dados, ele é crucial para a modelagem e previsão, facilitando a análise de grandes volumes de dados e a implementação de algoritmos que dependem de pressupostos normais. Em suma, o Teorema Central do Limite não só proporciona uma base teórica sólida, mas também une teoria e prática, permitindo a solução eficaz de problemas complexos e a interpretação confiável de resultados em diversas disciplinas.		
	\end{block}
\end{frame}

\begin{frame}{Teorema Central do Limite}
\begin{block}{}
\justifying
Seja $\{X_{n}\}_{n\geq 1}$ uma sequência de variáveis aleatórias iid com média $\mu$ e variância $\sigma^{2}<\infty.$ Então,
\begin{align*}
   \dfrac{\Sumi X_{i}-n\mu}{\sqrt{n}\sigma}=\dfrac{\sqrt{n}(\Bar{X}_{n}-\mu)}{\sigma} \ConvD N(0,1)
\end{align*}
\end{block}
\end{frame}

\begin{frame}{Demonstração}
\begin{block}{}
\justifying
Assuma, sem perda de generalidade, $\mu=0$ e $\sigma^{2}=1.$
\begin{align*}
    M_{\frac{\sum X_{i}-n\mu}{\sqrt{n}\sigma}}(t)&=M_{\frac{\sum X_{i}}{\sqrt{n}}}(t)
    =E\Big(e^{\frac{t\sum X_{i}}{\sqrt{n}}}\Big)=E\Big(\Prodi e^{\frac{tX_{i}}{\sqrt{n}}}\Big)\\
    &=\Prodi E\Big(e^{\frac{tX_{i}}{\sqrt{n}}}\Big)=\Prodi M_{X_{i}}\big(\frac{t}{\sqrt{n}}\big)=\Big(M_{X_{i}}(\dfrac{t}{\sqrt{n}})\Big)^{n}
\end{align*}
\end{block}
\pause
\begin{block}{}
Notemos que, $\ln{M_{\frac{\sum X_{i}}{\sqrt{n}}}(t)}=n\ln{M_{X_{i}}(\frac{t}{\sqrt{n}})}=\dfrac{\ln{M_{X_{i}}(\frac{t}{\sqrt{n}})}}{\frac{1}{n}}.$ Além disso, sabemos que $\Lim \ln{[f(n)]}=\ln{\Lim f(n)},$ sempre que $\Lim f(n)>0$ e que para $n$ grande $\frac{t}{\sqrt{n}}\approx 0~\text{e}~M_{X_{i}}(0)=1.$
\end{block}
\end{frame}

\begin{frame}{}
\begin{block}{}
\justifying
Aplicando L'Hôpital, temos:
\begin{align*}
    \Lim \frac{\ln{M_{X_{i}}(\frac{t}{\sqrt{n}})}}{\frac{1}{n}}&=\Lim \dfrac{\frac{M_{X_{i}}^{'}(\frac{t}{\sqrt{n}})}{M_{X_{i}}(\frac{t}{\sqrt{n}})}t(-\frac{1}{2})n^{-\frac{3}{2}}}{-\frac{1}{n^{2}}}\\
    &=\dfrac{t}{2}\Lim \sqrt{n}\dfrac{M_{X_{i}}^{'}(\frac{t}{\sqrt{n}})}{M_{X_{i}}(\frac{t}{\sqrt{n}})}\\
    &=\star
    %&=\dfrac{t}{2}\Lim \dfrac{M_{X_{i}}^{'}(\frac{t}{\sqrt{n}})}{\frac{1}{\sqrt{n}}}\\
    %&=\dfrac{t}{2}\Lim \dfrac{M_{X_{i}}^{''}(\frac{t}{\sqrt{n}})t(-1/2)n^{-3/2}}{-\frac{1}{2}n^{-3/2}}\\
    %&=\dfrac{t^{2}}{2}\Lim M_{X_{i}}^{''}(\frac{t}{\sqrt{n}})=\dfrac{t^{2}}{2}
\end{align*}
Sabe-se que o $k$-ésimo momento de $X_{i}$ é dado por $M_{X_{i}}^{(k)}(0)=E(X_{i}^{k}),$ ou seja,
$$M_{X_{i}}^{(')}(0)=E(X_{i})=\mu=0~\text{e}~M_{X_{i}}^{('')}(0)=E(X_{i}^{2})=\sigma^{2}=1~(\text{por hipótese}) $$
%
\end{block}
\end{frame}

\begin{frame}{}
	\begin{block}{}
		\justifying
		Logo,
		\begin{align*}
		\star&=\dfrac{t}{2}\Lim \sqrt{n}\dfrac{M_{X_{i}}^{'}(\frac{t}{\sqrt{n}})}{M_{X_{i}}(\frac{t}{\sqrt{n}})}\\
		&= \dfrac{t}{2}\Lim \dfrac{\dfrac{M_{X_{i}}^{'}(\frac{t}{\sqrt{n}})}{M_{X_{i}}(\frac{t}{\sqrt{n}})}}{\dfrac{1}{\sqrt{n}}}
		\end{align*}
	\end{block}
	\pause
	\begin{block}{}
		\justifying
		aplicando L'Hôpital novamente, temos:
		\begin{align*}
 \star&=\dfrac{t}{2}\Lim
 \dfrac{\dfrac{M_{X_{i}}(\frac{t}{\sqrt{n}})M_{X_{i}}^{''}(\frac{t}{\sqrt{n}})(-\frac{1}{2}tn^{-\frac{3}{2}})-(M_{X_{i}}^{'}(\frac{t}{\sqrt{n}}))^{2}(\frac{1}{2}tn^{-\frac{3}{2}})}{\left(M_{X_{i}}(\frac{t}{\sqrt{n}})\right)^{2}}}{-(\frac{1}{2})n^{-\frac{3}{2}}}
		\end{align*} 
	\end{block}
\end{frame}

\begin{frame}{}
	\begin{block}{}
		\justifying
		Simplificando os termos $-(\frac{1}{2})n^{-\frac{3}{2}}$ e observando que para $n$ suficientemente grande, $\dfrac{t}{\sqrt{n}}\approx 0~\Rightarrow ~ M_{X_{i}}(0)=1,  M_{X_{i}}^{(')}(0)=0$ e $M_{X_{i}}^{('')}(0)=1,$ temos: 
		\begin{align*}
		\dfrac{t}{2}\Lim	\dfrac{\dfrac{M_{X_{i}}(\frac{t}{\sqrt{n}})M_{X_{i}}^{''}(\frac{t}{\sqrt{n}})(-\frac{1}{2}tn^{-\frac{3}{2}})-(M_{X_{i}}^{'}(\frac{t}{\sqrt{n}}))^{2}(\frac{1}{2}tn^{-\frac{3}{2}})}{\left(M_{X_{i}}(\frac{t}{\sqrt{n}})\right)^{2}}}{-(\frac{1}{2})n^{-\frac{3}{2}}}&=\dfrac{t^{2}}{2}
		\end{align*}
	\end{block}
\pause
\begin{block}{}
Portanto, $\Lim M_{\frac{\sum X_{i}}{\sqrt{n}}}(t)=\exp{\Lim\ln{M_{\frac{\sum X_{i}}{\sqrt{n}}}(t)}}=\exp{\left(\frac{t^{2}}{2}\right)},$ ou seja, $\dfrac{\sum X_{i}}{\sqrt{n}}\ConvD X\sim N(0,1).$
\end{block}
\end{frame}

\section{Método Delta}
\begin{frame}{Método Delta}
\begin{block}{}
\justifying
Suponha que conhecemos a distribuição de uma variável aleatória, mas que queremos determinar a distribuição de uma função dela. Isso também é verdade na teoria assintótica, o teorema de Slutsky's e o teorema visto em aula, imediatamente anterior a ele, são ilustrações disso. Outro resultado desse tipo é chamado de método delta. Vejamos o próximo slide!
\end{block}
\end{frame}

%\begin{frame}{}
%\begin{Teorema}
%\justifying   
%Suponha que g(x) seja diferenciável em x. Então podemos escrever
%$$g(y) = g(x) + g'(x)(y - x) + o(|y - x|).$$ Em que, a notação $o$ significa que $a = o(b)$ se e somente se $\frac{a}{b} \rightarrow 0$, conforme $b \rightarrow 0$.
%\end{Teorema}
%\end{frame}

\begin{frame}{Método Delta}
\begin{Teorema}
\justifying
Seja $\{X_{n}\}_{n\geq 1}$ uma sequência de variáveis aleatórias satisfazendo a seguinte convergência em distribuição:
$$\sqrt{n}(X_{n}-\theta)\ConvD N(0,\sigma^{2}).$$ Se $g$ é uma função diferenciável em $\theta$ e $g'(\theta)\neq 0,$ então $$\sqrt{n}(g(X_{n})-g(\theta))\ConvD N(0, [g'(\theta)]^{2}\sigma^{2}).$$
\end{Teorema}
\end{frame}

\begin{frame}{Demonstração}
\begin{block}{}
\justifying
Vamos mostrar primeiro que $X_{n}\ConvP \theta.$ Seja $\varepsilon>0$ e $m>0$ inteiro ($m\in \N^{*}$) fixado. 
\begin{align*}
    P(|X_{n}-\theta|<\varepsilon)&=P(|\sqrt{n}(X_{n}-\theta)|<\varepsilon\sqrt{n})\\
    &\SetaUP{\text{para}~n\geq \Big(\dfrac{m}{\varepsilon}\Big)^{2}}{\geq} P(|\sqrt{n}(X_{n}-\theta)|<m)
\end{align*}
\end{block}
\end{frame}

\begin{frame}{}
\vspace{-0.2cm}
\begin{block}{}
\justifying
Segue que, 
\begin{align*}
    \LimInf P(|X_{n}-\theta|<\varepsilon)&\geq \LimInf P(|\sqrt{n}(X_{n}-\theta)|<m)\\
    &=\Lim P(|\sqrt{n}(X_{n}-\theta)|<m)\\
    &\SetaUP{\text{Usando o fato de que}\\ \sqrt{n}(X_{n}-\theta)\ConvD N(0,\sigma^{2})}{=}P\left(|Z|<\frac{m}{\sigma}\right),~Z\sim N(0,1)
\end{align*}
\end{block}
\pause
\vspace{-0.2cm}
\begin{block}{}
\justifying
Assim,
\begin{align*}
    \LimInf P(|X_{n}-\theta|<\varepsilon)&\geq P\left(|Z|<\frac{m}{\sigma}\right)=\Phi\left(\frac{m}{\sigma}\right)-\Phi\left(-\frac{m}{\sigma}\right)\\
    &=2\Phi\left(\frac{m}{\sigma}\right)-1
\end{align*}
\end{block}
\end{frame}

\begin{frame}{}
\begin{block}{}
\justifying
Em resumo,
\begin{align*}
    \LimInf P(|X_{n}-\theta|<\varepsilon)&\geq 2\Phi\left(\frac{m}{\sigma}\right)-1,~\forall m\in \mathbb{Z}^{*}\\  
    \LimInf P(|X_{n}-\theta|<\varepsilon)&\geq \lim_{m\rightarrow +\infty} \Big[2\Phi\left(\frac{m}{\sigma}\right)-1\Big]\\
    &=1
\end{align*}
\end{block}
\pause
\begin{block}{}
\justifying
Como $\limsup\geq \liminf,$ temos que
\begin{align}\label{num1}
    \limsup P(|X_{n}-\theta|<\varepsilon)=1\Rightarrow \Lim P(|X_{n}-\theta|<\varepsilon)=1
\end{align}
\end{block}
\pause
\begin{block}{}
\justifying
Ou seja, $X_{n}\ConvP \theta. \cqd$
\end{block}
\end{frame}

\begin{frame}{}
	\begin{block}{Série de Taylor}
		\justifying
		Agora, lembre-se que uma série de Taylor é a série de funções da forma:
		\begin{align*}
			f(x)=\displaystyle{\sum_{n=0}^{\infty}\dfrac{f^{(n)}(a)}{n!}},
		\end{align*}
Neste caso, a série acima é dita ser a série de Taylor de $f(x)$ em torno do ponto $x=a$.
	\end{block}
	\pause
	\begin{block}{}
		\justifying
		Expandindo $g$ em série de Taylor até a primeira ordem em torno de $\theta,$ temos que $$g(x)=g(\theta)+g'(\theta)(x-\theta)+\mathcal{C}(x)(x-\theta),$$ em que ${\displaystyle \lim_{x\rightarrow \theta} \mathcal{C}(x)=0}.$
	\end{block}
\end{frame}

\begin{frame}{}
\begin{block}{}
\justifying
Segue que, 
\begin{align*}
g(x)-g(\theta)&=g'(\theta)(x-\theta)+\mathcal{C}(x)(x-\theta)\\
\sqrt{n}\big(g(X_{n})-g(\theta)\big)&=\sqrt{n}(X_{n}-\theta)\big(g'(\theta)+\mathcal{C}(X_{n})\big).
\end{align*}
Para mostrar o resultado desejado, devemos mostrar que $\mathcal{C}(X_{n})\ConvP 0.$ Com isso, utilizando o teorema de Slutsky, o resultado é obtido. Como ${\displaystyle \lim_{x\rightarrow \theta} \mathcal{C}(x)=0},~\text{para}~\varepsilon>0, \exists~\delta>0$ tal que $|x-\theta|<\delta\Rightarrow~|\mathcal{C}(x)|<\varepsilon.$
\end{block}
\end{frame}

\begin{frame}{}
	\begin{block}{}
		\justifying
Ora, 

\begin{itemize}
	\item $\{|X_{n} - \theta| < \delta\}$: Este evento ocorre quando a variável aleatória $X_{n}$ está suficientemente próxima de $\theta$, ou seja, dentro de uma faixa de tamanho $\delta$ ao redor de $\theta$.
	
	\item $\{|\mathcal{C}(X_{n})| < \varepsilon\}$: Este evento ocorre quando o valor absoluto de $\mathcal{C}(X_{n})$ está dentro de uma faixa de tamanho $\varepsilon$.
\end{itemize}
Como a função $\mathcal{C}(x)$ tende a $0$ quando $x$ tende a $\theta$ ($\lim_{x \to \theta} \mathcal{C}(x) = 0$), para qualquer $\varepsilon > 0$, existe um $\delta > 0$ tal que, se $x$ estiver suficientemente próximo de $\theta$ (ou seja, se $|x - \theta| < \delta$), então $|\mathcal{C}(x)| < \varepsilon$.

Dessa forma, se $x$ pertence ao conjunto $\{|x - \theta| < \delta\}$, isso implica que $x$ também pertence ao conjunto $\{|\mathcal{C}(x)| < \varepsilon\}$, ou seja:
\[
\{|X_n - \theta| < \delta\} \subseteq \{|\mathcal{C}(X_n)| < \varepsilon\}
\]
	\end{block}
\end{frame}

\begin{frame}{}
	\begin{block}{}
		\justifying
		Daí,
		\begin{align*}
			\{|X_{n}-\theta|<\delta\}&\subset \{|\mathcal{C}(X_{n})|<\varepsilon\}\Rightarrow P(|X_{n}-\theta|<\delta)\leq P(|\mathcal{C}(X_{n})|<\varepsilon)
		\end{align*}
		Esse resultado, juntamente com $(\ref{num1})$ implica que,
		\begin{align*}
			\Lim P(|\mathcal{C}(X_{n})|<\varepsilon)=1,~\text{ou seja},~\mathcal{C}(X_{n})\ConvP 0.
		\end{align*}
	\end{block}
	\pause
	\begin{block}{}
	Portanto, como por hipótese, $\sqrt{n}(X_{n}-\theta)\ConvD N(0,\sigma^{2}),$ temos que 
	$$\sqrt{n}\big(g(X_{n})-g(\theta)\big)=\sqrt{n}(X_{n}-\theta)g'(\theta)\ConvD N(0,[g'(\theta)]^{2}\sigma^{2})$$
	\end{block}
\end{frame}

\begin{frame}{Exercício}
\begin{block}{\Home}
\justifying
Seja $X_{i}\Sim Binomial(1,p),~ i=1,\cdots,n.$ Mostre que 
\begin{align*}
    \sqrt{n}\big(\arcsin{\sqrt{\bar{X}}}-\arcsin{\sqrt{p}}\big)\ConvD N\big(0,\dfrac{1}{4}\big), \text{em que}~\bar{X}=\displaystyle{\sum_{i=0}^{n}\dfrac{X_{i}}{n}}
\end{align*}
\end{block}
\nocite{casella2021statistical} \nocite{hogg}
\end{frame}

\begin{frame}{}
\begin{block}{\Home}
\justifying
Exercícios 5.3.1 à 5.3.8, 5.3.11, 5.3.12
\end{block}
\end{frame}


\begin{frame}[allowframebreaks]
\frametitle{\bf Referências}
\printbibliography
\end{frame}


\end{document}
